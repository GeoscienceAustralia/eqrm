% Use pdflatex to compile.
\documentclass[a4paper, 12pt]{report}

\usepackage{amsmath}
\usepackage{graphicx}
\usepackage{supertabular}

% ANU Thesis page specifications
\setlength{\oddsidemargin}{-5mm}
\setlength{\evensidemargin}{-5mm}
\setlength{\topmargin}{-5mm}
\setlength{\textheight}{235mm}
\setlength{\textwidth}{155mm}

\begin{document}

\newcommand{\indexfunc}[1]{\index{#1@\texttt{#1}}\index{function!#1@\texttt{#1}}}
\newcommand{\typepar}[3]{\linebreak[2]\texttt{#1}\-\texttt{#2}\-\texttt{#3}}
\newcommand{\typeself}[3]{\linebreak[2]\texttt{#1}\-\texttt{#2}\-\texttt{#3}\index{#1#2#3@\texttt{#1#2#3}}}
\newcommand{\indexpar}[1]{\index{#1@\texttt{#1}}\index{setdata parameter@ \texttt{setdata} parameter!#1@\texttt{#1}}}
\newcommand{\typefunc}[3]{\linebreak[2]\texttt{#1}\-\texttt{#2}\-\texttt{#3}\indexfunc{#1#2#3}}

\newcommand{\manual}[1]{Manual: #1}

\LARGE \noindent \textbf{The \texttt{setdata} python file}

\vspace{1em} \normalsize The \texttt{setdata} python file is used to
control EQRM simulations. It contains a series of input parameters
that define the manner in which the EQRM is operated. For example,
there is a parameter to control whether the EQRM models hazard or
risk. Other parameters can be used to define which ground motion
model is used and whether site amplification is considered. A
description of all of the parameters in the
 \texttt{setdata} python file is given below.

\vspace{1em} Note that it not essential to supply all parameters for
each simulation. For example, if amplification is not being used
(i.e. \texttt{use\_amplification} $=$ \texttt{False}) it is not
necessary to supply the remaining amplification parameters.
Furthermore, default values are set by the EQRM for several
parameters. These are indicated in the following when applicable.
Omission of these input parameters in the \texttt{setdata} file will
lead to use of the default values. For example, the default value
for \texttt{atten\_threshold\_distance} is 400\,km.

\vspace{1em} The following also provides suggested values for
several parameters. Users are free to change these values as
desired. The developers are merely suggesting the value they would
use in most circumstances. For example, the suggested value for
\texttt{loss\_min\_pga} is 0.05.

\vspace{1em}
Finally, the term preferred is used to indicate those parameters that the developers believe to
be the most appropriate. For example, the preferred value for \typepar{csm\_}{hysteretic\_}{damping}
is \texttt{curve}. In this case the alternative choices of \texttt{None} and \texttt{curve}
would typically only be used for experimental purposes.


\vspace{2em}

\begin{tabular}{l}
\hline
\textbf{Acronyms}: \\
\hline
PSHA is probabilistic seismic hazard analysis \\
PSRA is probabilistic seismic risk analysis  \\
GMPE is ground motion prediction equation \\
PGA is peak ground acceleration (usually in units of g) \\
RSA is response spectral acceleration (usually in units of g) \\
CSM is capacity spectrum method \\
\hline
\end{tabular}

\vspace{2em}
\begin{tabular}{|p{\textwidth}|}
%---------------------------------------------------------
\hline
\vspace{0.3em} \noindent \Large \textbf{General Input:} \normalsize \\
%---------------------------------------------------------
\hline \vspace{0.1em} \texttt{run\_type}: \\
Defines the operation mode of the EQRM: \\
  \hspace{0.5em} 'hazard' $\Rightarrow$ Scenario RSA and PSHA (probabilistic hazard); \\
  \hspace{0.5em} 'risk' $\Rightarrow$ Scenario Loss and PSRA (probabilistic risk); \\
%---------------------------------------------------------
 \hline
\vspace{0.1em} \texttt{is\_scenario}: \\
Event simulation type: \\
  \hspace{0.5em} True $\Rightarrow$ a specific scenario event (Use Scenario input); \\
  \hspace{0.5em} False $\Rightarrow$ probabilistic simulation, PSHA or PSRA (Use Probabilistic input) \\
%---------------------------------------------------------
\hline \vspace{0.1em} \texttt{max\_width}: \\
Maximum width along
virtual faults i.e. synthetic rupture width can
not exceed \typepar{max\_width}{}{} (km).\\
%---------------------------------------------------------
\hline \vspace{0.1em} \texttt{site\_tag}: \\
String used in input and
output file
 names. Typically used to define the city or study of interest (e.g.
 \texttt{newc} is used in the demos).\\
%---------------------------------------------------------
\hline \vspace{0.1em} \texttt{site\_db\_tag}: \\
DEFAULT = `' \\
 String used to specify the
 exposure or building data base.  The file name is \texttt{sitedb\_[site\_tag][site\_db\_tag].csv}\\
%---------------------------------------------------------
\hline \vspace{0.1em} \texttt{return\_periods}: \\
List whose
elements represent the
return periods to be considered for PSHA. \\
%---------------------------------------------------------
\hline \vspace{0.1em} \texttt{input\_dir}: \\
Directory containing any local input files.\\
%---------------------------------------------------------
\hline \vspace{0.1em} \texttt{output\_dir}: \\
Directory for output files.    \\
%---------------------------------------------------------
    \hline \vspace{0.1em} \texttt{use\_site\_indexes}: \\
DEFAULT = \texttt{False} \\
 \hspace{0.5em} True $\Rightarrow$ sample sites with indices in
\typepar{site\_indexes}{}{} (for testing simulations);  \\
 \hspace{0.5em} \texttt{False} $\Rightarrow$ No sub-sampling. \\
%---------------------------------------------------------
\hline \vspace{0.1em} \texttt{site\_indexes}: \\
List whose elements represent the site
 indices to be used (if $\typepar{use}{\_site}{\_indexes}=\texttt{True}$). The
 index of the first row of data  (i.e. first data row in file) is 1.\\
 \hline
 \end{tabular}

\vspace{2em}
\begin{tabular}{|p{\textwidth}|}
%---------------------------------------------------------
\hline
\vspace{0.3em} \noindent \Large \textbf{Scenario Input:} \normalsize \\
%---------------------------------------------------------
\hline \vspace{0.1em} \texttt{scenario\_azimuth}: \\
Azimuth of the scenario event (degrees from true North).   \\
%---------------------------------------------------------
\hline \vspace{0.1em} \texttt{scenario\_latitude}: \\
Latitude of rupture centroid. \\
%---------------------------------------------------------
\hline \vspace{0.1em} \texttt{scenario\_longitude}: \\
Longitude of rupture centroid.    \\
%---------------------------------------------------------
\hline \vspace{0.1em} \texttt{scenario\_magnitude}: \\
 Moment magnitude of event.    \\
%---------------------------------------------------------
\hline \vspace{0.1em} \texttt{scenario\_depth}: \\
Depth to event centroid (km).    \\
%---------------------------------------------------------
\hline \vspace{0.1em} \texttt{scenario\_dip}: \\
Dip of rupture plane (degrees from horizontal).  \\
%---------------------------------------------------------
\hline \vspace{0.1em} \texttt{scenario\_number\_of\_events}: \\
 The desired number of
copies of the event to be generated. Typically copies are taken
if random sampling is used to incorporate aleatory uncertainty
in GMPE (i.e. \typepar{atten\_}{variability\_}{method}$=2$),
amplification (i.e. \typepar{amp\_}{variability\_}{method}$=2$)
or the CSM (\typepar{csm\_}{variability\_}{method}$=3$)\\
\hline
%---------------------------------------------------------
 \end{tabular}


\vspace{2em}
\begin{tabular}{|p{\textwidth}|}
%---------------------------------------------------------
\hline
\vspace{0.3em} \noindent \Large \textbf{Probabilistic input:} \normalsize \\
%---------------------------------------------------------
\hline \vspace{0.1em} \texttt{prob\_azimuth\_in\_zones}: \\
Predominant azimuth of events. Note that this can be
a single value (used in all source zones) or a vector with different elements for each source zone.\\
%---------------------------------------------------------
\hline \vspace{0.1em} \texttt{prob\_delta\_azimuth\_in\_zones}: \\
Azimuth range for events (i.e. \texttt{azimuth} $\pm$
\typepar{delta\_azimuth}{}{}). Note that this can be a single value
(used in all source zones) or
a list with different elements for each source zone.  \\
%---------------------------------------------------------
\hline \vspace{0.1em} \texttt{prob\_number\_of\_events\_in\_zones}: \\
 List whose elements represent the
desired number of simulated evenst for each generation zone.\\
%---------------------------------------------------------
\hline \vspace{0.1em} \texttt{prob\_dip\_in\_zones}: \\
Dip of simulated events.  Note that this can be a single
 value (used in all source zones) or a list with different dips for each source zone.  \\
%---------------------------------------------------------
\hline
\vspace{0.1em} \texttt{prob\_min\_mag\_cutoff}: \\
Minimum magnitude below which hazard is not considered.     \\
%---------------------------------------------------------
\hline \vspace{0.1em} \texttt{prob\_number\_of\_mag\_sample\_bins}: \\
SUGGESTED $=$ 15 \\
Number of magnitude bins used to sample
the magnitude recurrence probability density function. \\
\hline
%---------------------------------------------------------
 \end{tabular}

\vspace{2em}
\begin{tabular}{|p{\textwidth}|}
%---------------------------------------------------------
\hline
\vspace{0.3em} \noindent \Large \textbf{Ground Motion Input:} \normalsize \\
%---------------------------------------------------------
\hline \vspace{0.1em} \texttt{atten\_models}: \\
A list of the ground motion models (or GMPEs) to be used. Current choices are; \\
 \hspace{0.5em} \texttt{`Gaull\_1990\_WA'} $\Rightarrow$ Gaull \textit{et al.} (1990); \\
 \hspace{0.5em}  \texttt{`Toro\_1997\_midcontinent'}  $\Rightarrow$ Toro \textit{et al.} (1997); \\
 \hspace{0.5em}  \texttt{`Atkinson\_Boore\_97'}  $\Rightarrow$ Atkinson \textit{et al.} (1997); \\
 \hspace{0.5em}  \texttt{`Sadigh\_97'}  $\Rightarrow$ Sadigh \textit{et al.} (1997); \\
 \hspace{0.5em}  \texttt{`Youngs\_97\_interface'}  $\Rightarrow$ Youngs \textit{et al.} (1997) interface ($Z_T$=0); \\
 \hspace{0.5em}  \texttt{`Youngs\_97\_intraslab'}  $\Rightarrow$ Youngs \textit{et al.} (1997) intraslab ($Z_T$=1); \\
 \hspace{0.5em}  \texttt{`Combo\_Sadigh\_Youngs\_M8'}  $\Rightarrow$ combined Youngs \textit{et al.} (1997) and Sadigh \\
 \hspace{14.5em} \textit{et al.} (1997); \\
 \hspace{0.5em}  \texttt{`Boore\_08'}  $\Rightarrow$ Boore \textit{et al.} (2008); \\
 \hspace{0.5em} \texttt{`Sommerville\_Yilgarn'} $\Rightarrow$ Sommerville (2009) Yilgarn Craton; \\
  \hspace{0.5em} \texttt{`Sommerville\_Non\_Cratonic'} $\Rightarrow$ Sommerville (2009) Average Non Cratonic \\
  \hspace{14.5em}  model. \\
%---------------------------------------------------------
\hline \vspace{0.1em} \texttt{atten\_model\_weights}: \\
DEFAULT $=$ [1] \\
A list of the weights to be applied to each of the GMPEs in
\texttt{atten\_models}.  Note, $\sum  atten\_model\_weights=1$. \\
%---------------------------------------------------------
\hline \vspace{0.1em}
\texttt{atten\_aggregate\_Sa\_of\_atten\_models}: \\
DEFAULT $=$ \texttt{False} \\
Set to \texttt{True} to collapse the surface acceleration's when
multiple GMPEs are used.\\
%---------------------------------------------------------
\hline \vspace{0.1em} \texttt{atten\_variability\_method}: \\
DEFAULT  = 2 \\
 Technique used to
incorporate GMPE aleatory uncertainty: \\
 \hspace{0.5em} None $\Rightarrow$ No sampling; \\
 \hspace{0.5em} 2 $\Rightarrow$ random sampling; \\
 \hspace{0.5em} 3 $\Rightarrow$ $+2\sigma$; \\
 \hspace{0.5em} 4 $\Rightarrow$ $+\sigma$; \\
 \hspace{0.5em} 5 $\Rightarrow$ $-\sigma$; \\
 \hspace{0.5em} 6 $\Rightarrow$ $-2\sigma$.\\
%---------------------------------------------------------
\hline \vspace{0.1em} \texttt{atten\_periods}: \\
Periods for $S_a$. Values must ascend. \\
%---------------------------------------------------------
\hline \vspace{0.1em} \texttt{atten\_threshold\_distance}: \\
DEFAULT $=$ 400 \\
Threshold distance beyond which motion is assigned to zero (km). \\
%---------------------------------------------------------
\hline \vspace{0.1em} \texttt{atten\_override\_RSA\_shape}: \\
DEFAULT = \texttt{None} \\
Use GMPE for PGA only and change shape of RSA as follows\\
 \hspace{0.5em} \texttt{`Aust\_standard\_Sa'} $\Rightarrow$
use RSA shape from Australian earthquake loading \\
\hspace{11em} standard; \\
 \hspace{0.5em} \texttt{`HAZUS\_Sa'}  $\Rightarrow$ use RSA shape defined by HAZUS;\\
\hline
%---------------------------------------------------------
\end{tabular}

\begin{tabular}{|p{\textwidth}|}
\hline \vspace{0.1em}
\texttt{atten\_cutoff\_max\_spectral\_displacement}: \\
DEFAULT = \texttt{False} \\
\hspace{0.5em} \texttt{True} $\Rightarrow$ cutoff maximum spectral displacement. \\
\hspace{0.5em} \texttt{False} $\Rightarrow$ no cutoff applied to spectral displacement. \\
%---------------------------------------------------------
\hline \vspace{0.1em} \texttt{atten\_pga\_scaling\_cutoff}: \\
DEFAULT = 2 \\
The maximum acceptable PGA in units g. $S_a$ at all periods re-scaled accordingly.      \\
%---------------------------------------------------------
\hline \vspace{0.1em}
\texttt{atten\_smooth\_spectral\_acceleration}: \\
DEFAULT = \texttt{False} \\
\hspace{0.5em} \texttt{True} $\Rightarrow$  Smooth RSA; \\
\hspace{0.5em} \texttt{False} $\Rightarrow$  No smoothing applied to RSA. \\
%---------------------------------------------------------
\hline \vspace{0.1em} \texttt{atten\_log\_sigma\_eq\_weight}: \\
DEFAULT = 0 \\
Used to separate the  total GMPE aleatory uncertainty or
$\sigma_{total}$ into intra- $\sigma_{intra}$  and inter-
$\sigma_{inter}$ event components.
 More specifically, \typepar{atten\_log\_sigma\_eq\_weight}{}{}
 represents the fraction of $\sigma_{total}$ associated with
 intra-event aleatory uncertainty. Similarly, \\
 1 - \typepar{atten\_log\_sigma\_eq\_weight}{}{} is the fraction of $\sigma_{total}$
 associated with the earthquake/site pair or inter-event aleatory uncertainty.
 This variable defaults to 0 if not set.  Use a value between 0 and 1.\\
 \hline
%---------------------------------------------------------
 \end{tabular}

\vspace{2em}
\begin{tabular}{|p{\textwidth}|}
\hline
\vspace{0.3em} \noindent \Large \textbf{Amplification Input:} \normalsize \\
\hline \vspace{0.1em} \texttt{use\_amplification}: \\
If set to \texttt{True} use amplification associated with the local regolith: \\
%---------------------------------------------------------
\hline \vspace{0.1em} \texttt{amp\_variability\_method}: \\
DEFAULT = 2 \\
Technique used to incorporate amplification aleatory uncertainty: \\
 \hspace{0.5em} None $\Rightarrow$ No sampling; \\
 \hspace{0.5em} 2 $\Rightarrow$ random sampling; \\
 \hspace{0.5em} 3 $\Rightarrow$ $+2\sigma$; \\
 \hspace{0.5em} 4 $\Rightarrow$ $+\sigma$; \\
 \hspace{0.5em} 5 $\Rightarrow$ $-\sigma$; \\
 \hspace{0.5em} 6 $\Rightarrow$ $-2\sigma$.\\
 \hspace{0.5em} 7 $\Rightarrow$ $-2\sigma$.\\
%---------------------------------------------------------
\hline \vspace{0.1em} \texttt{amp\_min\_factor}: \\
SUGGESTED = 0.6 \\
Minimum accepted value for amplification factor.    \\
%---------------------------------------------------------
\hline \vspace{0.1em} \texttt{amp\_max\_factor}: \\
SUGGESTED = 10000 \\
Maximum accepted value for amplification factor.   \\
\hline
%---------------------------------------------------------
\end{tabular}

\vspace{2em}
\begin{tabular}{|p{\textwidth}|}
%---------------------------------------------------------
\hline
\vspace{0.3em} \noindent \Large \textbf{Building Classes Input:} \normalsize \\
%---------------------------------------------------------
\hline \vspace{0.1em} \texttt{buildings\_usage\_classification}: \\
Building usage classification system - \texttt{'HAZUS'} or \texttt{'FCB'} \\
%---------------------------------------------------------
\hline \vspace{0.1em}
\texttt{buildings\_set\_damping\_Be\_to\_5\_percent}: \\
SUGGESTED = \texttt{False} \\
If \texttt{True} use a damping $B_e$ of $5\%$ for all building structures.\\
\hline
%---------------------------------------------------------
 \end{tabular}

\vspace{2em}
\begin{tabular}{|p{\textwidth}|}
%---------------------------------------------------------
\hline
\vspace{0.3em} \noindent \Large \textbf{Capacity Spectrum Method Input:} \normalsize \\
%---------------------------------------------------------
\hline \vspace{0.1em} \texttt{csm\_use\_variability}: \\
SUGGESTED = \texttt{True} \\
\hspace{0.5em} \texttt{True} $\Rightarrow$ use the variability method described by
\texttt{csm\_variability\_method};    \\
\hspace{0.5em} \texttt{False} $\Rightarrow$ no aleatory variability applied.  \\
%---------------------------------------------------------
\hline \vspace{0.1em} \texttt{csm\_variability\_method}: \\
SUGGESTED = 3 \\
Method used to incorporate variability in capacity curve\index{capacity curve}: \\
 \hspace{0.5em} None $\Rightarrow$ No sampling; \\
 \hspace{0.5em} 3 $\Rightarrow$ Random sampling applied to ultimate
 point only and yield point \\
 \hspace{2.5em} re-calculated to satisfy capacity curve `shape' constraint. \\
%---------------------------------------------------------
\hline \vspace{0.1em} \texttt{csm\_standard\_deviation}: \\
SUGGESTED = 0.3 \\
Standard deviation for capacity curve\index{capacity curve} log--normal PDF.      \\
%---------------------------------------------------------
\hline \vspace{0.1em} \texttt{csm\_damping\_regimes}: \\
SUGGESTED = 0 \\
 Damping multiplicative formula to be
used - \manual{Section 7.2.2}: \\
 \hspace{0.5em} 0 $\Rightarrow$ \small{PREFERRED}: use $R_a$, $R_v$, and $R_d$; \\
 \hspace{0.5em} 1 $\Rightarrow$ use $R_a$, $R_v$ and assign $R_d= R_v$; \\
 \hspace{0.5em} 2 $\Rightarrow$ use $R_v$ only and assign $R_a=R_d=R_v$. \\
%---------------------------------------------------------
\hline \vspace{0.1em} \texttt{csm\_damping\_modify\_Tav}: \\
SUGGESTED = \texttt{True} \\
Modify transition building period i.e. corner period $T_{av}$: \\
 \hspace{0.5em} \texttt{True} $\Rightarrow$ \small{PREFERRED}: modify as in HAZUS; \\
 \hspace{0.5em}  \texttt{False} $\Rightarrow$ do NOT modify. \\
%---------------------------------------------------------
\hline \vspace{0.1em} \texttt{csm\_damping\_use\_smoothing}: \\
SUGGESTED = \texttt{True} \\
Smoothing of damped curve: \\
 \hspace{0.5em} \texttt{True} $\Rightarrow$ \small{PREFERRED}: apply smoothing; \\
 \hspace{0.5em} \texttt{False} $\Rightarrow$ NO smoothing.\\
\hline
%---------------------------------------------------------
\end{tabular}

\begin{tabular}{|p{\textwidth}|}
%---------------------------------------------------------
\hline \vspace{0.1em} \texttt{csm\_hysteretic\_damping}: \\
SUGGESTED = \texttt{`curve'} \\
Technique for Hysteretic  damping: \\
 \hspace{0.5em} \texttt{None} $\Rightarrow$ no hysteretic  damping \\
 \hspace{0.5em} \texttt{`trapezoidal'} $\Rightarrow$ Hysteretic  damping via trapezoidal approximation; \\
 \hspace{0.5em} \texttt{`curve'} $\Rightarrow$ \small{PREFERRED}: Hysteretic  damping via curve fitting. \\
%---------------------------------------------------------
\hline \vspace{0.1em} \texttt{csm\_SDcr\_tolerance\_percentage}: \\
SUGGESTED = 1.0 \\
Convergence tolerance as a percentage for critical spectral
displacement in nonlinear damping calculations. \\
%---------------------------------------------------------
\hline \vspace{0.1em} \texttt{csm\_damping\_max\_iterations}: \\
SUGGESTED = 7 \\
 Maximum iterations for nonlinear damping calculations.\\
 \hline
%---------------------------------------------------------
 \end{tabular}


\vspace{2em}
\begin{tabular}{|p{\textwidth}|}
%---------------------------------------------------------
\hline
\vspace{0.3em} \noindent \Large \textbf{Loss Input:} \normalsize \\
\hline \vspace{0.1em} \texttt{loss\_min\_pga}: \\
SUGGESTED = 0.05 \\
Minimum PGA(g) below which financial loss is assigned to zero. \\
%---------------------------------------------------------
\hline
\vspace{0.1em} \texttt{loss\_regional\_cost\_index\_multiplier}: \\
SUGGESTED = 1 \\
Regional cost index multiplier to convert dollar values in building
database to desired regional and temporal (i.e. inflation) values.\\
%---------------------------------------------------------
\hline \vspace{0.1em} \texttt{loss\_aus\_contents}: \\
SUGGESTED = 0 \\
Contents value for residential buildings and salvageability after complete building damage:   \\
\hspace{0.5em} 0 $\Rightarrow$ contents value as defined in building
  database and salvageability of \\
  \hspace{2.5em} $50\%$;\\
\hspace{0.5em} 1 $\Rightarrow$ $60\%$ of contents value as defined in
building database and salvageability \\
\hspace{2.5em} of zero.\\
  \hline
%---------------------------------------------------------
 \end{tabular}

\vspace{2em}
\begin{tabular}{|p{\textwidth}|}
%---------------------------------------------------------
\hline
\vspace{0.3em} \noindent \Large \textbf{Save Input:} \normalsize \\
\hline \vspace{0.1em} \texttt{save\_hazard\_map}: \\
DEFAULT = \texttt{False} \\
\texttt{True} $\Rightarrow$ Save data for hazard maps (Use for saving PSHA results). \\
%---------------------------------------------------------
\hline \vspace{0.1em} \texttt{save\_total\_financial\_loss}: \\
DEFAULT = \texttt{False} \\
\texttt{True} $\Rightarrow$ Save total financial loss. \\
%---------------------------------------------------------
 \hline
\vspace{0.1em} \texttt{save\_building\_loss}: \\
DEFAULT = \texttt{False} \\
\texttt{True} $\Rightarrow$ Save building loss. \\
%---------------------------------------------------------
 \hline
\vspace{0.1em} \texttt{save\_contents\_loss}: \\
DEFAULT = \texttt{False} \\
 \texttt{True} $\Rightarrow$ Save contents loss. \\
%---------------------------------------------------------
\hline \vspace{0.1em} \texttt{save\_motion}: \\
DEFAULT = \texttt{False} \\
\texttt{True} $\Rightarrow$ Save RSA motion (use for saving scenario ground motion results). \\
%---------------------------------------------------------
\hline \vspace{0.1em} \texttt{save\_prob\_strucutural\_damage}: \\
DEFAULT = \texttt{False} \\
\texttt{True} $\Rightarrow$ Save structural non-cumulative
probability of being in each
damage state.  Note this is only allowed when there is one event. \\
\hline
%---------------------------------------------------------
 \end{tabular}


\end{document}

\chapter{Losses}
\label{ch:losses}

\section{Overview}
The EQRM considers two types of loss:
\begin{enumerate}
\item \textbf{direct financial loss} defined as the cost involved
in replacing damaged building components and/or contents; and
\item \textbf{social loss} defined as the number (or probability)
of casualties and injuries as a result of a simulated scenario.
\end{enumerate}
This chapter describes the direct financial and social loss
modules.

\section{Direct financial loss}

\subsection{General financial loss equations: loss for a single building}

Recall that the capacity spectrum method\index{capacity spectrum
method} assumes that each building comprises three main
components, namely a structural, non-structural drift sensitive
and non-structural acceleration sensitive component.
\sref{ch:damage} described how the damage experienced by each
building is computed separately for each of the components. It is
therefore necessary to partition the replacement cost of the
building into the replacement cost for each of the three
components. The proportion chosen for each building component is a
function of the buildings construction and usage type as well as
the usage classification system. For example
\tref{tab:grids-costproportions} illustrates the proportion of the
replacement value corresponding to a couple of different buildings
for both the HAZUS\index{building usgage!HAZUS} and
FCB\index{building usgage!FCB} classification system. 

\begin{table}
\centering \caption{Examples of costing splits for the
FCB\index{building usgage!FCB} and HAZUS usage
classification\index{building usgage!HAZUS}.}
\label{tab:grids-costproportions}  \vspace{0.8em}
\begin{tabular}{|l|l|}

 \hline
\textbf{FCB Usage Classification} &  \\
 & \\
\textit{111 -- W1BVTILE:} & \\
structural & 23.44\%\\
non-structural drift sensitive & 50.00\%\\
non-structural acceleration sensitive & 26.56\%\\
 & \\
\textit{491 -- C1MSOFT:} & \\
structural & 15.32\%\\
non-structural drift sensitive & 34.23\%\\
non-structural acceleration sensitive & 50.45\%\\
\hline
\textbf{HAZUS Usage Classification} &  \\
 & \\
\textit{RES1  -- W1BVTILE:} & \\
structural & 23.44\%\\
non-structural drift sensitive & 50.00\% \\
non-structural acceleration sensitive & 26.56\%\\
 & \\
\textit{COM5 -- URMLMETAL:} & \\
structural & 13.79\%\\
non-structural drift sensitive & 34.48\%\\
non-structural acceleration sensitive & 51.72\%\\
\hline
\end{tabular}
\end{table}

Let $P_{i \alpha}$ denote the probability of being in a damage
state $\alpha=(1,2,3,4)=(S,M,E,C)$ corresponding to Slight,
Moderate, Extensive and Complete damage, where the index
$i=(1,2,3)=(s, n_d, n_a)$ corresponds to the type of damage: drift
sensitive structural damage (s), drift sensitive non-structural
damage ($n_d$) and acceleration sensitive non-structural damage
($n_a$). Contents damage is factored in later.

We also define $L_i$ as the financial loss corresponding to the 3
building components (structural damage, $i=s=1$; drift-sensitive
non-structural damage, $i=n_d=2$; acceleration-sensitive
non-structural damage $i=n_a=3$; and $L_4$ as the financial loss
due to damage of contents.


Let $R_i$ denote the replacement cost component of the building
per unit floor area, for $i=(1,2,3)=(s,n_d,n_a)$. Thus
$R=R_1+R_2+R_3$ is the total replacement cost (per unit floor
area) of the building (excluding contents). The financial loss,
for a single building, excluding contents, is then given as the
weighted sums of the probabilities
\begin{align*}
\label{eq:this-loss}
 L_1 &= C_0\sum_{\alpha=1}^4
   f_{\alpha, 1} R_1 A P_{\alpha, 1} =
   \sum_{\alpha=1}^4 f_{\alpha, 1} g_1R A P_{\alpha, 1},\\
L_2 &= C_0 \sum_{\alpha=1}^4
   f_{\alpha, 2} R_2 A P_{\alpha, 2} =
   \sum_{\alpha=1}^4 f_{\alpha, 2} g_2R A P_{\alpha, 2}, and\\
L_3 &= C_0 \sum_{\alpha=1}^4
   f_{\alpha, 3} R_3 A P_{\alpha, 3} =
    \sum_{\alpha=1}^4 f_{\alpha, 3} g_3R A P_{\alpha, 3},
\end{align*}
where $A$ is the floor area of the building (in $\mathrm{m}^2$).
Note that $R$ is the replacement cost of the building, $f_{\alpha,
i}$ is a repair cost fraction of replacement cost for the given
damage state, $g_i$ is the damage component replacement value as a
fraction of the replacement value, and $C_0$ is a regional cost
factor\index{regional cost factor}. The total loss of the
building, excluding contents, is $L=L_1+L_2+L_3$.

Note that for percentage loss (loss divided by the value of the
building) the quantity $c_0R$ cancels.


For example, the total repair cost for a building (excluding contents)
in damage
state $\alpha=3=E$ is $f_{3,1}R_1+f_{3,2}R_2+f_{3,3}R_3$.
The probabilities $P_{\alpha, i}$ correspond to the probability
of the building component $i=(s, n_d, n_a)$ being in damage state
$\alpha=(S,M,E,C)$.

The regional cost factor\index{regional cost factor}, $C_0$, is a
normalising factor to calibrate the replacement costs if
necessary. For example in the Newcastle risk assessment
\citep{dr_Fulford02a} the HAZUS cost values were used (see
\tref{tab:replace_costs}) and converted to the Newcastle region
using $C_0= 1.4516$. In this particular case the $C_0$ was
computed by assuming that a $100\,\mathrm{m^2}$ brick veneer
residential house (RES1, W1BVTILE) had a replacement cost of
AUS\$1,000 per $\mathrm{m}^2$ (this value of \$1000 having been
obtained from NRMA web site for the NSW region). In other studies,
such as the Perth Cities case study, the replacement costs were defined for the region and
no further correction was required. Note that the parameter $C_0$
is defined by the user.

\begin{table}[p]
\centering \caption{Calculated replacement costs (AUD
$\mathrm{m^2}$) of building
  usage types.}
  \vspace{0.8em}
\label{tab:replace_costs} \small
\input tab-v-losses-repcosts
\end{table}


The repair cost factors $f_{\alpha, i}$
are the proportions of the replacement costs (for each building
component $i=(1,2,3)=(s, n_d, n_a)$  per floor area.
For structural damage,
\begin{equation}
 f_{\alpha, 1} = (2\%, 10\%, 50\%, 100\%),
\end{equation}
for non-structural damage (drift-sensitive),
\begin{equation}
 f_{\alpha, 2} = (2\%, 10\%, 50\%, 100\%),
\end{equation}
and for non-structural damage (acceleration-sensitive)
\begin{equation}
 f_{\alpha, 3} = (2\%, 10\%, 30\%, 100\%),
\end{equation}
These values are taken from \cite{dr_FEMA99b}. For example, the
repair cost for the acceleration-sensitive components in the
extensive damage state are given by the product $f_{33}R_3$, so
that if the replacement cost for complete damage is \$500 per
square metre, the repair cost would be $30/100\times\$500$.


The EQRM assumes the replacement cost components to be
independent of the construction type. There are however a few exceptions,
which have not been implemented. These are based on the tables
given in \cite{dr_FEMA99b}, in particular, Table 15.2a (page
15-12), Table 15.3 (page 15-14) and Table 15.4 (page 15). In
principle the EQRM has an option that also allows the replacement
cost components to be a function of both usage and construction
type. This functionality
attributes proportions of the building's total value to its
different components. Recall that the replacement costs for
each building is stored with the building database.

The contents damage, $L_4$, is based only on the probabilities for
acceleration sensitive non-structural damage being in Slight,
Moderate, Extensive or Complete states, and on the total contents
repair costs $R_4$ defined by the building database\index{building
database}. The cost $L_4$
is then added to the building damage cost $L$ to get the overall
loss $L^*$ which includes contents.

As with the building components, the loss for contents damage, is
expressed as a weighted sum of probabilities of the acceleration
sensitive components for each damage state,
\begin{equation}
\label{eq:vlosses-contents}
 L_{4} = C_0 \sum_{\alpha=1}^{4} f_{\alpha,4}R_4P_{\alpha 3},
\end{equation}
where $P_{\alpha 3}$ is the probability of the acceleration
sensitive component of the building being in damage state
$\alpha=(1,2,3,4)=(S,M,E,C)$ and $f_{\alpha,4}$ is the repair cost
fraction of the replacement value for the contents in damage state
$\alpha=(S,M,E,C)$. This factor is expressed as a percentage in
\tref{tab:replace_costs} and has been taken from \citet[Table
15.6, page 15-21]{dr_FEMA99b}. Furthermore, for contents damage,
\begin{equation}
 f_{\alpha 4} = (1\%, 5\%, 25\%, 50\%).
\end{equation}
These values assume that 50\%\ of the contents in complete damage
can be salvaged with similar proportions for lesser damage
\citep{dr_FEMA99b}. If the user identifies the contents as being Australian specific, the following
modifications are made as to suggested by George Walker (Aon Re):
\begin{enumerate}
\item HAZUS usage\index{building usgage!HAZUS} only: The contents
value of usage types 1 to 10, 24 and 29 (see
\tref{tab:grids-HAZUSusage}) are re-assigned to 60\% of $R_4$.
\item $f_{\alpha 4}$ is re-assigned to $(2\%, 10\%, 50\%, 100\%)$
which assumes that no contents will be salvaged from complete
damage of Australian buildings.
\end{enumerate}

\subsection{Aggregated loss and survey factors}
\label{sec:loss-surveyfacts}

Each building in the database represents a sample from its
surrounding area. There is a building survey factor associated
with each building, which represents the additional number of
buildings that are represented by the modelled building. Typically the extra buildings
are of the same type and are located in close proximity to the
modelled building. The aggregated loss is therefore the weighted sum of the losses of
each building in the database with its corresponding survey factor.


\subsection{Cutoff values}

The damage and financial loss models estimate small (but finite)
damage for very small ground accelerations. This arises due to the
asymptotic nature of the fragility curves. A cutoff value has been
implemented in the code to prevent such small values being
calculated. The cutoff is in terms of PGA and is controlled by
the user. For all those events having peak ground acceleration (PGA) values
smaller than the PGA cutoff at the building
location, the financial loss is set to zero. Typically it is
assumed that no damage worth reporting occurs for ground
accelerations smaller than 0.05g.


\section{Social losses}

The EQRM includes a module for computing the injuries and
casualties associated with a scenario simulation (see
\sref{sec:source-scenario}). The code for considering social
losses associated with probabilistic simulations has not yet been
written. 

For the Newcastle 1989 simulation the results are slightly greater
than those observed in the actual earthquake (e.g. median of 45
casualties vs 13 casualties actually recorded). Further
investigation of the model is recommended before it is widely
used. Firstly, a check on the accuracy of the distribution of the
population would be useful. Secondly, some thought should be given
to a model which distributes the population randomly. It is
expected that this will result in a greater spread of the
distribution of injuries.

Chapter 13 of \cite{dr_FEMA99b} provides a detailed description of
the methodology behind social loss calculations.



\chapter{Appendicies}

\section{A selection of GMPE formulae}
\label{app:GMPE_eqns}

This appendix discusses individual GMPE's described in the previous
version of the EQRM manual.  This does not represent all of the GMPE's
available for use within EQRM, but rather a subset of the available
models. More will come in the future.


\subsection{Toro ground motion formula}

The \citet{dr_Toro97a} ground motion relation used in the EQRM code
is based on the Joyner-Boore distance $\Rjb$ and moment-magnitude.
Note that this reference largely describes the variability and not
how the coefficients were derived. \citet{dr_EPRI93a} describes
how the coefficients were derived.

The \citet{dr_Toro97a} ground motion model divides North America
into two regions (Gulf and Mid-Continent) and can  be used with
two different measures of magnitude (local and moment). Currently
the EQRM can only use the Mid-Continent -- moment magnitude
version of the ground motion model. This version of the ground motion
model is used because the earthquake mechanism in Mid-Continent
USA is believed by some to be similar to that in Australia (i.e. both are
intraplate settings).

The ground motion formula is
\begin{equation}
\begin{split}
\mu_{log(S_a)}(T_o,r_m,\Rjb) &= C_1 + C_2(r_m-6) + C_3(r_m-6)^2 - C_4\ln(R_M) \\
       &\quad  - (C_5-C_4)\max\left[\ln\left(\frac{R_M}{100}\right),0\right] -C_6R_M,
\end{split}
\end{equation}
where
\begin{equation}
 R_M = \sqrt{ \Rjb^2 + C_7^2 }.
\end{equation}

The coefficients used in the code, $C_1,\ldots,C_7$, are functions
of $T_o$ and are tabulated in \citet[Table 2]{dr_Toro97a}.

The `uncertainty parameter' ($\sigma_{log(S_a)}$) can be
decomposed as follows:
\begin{equation}
 \sigma_{log(S_a)}(T_o,r_m,\Rjb) =
 \sqrt{\sigma_a(T_o,r_m,\Rjb)+\sigma_e(r_m)}.
\end{equation}

The aleatory or random uncertainty $\sigma_a$ can be separated
into a source dependant component $\sigma_{a,r_m}$ (see
\citet[Table 3]{dr_Toro97a}) and a path dependant
component $\sigma_{a,\Rjb}$ (see \citet[Table 4]{dr_Toro97a}) as follows:
\begin{equation}
\label{eq:attn-toro-aleat} \sigma_a =
\sqrt{\sigma_{a,r_m}^2(T_o,r_m) + \sigma_{a,\Rjb}^2(T_o,\Rjb)}.
\end{equation}
The EQRM uses the aleatory uncertainty as defined by
\eref{eq:attn-toro-aleat}.

The epistemic or knowledge uncertainty $\sigma_e$ is dependant
upon source only, that is \mbox{$\sigma_e =\sigma_{e,r_m}$} (see
\citet[Page 48]{dr_Toro97a} or
\typefunc{attn}{\_preptoro}{\_epistemic}). Note that
$\sigma_{a,r_m}$ is a function of $T_o$ and $r_m$;
$\sigma_{a,\Rjb}$ is a function of $T_o$ and $\Rjb$; and
$\sigma_{e,r_m}$ is a function of $r_m$ only.

\subsection{Gaull ground motion formula}
\label{attn:atten-formula-Gaull}


The \citet{dr_Gaull90a} ground motion relation, as used in the EQRM
code, is based on hypocentral distance $\Rhyp$ and local magnitude
$r_{m_l}$.The Gaull ground motion model can be used to compute (i)
the mean of the logarithm of PGA $\mu_{log(PGA)}$ or (ii) the mean
of the logarithm of the Modified Mercalli Intensity
$\mu_{log(\Imm)}$. The formula is based on empirical intensity
data in the Australian region and the PGA extension created using
Papua New Guinea data. For the purpose of this ground motion model
the Australian region is divided into four sub-regions, Western
Australia, Southeastern Australia, Northeastern Australia and
Indonesia. The ground motion formulae is:
\begin{equation}
lnY = \ln a-c\ln \Rhyp + br_{m_l}
\end{equation}
where $a$, $b$ and $c$ are constants, $r_{m_l}$ is the local
magnitude of the rupture and $lnY$ represents $\mu_{log(PGA)}$ or
$\mu_{log(MMI)}$ depending on the value of the constants. The
constants are tabulated in \citet[Table 4]{dr_Gaull90a}.

Note that in the case of the \citet{dr_Gaull90a} ground motion model
(PGA version) there is no need to prepare the ground motion
coefficients because the ground motion model is only defined for
$T_o=0 sec$ (PGA). To compute $S_a(T_o,r_m,R)$ (more specifically,
$\mu_{log(S_a)}$), we can extend the $PGA$ estimate using the
Australian Standard Response Spectral Acceleration
\citep{dr_Standards93a}. 

\citet[Table 4]{dr_Gaull90a} tabulate the `uncertainty
parameter(s)' ($\sigma_{log(Y)}$) and indicate that they do not
depend on $r_{m_l}$ or $\Rhyp$. Furthermore, when extending the
PGA estimate to a complete $S_a$ it is assumed that
$\sigma_{log(S_a)}$ is not a function of $T_o$ and that
\begin{equation}
\sigma_{log(S_a)}(T_o) = \sigma_{log(PGA)}.
\end{equation}
Personal discussions with Brian Gaull (2002 and 2003) have
revealed that the following values of $\sigma_{log(Y)}$; 0.7 (for
PGA) and 0.925 (for $\Imm$) should be used in preference to the
published values of 0.28 and 0.37 respectively.


\subsection{Atkinson and Boore ground motion formula}

The \cite{dr_Atkinson97a} ground motion relation, as used in the
EQRM code, is based on hypocentral distance $\Rhyp$ and moment
magnitude $r_m$. The formula is
\begin{equation}
\mu_{log(S_a)}(T_o,r_m,\Rhyp) = C_1 + C_2(r_m-6) + C_3(r_m-6)^2
-ln(\Rhyp)-C_4\Rhyp.
\end{equation}
The coefficients used in the code, $C_1,\ldots,C_7$, are functions
of $T_o$ and are tabulated in \citet[Table 1]{dr_Atkinson97a}.

The uncertainty is dependant on $T_o$ only.. It represents aleatory uncertainty,
$\sigma_a$ and no attempt is made to separate it into into source
and path components.


\subsection{Sadigh ground motion formula}

The \cite{dr_Sadigh97a} ground motion relation, as used in the EQRM
code, is based on rupture distance $\Rrup$ and moment-magnitude
$r_m$. Using the convention defined by \citet{dr_Campbell03a} (in
accompanying appendix), the ground motion formula is
\begin{equation}
\begin{split}
\mu_{log(S_a)}(T_o,r_m,\Rrup) & = C_1F +C_2 + C_3r_m
+C_4(8.5-r_m)^2.5 + \\ c_5ln(r_{rup}+C_7\exp{c_8r_m}) +
C_6ln(r_{rup}+2),
\end{split}
\end{equation}
where the parameter $F$ is hard wired in the EQRM to 1 (hence
assuming reverse faulting). The coefficients used in the code,
$C_1,\ldots,C_7$, are functions of $T_o$ and $ r_m$. They are
tabulated in \citet[Table 2]{dr_Sadigh97a}.

\cite{dr_Sadigh97a} define a magnitude $r_m$ and period $T_o$
dependant `standard error' $\sigma(r_m,T_o)$ (see \citealt[Table
3]{dr_Sadigh97a}) which can be generalised as follows
\begin{equation}
\sigma(r_m,T_o) = \left \{ \begin{array}{ll}
C_{10}-C_{11}r_m & \textrm{for $0<r_m<7.21$} \\
c_{12} & \textrm{for $r_m \geq 7.21$} \\
\end{array} \right.
\end{equation}
where the coefficients $C_{10}$, $C_{11}$ and $C_{12}$ are
functions of $T_o$, and are defined in \cite[Table
A-14]{dr_Campbell03a}. For the purpose of
consistency, $\sigma(r_m,T_o)$ is assumed to be the aleatory
uncertainty $\sigma_a$.


\subsection{Somerville ground motion formula}
The \citet{dr_Somerville01a} ground motion relation, as used in the
EQRM code, is based on Joyner--Boore distance $R_{jb}$ and moment
magnitude $r_m$. The \citet{dr_Somerville01a} ground motion model
can be used for rift or nonrift zones and for vertical or
horizontal shaking (i.e. four seperate versions). The default
setting in the EQRM is nonrift.



The ground motion formula is
\begin{equation}
\begin{split}
\mu_{log(S_a)}(T_o,r_m,\Rjb) &= C_1 + C_2(r_m-6.4) + C_3ln(R_M) + C_4(r_m-6.4))\ln(R_M) \\
       &\quad  + C_5 \Rjb+ C_7(8.5-r_m)^2,
\end{split}
\end{equation}
for $\Rjb<50$, and
\begin{equation}
\begin{split}
\mu_{log(S_a)}(T_o,r_m,\Rjb) &= C_1 + C_2(r_m-6.4) + C_3ln(R_M) + C_4(r_m-6.4))\ln(R_M) \\
       &\quad  + C_5 \Rjb + C_6(ln(R_M)-ln\sqrt{2536}) +    C_7(8.5-r_m)^2,
\end{split}
\end{equation}
for $\Rjb \geq 50$, where
\begin{equation}
 R_M = \sqrt{ \Rjb + 6^2 }.
\end{equation}

The coefficients used in the code, $C_1,\ldots,C_7$, are functions
of $T_o$ and are tabulated in \citet[Table 9]{dr_Somerville01a}.

\cite{dr_Somerville01a} describe five types of uncertainty as
follows:

\begin{supertabular}{ll}
$\sigma_{\mu}$ & uncertainty in the median ground motion = 0.2. \\
$\sigma_{\sigma}$ & uncertainty in the scatter about median ground motion = 0.15. \\
$\sigma_{modeling}(T_o)$ & representing the discrepancy between
actual
physical processes \\
  & and the simplified representation of the model.\\
$\sigma_a(T_o)$ & arising from parameters varied in the study of \\
 & \cite{dr_Somerville01a}. \\
$\sigma_b(T_o)$ & arising from earlier studies by the same
authors.\\
\end{supertabular}

The terms $\sigma_{\mu}$ and $\sigma_{\sigma}$ are clearly
epistemic uncertainty. \cite{dr_Somerville01a} then define
$\sigma_{modeling}(T_o)$, $\sigma_a(T_o)$ and $\sigma_b(T_o)$ as
contributions to the scatter and as such the aleatory uncertainty
is taken to be:
\begin{equation}
\sigma_{a,log(S_a)}(T_o) = \sqrt{
\sigma_{modeling}^2(T_o)+\sigma_a^2(T_o)+\sigma_b^2(T_o) }.
\end{equation}

\
\eject






The EQRM control file is the primary input file for an EQRM
simulation. It:
\begin{enumerate}
\item contains a series of input variables (or parameters) that define
the manner in which the EQRM is operated; and
\item initialises the simulation.
\end{enumerate}
For example, there is a parameter to control whether the EQRM models
hazard or risk. Other parameters can be used to identify return 
periods or indicate whether site amplification is considered. A
description of all of the parameters is given below.

Note that it not essential to supply all parameters for each
simulation. For example, if amplification is not being used (i.e.
\texttt{use\_amplification} $=$ \texttt{False}) it is not necessary
to supply the remaining amplification parameters. Furthermore,
default values are set by the EQRM for several parameters. These are
indicated below when applicable. Omission of these input
parameters in the EQRM control file will lead to use of the
default values. For example, the default value for
\texttt{atten\_threshold\_distance} is 400\,km.

The following also provides suggested values for several parameters.
Users are free to change these values as desired. The developers are
merely suggesting the value they would use in most circumstances.
For example, the suggested value for \texttt{loss\_min\_pga} is
0.05.

The term preferred is used to indicate those parameters
that the developers believe to be most appropriate. For example,
the preferred value for \typepar{csm\_}{hysteretic\_}{damping} is
\texttt{curve}. In this case the alternative choices of
\texttt{None} and \texttt{trapezoidal} would typically only be used for
experimental purposes.

A simulation can be started by executing an EQRM control file or by
executing \texttt{analysis.py} with a control file as the first
parameter.  


\begin{center}
\begin{tabular}{|c|}
\hline
 \\
 Two ways of running EQRM:\\
\texttt{>python EQRM\_control\_file.py}\\
OR\\
\texttt{>python analysis.py EQRM\_control\_file.py}\\
\\
\hline
\end{tabular}
\end{center}

The control file is a Python script, so Python code can be
used to manipulate parameter values.  Note though that all variables
other than the parameter values must be deleted to avoid passing
unknown variables into EQRM.

\vspace{2em}

\begin{tabular}{l}
\hline
\textbf{Acronyms}: \\
\hline
PSHA is probabilistic seismic hazard analysis \\
PSRA is probabilistic seismic risk analysis  \\
GMPE is ground motion prediction equation \\
PGA is peak ground acceleration (usually in units of g) \\
RSA is response spectral acceleration (usually in units of g) \\
CSM is capacity spectrum method \\
\hline
\end{tabular}

\begin{table}
\caption{General input for the EQRM control file.}
\label{tab:general_input-cf} \centering
\vspace{2em}
\begin{tabular}{|p{\textwidth}|}
%---------------------------------------------------------
\hline
\vspace{0.3em} \noindent \Large \textbf{General Input:} \normalsize \\
%---------------------------------------------------------
\hline \vspace{0.1em} \texttt{run\_type}: (Mandatory) \\
Defines the operation mode of the EQRM: \\
  \hspace{0.5em} `hazard' $\Rightarrow$ Scenario RSA and PSHA (probabilistic hazard); \\
  \hspace{0.5em} `risk' $\Rightarrow$ Scenario Loss and PSRA (probabilistic risk); \\
%---------------------------------------------------------
 \hline
\vspace{0.1em} \texttt{is\_scenario}: (Mandatory) \\
Event simulation type: \\
  \hspace{0.5em} True $\Rightarrow$ a specific scenario event (Use Scenario input); \\
  \hspace{0.5em} False $\Rightarrow$ probabilistic simulation, PSHA or PSRA (Use Probabilistic input) \\
%---------------------------------------------------------
\hline \vspace{0.1em} \texttt{site\_tag}: (Mandatory)\\
String used in input and
output file
 names. Typically used to define the city or region of interest (e.g.
 \texttt{newc} is used in the demos).\\
%---------------------------------------------------------
\hline \vspace{0.1em} \texttt{site\_db\_tag}: \\
DEFAULT = `' \\
 String used to specify the
 exposure or building data base.  The building data base file name is
 \texttt{sitedb\_<site\_tag><site\_db\_tag>.csv}  The exposure data base file name is
 \texttt{<site\_tag>\_par\_site<site\_db\_tag>.csv}\\
%---------------------------------------------------------
\hline \vspace{0.1em} \texttt{return\_periods}: \\
List whose
elements represent the
return periods to be considered for PSHA. \\
%---------------------------------------------------------
\hline \vspace{0.1em} \texttt{input\_dir}: \\
Directory containing any local input files.\\
%---------------------------------------------------------
\hline \vspace{0.1em} \texttt{output\_dir}: \\
Directory for output files. This directory will be created
if not present   \\
%---------------------------------------------------------
    \hline \vspace{0.1em} \texttt{use\_site\_indexes}: \\
DEFAULT = \texttt{False} \\
 \hspace{0.5em} True $\Rightarrow$ sample sites with indices in
\typepar{site\_indexes}{}{} (for testing simulations);  \\
 \hspace{0.5em} \texttt{False} $\Rightarrow$ No sub-sampling. \\
%---------------------------------------------------------
\hline \vspace{0.1em} \texttt{site\_indexes}: \\
List whose elements represent the site
 indices to be used (if $\typepar{use}{\_site}{\_indexes}=\texttt{True}$). The
 index of the first row of data  (i.e. first data row in site file) is 1.\\
 \hline
\end{tabular}
\end{table}

\vspace{2em}
\begin{tabular}{|p{\textwidth}|}
%---------------------------------------------------------
\hline \vspace{0.1em} \texttt{fault\_source\_tag}: \\
Tag for specifying a source fault file. \\
If  \texttt{fault\_source\_tag} is defined the
filename for the fault source file is
\texttt{<site\_tag>\_fault\_source\_<fault\_source\_tag>.xml}.
Otherwise it is not used. \\
Note that one of \texttt{fault\_source\_tag} and \texttt{zone\_source\_tag} must
be set. \\
\\
%---------------------------------------------------------
\hline \vspace{0.1em} \texttt{zone\_source\_tag}: \\
Tag for specifying a source zone file. \\
If  \texttt{zone\_source\_tag} is defined the
filename for the zone source file is
\texttt{<site\_tag>\_zone\_source\_<zone\_source\_tag>.xml}.
Otherwise it is not used. \\
Note that one of \texttt{zone\_source\_tag} and \texttt{fault\_source\_tag} must
be set. \\
\\
%---------------------------------------------------------
\hline \vspace{0.1em} \texttt{event\_control\_tag}: \\
Tag for specifying a event control file. \\
If  \texttt{event\_control\_tag} is defined the
filename for the event control is
\texttt{<site\_tag>\_}\texttt{event\_control\_}\texttt{<event\_control\_tag>}\texttt{.xml}
Otherwise it is not used. \\
\\
 %---------------------------------------------------------
% DR - must check if this is operational for hazard site file
%\hline \vspace{0.1em} \texttt{site\_db\_tag}: \\
%Extra tag for specifying a unique site file. That is, the filename
%for the site file will be
%\texttt{<site\_loc>\_}\texttt{par\_site}\texttt{<site\_db\_tag>}\texttt{.csv}
%\\
\hline
 \end{tabular}

\vspace{2em}
\begin{tabular}{|p{\textwidth}|}
%---------------------------------------------------------
\hline
\vspace{0.3em} \noindent \Large \textbf{Scenario Input:} \normalsize \\
%---------------------------------------------------------
\hline \vspace{0.1em} \texttt{scenario\_azimuth}: \\
Azimuth of the scenario event (degrees from true North).   \\
%---------------------------------------------------------
\hline \vspace{0.1em} \texttt{scenario\_depth}: \\
Depth to event centroid (km).    \\
%---------------------------------------------------------
\hline \vspace{0.1em} \texttt{scenario\_latitude}: \\
Latitude of rupture centroid. \\
%---------------------------------------------------------
\hline \vspace{0.1em} \texttt{scenario\_longitude}: \\
Longitude of rupture centroid.    \\
%---------------------------------------------------------
\hline \vspace{0.1em} \texttt{scenario\_magnitude}: \\
 Moment magnitude of event.    \\
%---------------------------------------------------------
\hline \vspace{0.1em} \texttt{scenario\_dip}: \\
Dip of rupture plane (degrees from horizontal).  \\
%---------------------------------------------------------
\hline \vspace{0.1em} \texttt{scenario\_max\_width}: (Optional)\\
Maximum width along virtual faults i.e. rupture width can
not exceed \typepar{scenario\_max\_width}{}{} (km).\\
%---------------------------------------------------------
\hline \vspace{0.1em} \texttt{scenario\_width}: \\
DEFAULT = \texttt{None} \\
Width of rupture centroid (km). \\
If \texttt{None}, dip, magnitude, area and \texttt{scenario\_max\_width} used 
to calculate using a Wells and Coppersmith (1994) conversion. \\
%---------------------------------------------------------
\hline \vspace{0.1em} \texttt{scenario\_length}: \\
DEFAULT = \texttt{None} \\
Length of rupture centroid (km). \\
If \texttt{None}, calculated as area/width. \\
%---------------------------------------------------------
\hline \vspace{0.1em} \texttt{scenario\_number\_of\_events}: \\
 The desired number of
copies of the event to be generated. Typically, copies are taken
if random sampling is used to incorporate aleatory uncertainty
in GMPE (i.e. \typepar{atten\_}{variability\_}{method}$=2$),
amplification (i.e. \typepar{amp\_}{variability\_}{method}$=2$)
or the CSM (\typepar{csm\_}{variability\_}{method}$=3$).\\
\hline
%---------------------------------------------------------
 \end{tabular}


\vspace{2em}
\begin{tabular}{|p{\textwidth}|}
%---------------------------------------------------------
\hline
\vspace{0.3em} \noindent \Large \textbf{Probabalistic Input:} \normalsize \\
%---------------------------------------------------------
\hline \vspace{0.1em}

All of the parameters in this section can be specified in the fault
source or zone source xml files.  Specifying them here will override
the values in the xml files.\\

%---------------------------------------------------------
\hline \vspace{0.1em} \texttt{prob\_number\_of\_events\_in\_zones}: \\
A list, where each element is the desired number of events for each
zone source.\\
%---------------------------------------------------------
\hline \vspace{0.1em} \texttt{prob\_number\_of\_events\_in\_faults}: \\
A list, where each element is the desired number of events for each
fault source.\\
%---------------------------------------------------------
\hline
 \end{tabular}
 
\vspace{2em}
\begin{tabular}{|p{\textwidth}|}
%---------------------------------------------------------
\hline
\vspace{0.3em} \noindent \Large \textbf{Ground Motion Input:} \normalsize \\
%---------------------------------------------------------
\hline \vspace{0.1em} \texttt{atten\_models}: \\ The list of GMPEs,
for the logic tree.  Specifying them here will override the values in
the xml files. This should only be used for scenario simulations. 
See  \cref{sec:event-type-control_file} for GMPE name values.\\
%---------------------------------------------------------
\hline \vspace{0.1em} \texttt{atten\_model\_weights}: \\ The list of
GMPE weights, for the logic tree.  The weights must sum to
one. Specifying them here will override the values in the xml
files.This should only be used for scenario simulations. \\
%---------------------------------------------------------
\hline \vspace{0.1em}
\texttt{atten\_collapse\_Sa\_of\_atten\_models}: \\
DEFAULT $=$ \texttt{False} \\
Set to \texttt{True} to collapse the surface acceleration's when
multiple GMPEs are used.\\
%---------------------------------------------------------
\hline \vspace{0.1em} \texttt{atten\_variability\_method}: \\
DEFAULT  = 2 \\
 Technique used to
incorporate GMPE aleatory uncertainty: \\
 \hspace{0.5em} None $\Rightarrow$ No sampling; \\
 \hspace{0.5em} 1 $\Rightarrow$ spawning; \\
 \hspace{0.5em} 2 $\Rightarrow$ random sampling; \\
 \hspace{0.5em} 3 $\Rightarrow$ $+2\sigma$; \\
 \hspace{0.5em} 4 $\Rightarrow$ $+\sigma$; \\
 \hspace{0.5em} 5 $\Rightarrow$ $-\sigma$; \\
 \hspace{0.5em} 6 $\Rightarrow$ $-2\sigma$.\\
%---------------------------------------------------------
\hline \vspace{0.1em} \texttt{atten\_periods}: \\
Periods for RSA. Values must ascend. The first value must be 0.0.\\
%---------------------------------------------------------
\vspace{0.1em} \texttt{atten\_threshold\_distance}: \\
DEFAULT $=$ 400 \\
Threshold distance (km) beyond which motion is assigned to zero. \\
%---------------------------------------------------------
\vspace{0.1em} \texttt{atten\_spawn\_bins}: \\
DEFAULT $=$ 1 \\
Number of bins created when spawning. \\
%---------------------------------------------------------
\hline \vspace{0.1em} \texttt{atten\_override\_RSA\_shape}: \\
DEFAULT = \texttt{None} \\
Use GMPE for PGA only and change shape of RSA. If `None' use RSA
as defined by GMPE, otherwise if\\
 \hspace{0.5em} \texttt{`Aust\_standard\_Sa'} $\Rightarrow$
use RSA shape from Australian earthquake loading \\
\hspace{11em} standard; \\
 \hspace{0.5em} \texttt{`HAZUS\_Sa'}  $\Rightarrow$ use RSA shape defined by HAZUS;\\
\hline
%---------------------------------------------------------
\end{tabular}

\begin{tabular}{|p{\textwidth}|}
%---------------------------------------------------------
\hline
 \hspace{0.5em} \texttt{atten\_cutoff\_max\_spectral\_displacement}: \\
DEFAULT = \texttt{False} \\
\hspace{0.5em} \texttt{True} $\Rightarrow$ cutoff maximum spectral displacement. \\
\hspace{0.5em} \texttt{False} $\Rightarrow$ no cutoff applied to spectral displacement. \\
%---------------------------------------------------------
\hline \vspace{0.1em} \texttt{atten\_pga\_scaling\_cutoff}: \\
DEFAULT = 2 \\
The maximum acceptable PGA in units g. RSA at all periods re-scaled accordingly.      \\
%---------------------------------------------------------
\hline \vspace{0.1em}
\texttt{atten\_smooth\_spectral\_acceleration}: \\
DEFAULT = \texttt{False} \\
\hspace{0.5em} \texttt{True} $\Rightarrow$  Smooth RSA; \\
\hspace{0.5em} \texttt{False} $\Rightarrow$  No smoothing applied to RSA. \\
%---------------------------------------------------------
\hline
 \end{tabular}

\vspace{2em}
\begin{tabular}{|p{\textwidth}|}
\hline
\vspace{0.3em} \noindent \Large \textbf{Amplification Input:} \normalsize \\
\hline \vspace{0.1em} \texttt{use\_amplification}: \\
If set to \texttt{True} use amplification associated with the local regolith. 
Nature of amplification varies depending on the GMPE. If GMPE has a $V_{S30}$ term
then this will be used to compute RSA on regolith. Otherwise, RSA is computed on 
bedrock and amplification factors used to transfer this to regolith surface. \\
%---------------------------------------------------------
\hline \vspace{0.1em} \texttt{amp\_variability\_method}: \\
DEFAULT = 2 \\
Technique used to incorporate amplification aleatory uncertainty: \\
 \hspace{0.5em} None $\Rightarrow$ No sampling; \\
 \hspace{0.5em} 2 $\Rightarrow$ random sampling; \\
 \hspace{0.5em} 3 $\Rightarrow$ $+2\sigma$; \\
 \hspace{0.5em} 4 $\Rightarrow$ $+\sigma$; \\
 \hspace{0.5em} 5 $\Rightarrow$ $-\sigma$; \\
 \hspace{0.5em} 6 $\Rightarrow$ $-2\sigma$.\\
 \hspace{0.5em} 7 $\Rightarrow$ $-2\sigma$.\\
%---------------------------------------------------------
\hline \vspace{0.1em} \texttt{amp\_min\_factor}: \\
SUGGESTED = 0.6 \\
Minimum accepted value for amplification factor. 
This minimum is not used for $V_{s30}$ models.   \\
%---------------------------------------------------------
\hline \vspace{0.1em} \texttt{amp\_max\_factor}: \\
SUGGESTED = 10000 \\
Maximum accepted value for amplification factor. 
This maximum is not used for $V_{s30}$ models.    \\
\hline
%---------------------------------------------------------
\end{tabular}

\vspace{2em}
\begin{tabular}{|p{\textwidth}|}
%---------------------------------------------------------
\hline
\vspace{0.3em} \noindent \Large \textbf{Building Classes Input:} \normalsize \\
%---------------------------------------------------------
\hline \vspace{0.1em} \texttt{buildings\_usage\_classification}: \\
Building usage classification system - \texttt{'HAZUS'} or \texttt{'FCB'} \\
%---------------------------------------------------------
\hline \vspace{0.1em}
\texttt{buildings\_set\_damping\_Be\_to\_5\_percent}: \\
SUGGESTED = \texttt{False} \\
If \texttt{True} a damping $B_e$ of $5\%$ will be used for all building structures.\\
\hline
%---------------------------------------------------------
 \end{tabular}

\vspace{2em}
\begin{tabular}{|p{\textwidth}|}
%---------------------------------------------------------
\hline
\vspace{0.3em} \noindent \Large \textbf{Capacity Spectrum Method Input:} \normalsize \\
%---------------------------------------------------------
\hline \vspace{0.1em} \texttt{csm\_use\_variability}: \\
SUGGESTED = \texttt{True} \\
\hspace{0.5em} \texttt{True} $\Rightarrow$ use the variability method described by
\texttt{csm\_variability\_method};    \\
\hspace{0.5em} \texttt{False} $\Rightarrow$ no aleatory variability applied.  \\
%---------------------------------------------------------
\hline \vspace{0.1em} \texttt{csm\_variability\_method}: \\
SUGGESTED = 3 \\
Method used to incorporate variability in capacity curve\index{capacity curve}: \\
 \hspace{0.5em} None $\Rightarrow$ No sampling; \\
 \hspace{0.5em} 3 $\Rightarrow$ Random sampling applied to ultimate
 point only and yield point \\
 \hspace{2.5em} re-calculated to satisfy capacity curve `shape' constraint. \\
%---------------------------------------------------------
\hline \vspace{0.1em} \texttt{csm\_standard\_deviation}: \\
SUGGESTED = 0.3 \\
Standard deviation for capacity curve\index{capacity curve} log--normal PDF.      \\
%---------------------------------------------------------
\hline \vspace{0.1em} \texttt{csm\_damping\_regimes}: \\
PREFERRED = 0 \\
 Damping multiplicative formula to be
used: \\
 \hspace{0.5em} 0 $\Rightarrow$ \small{PREFERRED}: use $R_a$, $R_v$, and $R_d$; \\
 \hspace{0.5em} 1 $\Rightarrow$ use $R_a$, $R_v$ and assign $R_d= R_v$; \\
 \hspace{0.5em} 2 $\Rightarrow$ use $R_v$ only and assign $R_a=R_d=R_v$. \\
%---------------------------------------------------------
\hline \vspace{0.1em} \texttt{csm\_damping\_modify\_Tav}: \\
PREFERRED = \texttt{True} \\
Modify transition building period i.e. corner period $T_{av}$: \\
 \hspace{0.5em} \texttt{True} $\Rightarrow$ \small{PREFERRED}: modify as in HAZUS; \\
 \hspace{0.5em}  \texttt{False} $\Rightarrow$ do NOT modify. \\
 \hline
\end{tabular}




\vspace{2em}
\begin{tabular}{|p{\textwidth}|}
%---------------------------------------------------------
\hline \vspace{0.1em} \texttt{csm\_damping\_use\_smoothing}: \\
PREFERRED = \texttt{True} \\
Smoothing of damped curve: \\
 \hspace{0.5em} \texttt{True} $\Rightarrow$ \small{PREFERRED}: apply smoothing; \\
 \hspace{0.5em} \texttt{False} $\Rightarrow$ NO smoothing.\\
\hline \vspace{0.1em} \texttt{csm\_hysteretic\_damping}: \\
PREFERRED = \texttt{`curve'} \\
Technique for Hysteretic  damping: \\
 \hspace{0.5em} \texttt{None} $\Rightarrow$ no hysteretic  damping \\
 \hspace{0.5em} \texttt{`trapezoidal'} $\Rightarrow$ Hysteretic  damping via trapezoidal approximation; \\
 \hspace{0.5em} \texttt{`curve'} $\Rightarrow$ \small{PREFERRED}: Hysteretic  damping via curve fitting. \\
%---------------------------------------------------------
\hline \vspace{0.1em} \texttt{csm\_SDcr\_tolerance\_percentage}: \\
SUGGESTED = 1.0 \\
Convergence tolerance as a percentage for critical spectral
displacement in nonlinear damping calculations. \\
%---------------------------------------------------------
\hline \vspace{0.1em} \texttt{csm\_damping\_max\_iterations}: \\
SUGGESTED = 7 \\
 Maximum iterations for nonlinear damping calculations.\\
 \hline
%---------------------------------------------------------
 \end{tabular}


\vspace{2em}
\begin{tabular}{|p{\textwidth}|}
%---------------------------------------------------------
\hline
\vspace{0.3em} \noindent \Large \textbf{Loss Input:} \normalsize \\
\hline \vspace{0.1em} \texttt{loss\_min\_pga}: \\
SUGGESTED = 0.05 \\
Minimum PGA(g) below which financial loss is assigned to zero. \\
%---------------------------------------------------------
\hline
\vspace{0.1em} \texttt{loss\_regional\_cost\_index\_multiplier}: \\
SUGGESTED = 1 \\
Regional cost index multiplier to convert dollar values in building
database to desired regional and temporal (i.e. inflation) values.\\
%---------------------------------------------------------
\hline \vspace{0.1em} \texttt{loss\_aus\_contents}: \\
SUGGESTED = 0 \\
Contents value for residential buildings and salvageability after complete building damage:   \\
\hspace{0.5em} 0 $\Rightarrow$ contents value as defined in building
  database and salvageability of \\
  \hspace{2.5em} $50\%$;\\
\hspace{0.5em} 1 $\Rightarrow$ $60\%$ of contents value as defined in
building database and salvageability \\
\hspace{2.5em} of zero.\\
  \hline
%---------------------------------------------------------
 \end{tabular}

\vspace{2em}
\begin{tabular}{|p{\textwidth}|}
%---------------------------------------------------------
\hline
\vspace{0.3em} \noindent \Large \textbf{Save Input:} \normalsize \\
\hline \vspace{0.1em} \texttt{save\_hazard\_map}: \\
DEFAULT = \texttt{False} \\
\texttt{True} $\Rightarrow$ Save data for hazard maps (Use for saving
PSHA results). Specifically spectral acceleration with respect to
location, return period and period.\\
%---------------------------------------------------------
\hline \vspace{0.1em} \texttt{save\_total\_financial\_loss}: \\
DEFAULT = \texttt{False} \\
\texttt{True} $\Rightarrow$ Save total financial loss. \\
%---------------------------------------------------------
 \hline
\vspace{0.1em} \texttt{save\_building\_loss}: \\
DEFAULT = \texttt{False} \\
\texttt{True} $\Rightarrow$ Save building loss. \\
%---------------------------------------------------------
 \hline
\vspace{0.1em} \texttt{save\_contents\_loss}: \\
DEFAULT = \texttt{False} \\
 \texttt{True} $\Rightarrow$ Save contents loss. \\
%---------------------------------------------------------
\hline \vspace{0.1em} \texttt{save\_motion}: \\
DEFAULT = \texttt{False} \\
\texttt{True} $\Rightarrow$ Save RSA motion (use for saving scenario ground
motion results).  Specifically spectral acceleration with respect to
spawning, ground motion model, recurence model, location, event and period.\\
%---------------------------------------------------------
\hline \vspace{0.1em} \texttt{save\_prob\_strucutural\_damage}: \\
DEFAULT = \texttt{False} \\
\texttt{True} $\Rightarrow$ Save structural non-cumulative
probability of being in each
damage state.  Note this is only supported for a single event. \\
\hline
%---------------------------------------------------------
 \end{tabular}
 
 \vspace{2em}
\begin{tabular}{|p{\textwidth}|}
%---------------------------------------------------------
\hline
\vspace{0.3em} \noindent \Large \textbf{Data Input:} \normalsize \\
\hline \vspace{0.1em} \texttt{event\_set\_handler}: \\
DEFAULT = \texttt{`generate'} \\
Sets the mode that the event set generator uses to produce an event set to work
on. \\
 \hspace{0.5em} \texttt{`generate'} $\Rightarrow$ Generate a new event set
 sample. \\
 \hspace{0.5em} \texttt{`save'} $\Rightarrow$ Generate a new event set sample,
 save and exit. \\
 \hspace{0.5em} \texttt{`load'} $\Rightarrow$ Load an event set (generated using
 \texttt{`save'}). \\
%---------------------------------------------------------
\hline \vspace{0.1em} \texttt{event\_set\_name}: \\
DEFAULT = \texttt{`current\_event\_set'} \\
Name used to identify the event set data. \\
For \texttt{event\_set\_handler} options: \\
 \hspace{0.5em} \texttt{`save'} $\Rightarrow$ Save event set to
 \texttt{data\_dir}/\texttt{event\_set\_name}. \\
 \hspace{0.5em} \texttt{`load'} $\Rightarrow$ Load event set from
 \texttt{data\_dir}/\texttt{event\_set\_name}. \\
%---------------------------------------------------------
\hline \vspace{0.1em} \texttt{data\_dir}: \\
Directory used to save to and load from event set data files. \\
For \texttt{event\_set\_handler} options: \\
 \hspace{0.5em} \texttt{`generate'} $\Rightarrow$ If not set default to 
 \texttt{output\_dir}. \\
 \hspace{0.5em} \texttt{`save'} $\Rightarrow$ Mandatory and must exist. \\
 \hspace{0.5em} \texttt{`load'} $\Rightarrow$ Mandatory and must exist. \\
%---------------------------------------------------------
\hline \vspace{0.1em} \texttt{data\_array\_storage}: \\
DEFAULT = \texttt{data\_dir} \\
Directory used to store internal data files. Used to reduce the memory footprint
of EQRM. \\
Note: It is recommended that this be on a fast local filesystem. \\
%---------------------------------------------------------
\hline \vspace{0.1em} \texttt{file\_array}: \\
DEFAULT = \texttt{True} \\
Turn on/off file based array support. \\
%---------------------------------------------------------
 
\hline
%---------------------------------------------------------
 \end{tabular}
 
 \vspace{2em}
\begin{tabular}{|p{\textwidth}|}
%---------------------------------------------------------
\hline
\vspace{0.3em} \noindent \Large \textbf{Log Input:} \normalsize \\
\hline \vspace{0.1em} \texttt{file\_log\_level}: \\
DEFAULT = \texttt{`debug'} \\
Level of verbosity in file log. \\
Options (in decreasing verbosity): \\
 \hspace{0.5em} \texttt{`debug'} \\
 \hspace{0.5em} \texttt{`info'} \\
 \hspace{0.5em} \texttt{`warning'} \\
 \hspace{0.5em} \texttt{`error'} \\
 \hspace{0.5em} \texttt{`critical'} \\
%---------------------------------------------------------
\hline \vspace{0.1em} \texttt{console\_log\_level}: \\
DEFAULT = \texttt{`info'} \\
Level of verbosity in console output. \\
Same options as for \texttt{log\_level}. Must be set to less than or equal
verbosity to \texttt{log\_level}. \\
%---------------------------------------------------------
 
\hline
%---------------------------------------------------------
 \end{tabular} 

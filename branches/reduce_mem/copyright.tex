%*************************************************
%     LaTeX document
%
%  This is the EQRM Copyright statement
%
%  Version 1.0    May, 2007
%
%
%
%  Copyright
%  Copyright 2001-2009
%
%  David Robinson, Trevor Dhu, Peter Row and Duncan Gray
%
%  Document based on ANUGA manual, by:
%  Stephen Roberts, Australian National University
%  Ole Nielsen, Duncan Gray, Jane Sexton, Nick Bartzis, Geoscience Australia
%
%  COPYRIGHT PAGE
%
%
%*************************************************
\documentclass{article}
\begin{document}
\vspace*{0.5in}

Copyright \copyright 2001-2009 Geoscience Australia. All rights
reserved.

\vspace*{0.2in}

Permission to use, copy, modify, and distribute this software for any
purpose without fee is hereby granted under the terms of the GNU
General Public License as published by the Free Software Foundation;
either version 3 of the License, or (at your option) any later
version, provided that this entire notice is included in all copies
of any software which is or includes a copy or modification of this
software and in all copies of the supporting documentation for such
software.

This program is distributed in the hope that it will be useful, but
WITHOUT ANY WARRANTY; without even the implied warranty of
MERCHANTABILITY or FITNESS FOR A PARTICULAR PURPOSE.  See the GNU
General Public License (http://www.gnu.org/copyleft/gpl.html) for
more details. Otherwise, you can request a copy of the GNU General
Public License by writing to the Free Software Foundation, Inc., 59
Temple Place, Suite 330, Boston, MA 02111-1307

This work was produced at Geoscience Australia funded by the Commonwealth of Australia. Neither
the Australian Government, Geoscience Australia nor any of their employees, makes any warranty,
express or implied, or assumes any liability or responsibility for
the accuracy, completeness, or usefulness of any information,
apparatus, product, or process disclosed, or represents that its use
would not infringe privately-owned rights. Reference herein to any
specific commercial products, process, or service by trade name,
trademark, manufacturer, or otherwise, does not necessarily
constitute or imply its endorsement, recommendation, or favoring by
the Australian Government or Geoscience Australia.  The views and opinions of authors expressed
herein do not necessarily state or reflect those of the Australian
Government or Geoscience Australia, and shall not be used for advertising or product
endorsement purposes.

This document does not convey a warranty, express or implied,
of merchantability or fitness for a particular purpose.

  \vspace{0.2in}

\textbf{Credits}:
\begin{itemize}
\item The EQRM was developed and is maintained by David Robinson,
  Duncan Gray, Trevor Dhu, Peter Row, Ben Cooper and Ross Wilson.
\end{itemize}


\textbf{License}:
\begin{itemize}
\item The EQRM is freely available and distributed under the terms of the GNU General Public Licence.
\end{itemize}

\newpage
\textbf{Acknowledgments}:
\begin{itemize}

\item  The authors wish to thank John Schneider, leader of the Risk
Research Group at Geoscience Australia. John's expertise in the
areas of earthquake hazard and risk analysis have proven invaluable
throughout the project. Without his ongoing support and advice the
EQRM would not be half the product that it is today.

\item This version of the EQRM has been converted from an earlier
version coded in MATLAB. A number of contributions to the MATLAB
version are worthy of particular mention. In particular;
\begin{itemize}
\item Glenn Fulford (previously employed by Geoscience Australia) is
acknowledged for his contribution to the original
damage module for the MATLAB version of the EQRM. Glenn's efforts
were instrumental in getting the first version of the EQRM
operational. Many of his ideas are utilised in the current version.
\item Andres Mendez from Aon Re in Chicago is acknowledged for
providing MATLAB software that formed the backbone of the earthquake
catalogue generation for early versions of the EQRM. This generous
contribution, as well as his high level of collaborative support,
was fundamental to the project's success. \item Bradley Horton is
appreciated for the programming support that he provided towards the
MATLAB EQRM through a number of contracts with Ceanet. \item Kurt
Pudniks and Ken Dale, both of Geoscience Australia, are also
acknowledged for their contribution towards earlier versions of the
EQRM software.
\end{itemize}

\item The authors have sought advice on specific topics from a range
of people involved in the earthquake hazard and risk fields. The
following are acknowledged for their expertise and assistance;
George Walker of Aon Re, Professor John McAneney of Risk Frontiers,
Walt Silva of Pacific Engineering and Analysis, Don Windeler and his
team at Risk Management Solutions (RMS), Mike Griffith at The
University of Adelaide, Gary Gibson of the Seismology Research
Centre and Brian Gaull from Guria Consulting.

\item A number of people at Geoscience Australia have contributed
 to the development of the EQRM. Members of the Newcastle
and Lake Macquarie project team lead by Trevor Jones are
acknowledged for their assistance in shaping the early design of the
EQRM. Mark Edwards and Ken Dale are thanked for the countless
engineering questions that they have fielded over the six year
project. Matt Hayne, previous project leader of the Risk Assessment
Methods Project is thanked for creating the productive and
supportive environment that allowed this project to flourish. EQRM
users, Cvetan Sinadinovski, Annette Patchett, Ken Dale and Augusto
Sanabria are thanked for their feedback and for continually
inspiring the incorporation of new functionality. Ole Nielsen is
acknowledged for the provision of software engineering advice and
for generally improving the coding abilities of the authors.
\end{itemize}
\end{document}

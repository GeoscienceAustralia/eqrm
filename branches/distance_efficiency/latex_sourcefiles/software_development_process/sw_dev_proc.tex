% Use pdflatex to compile.
\documentclass[a4paper, 12pt]{article}

\usepackage{amsmath}
\usepackage{graphicx}
\usepackage{supertabular}
\usepackage{hyperref}


% A couple of changes to the formatting
\setlength{\parindent}{0in}  % remove paragraph indenting
\setlength{\parskip}{1em} % put a blank line between paragraphs
%\value{10}
% \setcounter{counter}{value}




\makeatletter
\def\maketitle{%
\null
\thispagestyle{empty}%
\hrule height 0.5pt \vskip 2.5cm
\begin{center}
\normalfont
{\Large \textbf\@title\par}%
\vskip 2.5cm \normalsize \textbf{EQRM Development Team} \vskip 2.5cm
{\normalsize \textbf \@author\par}%
\vskip 4.53cm
{\normalsize \textbf \@date\par}%
\vskip 3.5cm \hrule height 0.5pt
\end{center}%
\null \clearpage }

\makeatother
\title{EQRM Software Development process}
\author{Geoscience Australia}



\begin{document}

\maketitle

\section{Introduction}
EQRM follows a test-driven development software development process. 


\section{Development Process}

\subsection{Unit Tests}

Developers create automated unit tests that define code requirements
(immediately) before writing the code itself. The unit tests contain
assertions that are either true or false. Passing the tests confirms
the code is correct.  All unit tests are part of the \texttt{test\_all.py}
automated test suite.  Only when all the tests pass can the code can
be checked in.

The regression unit tests are executed with;
 \texttt{test\_all.py}.

\subsection{Implementation Tests}

In Addition to unit tests, tests covering a simulation are also
developed.  These are called implementation tests (or Characterisation
tests).  Each implementation test checks the results from a current
simulation against the standard output, which was generated by
initially running the test.  Implementation tests are useful for
capturing when the functionality of EQRM changes, but it does not
check if the results are actually correct or not.

The regression implementation tests are executed with;
 \texttt{check\_scenarios.py}.
 


Creating implementation tests is a bit painful.  
 
\end{document}

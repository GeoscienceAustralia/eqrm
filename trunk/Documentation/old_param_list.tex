% Use pdflatex to compile.
\documentclass[a4paper, 12pt]{report}

\usepackage{amsmath}
\usepackage{graphicx}
\usepackage{supertabular}

\begin{document}

\newcommand{\indexfunc}[1]{\index{#1@\texttt{#1}}\index{function!#1@\texttt{#1}}}
\newcommand{\typepar}[3]{\linebreak[2]\texttt{#1}\-\texttt{#2}\-\texttt{#3}}
\newcommand{\typeself}[3]{\linebreak[2]\texttt{#1}\-\texttt{#2}\-\texttt{#3}\index{#1#2#3@\texttt{#1#2#3}}}
\newcommand{\indexpar}[1]{\index{#1@\texttt{#1}}\index{setdata parameter@ \texttt{setdata} parameter!#1@\texttt{#1}}}
\newcommand{\typefunc}[3]{\linebreak[2]\texttt{#1}\-\texttt{#2}\-\texttt{#3}\indexfunc{#1#2#3}}

\newcommand{\manual}[1]{Manual: #1}

\noindent \textbf{The \texttt{setdata} file}

The full path to the \typeself{set}{da}{ta} file must be passed to
the EQRM when the program is run. It contains a series of input
variables (or parameters) that define the manner in which the EQRM
is operated. For example; there is a flag to control whether the
EQRM models hazard or risk. Other flags can be used to define which
attenuation model is selected, whether site amplification is
considered or which set of engineering parameters is used. A
description of all of the variables in the \typeself{set}{da}{ta}
file is given below. The \typeself{set}{da}{ta} file contains a
single variable
\typeself{eqrm}{\_param}{\_T}\index{setdata!\texttt{eqrm\_param\_T}}.

\vspace{2em} \noindent \textbf{Operation Mode:}

\begin{supertabular}{lp{0.6\textwidth}}
\typepar{run}{\_type}{} & Defines the operation mode of the EQRM: \\
  & \hspace{0.5em} 1 $\Rightarrow$ Scenario RSA and Probabilistic Hazard; \\
  & \hspace{0.5em} 2 $\Rightarrow$ Scenario Loss and Probabilistic Risk; \\
\end{supertabular}

\vspace{2em} \noindent
 \textbf{General:}

\begin{supertabular}{lp{0.6\textwidth}}
  \typepar{inp}{ut}{dir} &  Directory containing any local input files.\\
 \typepar{sav}{e}{dir}{} &   Directory for output files.    \\
 \typepar{site}{\_}{loc}    &   String used in input and output file names.    \\
\typepar{grid}{\_}{flag}   & Define grid file to use if \typepar{run}{\_type}{}=1 (note that both files have the same format): \\
  & \hspace{0.5em} 1 $\Rightarrow$ Loads \footnotesize{\typepar{site\_loc+'\_par}{\_}{site.csv'}}; \\
  & \hspace{0.5em} 2 $\Rightarrow$ Loads \footnotesize{\typepar{site\_loc+'\_par}{\_site}{\_uniform.csv'}}.\\
\typepar{small}{\_site}{\_flag} &   Sampling of sites: \\
 & \hspace{0.5em} 0 $\Rightarrow$  use all sites; \\
 & \hspace{0.5em} 1 $\Rightarrow$  sub-sample with indices in \typepar{Si}{te}{Ind};  \\
 & \hspace{0.5em} 2 $\Rightarrow$  sub-sample with 4 hard-wired indices \\
 & \hspace{0.5em} 3 $\Rightarrow$  sub-sample with 98 hard-wired indices \\
\typepar{Si}{te}{Ind} & Vector whose elements represent the site
 indices to be used (if $\typepar{small}{\_site}{\_flag}=1$). The
 index of the first row of data is 1.\\
\typepar{de}{str}{ing}    &   Optional string to be attached to some output files (only really used when outputs are converted to MATALB *.mat format - see post processing instructions in README-getting-started.txt).    \\
\typepar{rtrn}{\_}{per}   &   Vector whose elements represent the return periods to be considered for hazard.    \\
 \end{supertabular}

\vspace{2em} \noindent \textbf{Source:}

\begin{supertabular}{lp{0.6\textwidth}}
\typepar{ntrg}{vec}{tor}  &  Vector whose elements represent the desired number of simulated event\index{simulated event}s for each generation zone.   \\
\typepar{a}{z}{i} &     Predominant azimuth of events. Note that this can be a single value (used in all source zones) or a vector with different elements for each source zone.\\
\typepar{d}{\_a}{zi}   &   Azimuth range for events (i.e. \typepar{a}{z}{i} $\pm$ \typepar{d}{\_a}{zi}). Note that this can be a single value (used in all source zones) or a vector with different elements for each source zone.  \\
\typepar{d}{i}{p} &    Dip of virtual fault\index{virtual fault}s.  Note that this can be a single value (used in all source zones) or a vector with different elements for each source zone.  \\
\typepar{w}{d}{th} & Maximum width along virtual fault i.e. synthetic rupture width can not exceed \typepar{w}{d}{th}.\\
\typepar{f}{ty}{pe}   & \small{CURRENTLY NOT OPERATIONAL} \\
\typepar{min}{\_mag}{\_cutoff}  &  Minimum magnitude below which hazard is not considered.     \\
\typepar{n}{bin}{s}   &   Number of magnitude bins used to sample
event magnitudes.    \\
 \end{supertabular}


\vspace{2em} \noindent \textbf{Event Spawn:}

\begin{supertabular}{lp{0.6\textwidth}}
\typepar{src}{\_eps}{\_switch}  &   \small{CURRENTLY NOT OPERATIONAL}  \\
\typepar{m}{bn}{d} &  \small{CURRENTLY NOT OPERATIONAL} \\
\typepar{nsa}{mpl}{es} & \small{CURRENTLY NOT OPERATIONAL}  \\
\typepar{n}{sig}{ma} &  \small{CURRENTLY NOT OPERATIONAL} \\
 \end{supertabular}


\vspace{2em} \noindent \textbf{Scenario:}

\begin{supertabular}{lp{0.6\textwidth}}
\typepar{det}{erm}{\_flag} &  Event simulation type: \\
 & \hspace{0.5em} 0 $\Rightarrow$ standard probabilistic simulation; \\
 & \hspace{0.5em} 1 $\Rightarrow$ consider a specific scenario event. \\
\typepar{det}{erm}{\_ntrg} & The desired number of copies to be generated (if $\typepar{det}{erm}{\_flag}=1$)\\
\typepar{det}{erm}{\_azi}  &   Azimuth of event (if $\typepar{det}{erm}{\_flag}=1$).   \\
\typepar{det}{erm}{\_lat}  &  Latitude of rupture centroid \\
 & \hspace{0.5em} (if $\typepar{det}{erm}{\_flag}=1$). \\
\typepar{det}{erm}{\_lon}  &  Longitude of rupture centroid    \\
 & \hspace{0.5em} (if $\typepar{det}{erm}{\_flag}=1$). \\
\typepar{det}{erm}{\_mag}  &  Moment magnitude of event    \\
 & \hspace{0.5em} (if $\typepar{det}{erm}{\_flag}=1$). \\
\typepar{det}{erm}{\_r\_z}  &  Depth to event     \\
 & \hspace{0.5em} (if $\typepar{det}{erm}{\_flag}=1$). \\
 \end{supertabular}


\vspace{2em} \noindent \textbf{Attenuation:}

\begin{supertabular}{lp{0.6\textwidth}}
\typepar{atten}{uation}{\_flag}    &  A 2 row matrix containing one
column for each attenuation model to be used.
The first row contains a pointer to the attenuation model (see below) and the second row contains its weight. Note that if the $\sum weights =-1$ the logic tree is not collapsed, whereas if $\sum weights =1$ the logic tree is collapsed (see \manual{Chapter 5})\\
&  \textbf{Attenuation Model Pointers}: \\
 & \hspace{0.5em} 0 $\Rightarrow$ Gaull \textit{et al.} (1990); \\
 & \hspace{0.5em} 1 $\Rightarrow$ Toro \textit{et al.} (1997); \\
 & \hspace{0.5em} 2 $\Rightarrow$ Atkinson \textit{et al.} (1997); \\
 & \hspace{0.5em} 3 $\Rightarrow$ Sadigh \textit{et al.} (1997); \\
 & \hspace{0.5em} 4 $\Rightarrow$ \small{CURRENTLY NOT OPERATIONAL} \\
& \hspace{0.5em} 5 $\Rightarrow$ \small{CURRENTLY NOT OPERATIONAL} \\
 & \hspace{3em} reserved for Somerville \textit{et al.} (2001) \\
 & \hspace{0.5em} 6 $\Rightarrow$ Youngs \textit{et al.} (1997); \\
 & \hspace{0.5em} 7 $\Rightarrow$ combined Youngs \textit{et al.} (1997) and Sadigh \textit{et al.} (1997); \\
 & \hspace{0.5em} 8 $\Rightarrow$ Boore \textit{et al.} (2008); \\
 & \hspace{0.5em} 9 $\Rightarrow$ Summerville (2009) Yilgarn Craton; \\
 & \hspace{3em} reserved for Allen \textit{et al.} (2006) \\
& \hspace{0.5em} 100 $\Rightarrow$ \small{CURRENTLY NOT OPERATIONAL} \\
 & \hspace{3em} reserved for Gaull \textit{et al.} (1990) (MMI). \\
\typepar{var}{\_attn}{\_flag}   &  Variability for attenuation: \\
 & \hspace{0.5em} 0 $\Rightarrow$ NOT included; \\
  & \hspace{0.5em}  1 $\Rightarrow$ included.   \\
\typepar{var}{\_attn}{\_method} & Technique used to
incorporate attenuation aleatory uncertainty: \\
 & \hspace{0.5em} 1 $\Rightarrow$ \small{\small{CURRENTLY NOT OPERATIONAL}} \\
 & \hspace{0.5em} reserved for spawning \\
 & \hspace{0.5em} 2 $\Rightarrow$ random sampling; \\
 & \hspace{0.5em} 3 $\Rightarrow$ $+2\sigma$; \\
 & \hspace{0.5em} 4 $\Rightarrow$ $+\sigma$; \\
 & \hspace{0.5em} 5 $\Rightarrow$ $-\sigma$; \\
 & \hspace{0.5em} 6 $\Rightarrow$ $-2\sigma$.\\
\typepar{resp}{\_crv}{\_flag} & Secondary options for choice of
response spectral acceleration\index{response spectral acceleration} $S_a$: \\
 & \hspace{0.5em} 0 $\Rightarrow$ $S_a$ from attenuation model; \\
 & \hspace{0.5em} 1 $\Rightarrow$ $S_a$ from attenuation model and cutoff\\
 & \hspace{2.8em} maximum spectral displacement\index{response spectral displacement}; \\
 & \hspace{0.5em} 2 $\Rightarrow$ use PGA from attenuation model to scale\\
 & \hspace{2.8em} Australia Standard $S_a$; \\
 & \hspace{0.5em} 3 $\Rightarrow$ use PGA from attenuation model to scale \\
 & \hspace{2.8em} Australia Standard and cutoff at \\
 & \hspace{2.8em} maximum spectral displacement\index{response spectral displacement}; \\
 & \hspace{0.5em} 4 $\Rightarrow$ use PGA from attenuation model to scale \\
 & \hspace{2.8em} HAZUS $S_a$. \\
\typepar{R}{thr}{sh}  &  Threshold distance, beyond which motion is assigned to zero.       \\
\typepar{per}{io}{ds} &  Periods for $S_a$. Values must ascend.    \\
\typepar{pga}{cut}{off}   & PGA cutoff for re-scaling $S_a$ with PGA greater than \typepar{pga}{cut}{off}.      \\
\typepar{smoothed}{\_response}{\_flag}  & Controls smoothing of $S_a$: \\
 & \hspace{0.5em} 0 $\Rightarrow$ No smoothing; \\
 & \hspace{0.5em} 1 $\Rightarrow$ Smoothing.  \\
 \typepar{log}{\_sigma}{\_eq}{\_weight} &  The fraction of
 $\sigma$ assocaited with the earthquake. (1 -
 \typepar{log}{\_sigma}{\_eq}{\_weight}) is the fraction of $\sigma$
 assocaited with the earthquake/site pair.  This variable defaults to
 0 if not set.  Use a value between 0 and 1.\\
 \end{supertabular}


\vspace{2em} \noindent \textbf{Amplification:}

\begin{supertabular}{lp{0.6\textwidth}}
\typepar{amp}{\_swit}{ch}  &  Amplification associated with local regolith: \\
 & \hspace{0.5em} 0 $\Rightarrow$ NOT included; \\
 & \hspace{0.5em}  1 $\Rightarrow$ included. \\
\typepar{var}{\_amp}{\_flag}    &   Variability for amplification: \\
 & \hspace{0.5em} 0 $\Rightarrow$ NOT included; \\
 & \hspace{0.5em} 1 $\Rightarrow$ included.  \\
\typepar{var}{\_amp}{\_method} & Technique used to incorporate amplification aleatory uncertainty: \\
 & \hspace{0.5em} 1 $\Rightarrow$ Sample PDF - spawning; \\
 & \hspace{0.5em} 2 $\Rightarrow$ random sampling; \\
 & \hspace{0.5em} 3 $\Rightarrow$ $+2\sigma$; \\
 & \hspace{0.5em} 4 $\Rightarrow$ $+\sigma$; \\
 & \hspace{0.5em} 5 $\Rightarrow$ $-\sigma$; \\
 & \hspace{0.5em} 6 $\Rightarrow$ $-2\sigma$.\\
 & \hspace{0.5em} 7 $\Rightarrow$ $-2\sigma$.\\
\typepar{Max}{Amp}{Factor}    &   Maximum accepted value for amplification factor.   \\
\typepar{Min}{Amp}{Factor}    &   Minimum accepted value for amplification factor.    \\
 \end{supertabular}


\vspace{2em} \noindent \textbf{Bclasses:}

\begin{supertabular}{lp{0.6\textwidth}}
\typepar{b\_usage}{\_type}{\_flag} & Building usage classification system: \\
 & \hspace{0.5em} 1 $\Rightarrow$ HAZUS usage\index{building usgage!HAZUS} classification; \\
 & \hspace{0.5em} 2 $\Rightarrow$ FCB\index{building usgage!FCB} usage classification.\\
\typepar{hazus}{\_btypes}{\_flag}   & Building construction type classification system: \\
 & \hspace{0.5em} 0 $\Rightarrow$ refined HAZUS building classification\index{building type}; \\
 & \hspace{0.5em} 1 $\Rightarrow$ HAZUS building classification\index{building type}.     \\
\typepar{hazus}{\_damping}{is5\_flag}   &   Level of damping when \typepar{build}{pars}{\_flag} = 2: \\
 & \hspace{0.5em} 0 $\Rightarrow$ Use damping level corresponding to  \\
 & \hspace{2.8em} \typepar{build}{pars}{\_flag}=0; \\
 & \hspace{0.5em} 1 $\Rightarrow$ Use $5\%$.    \\
\typepar{build}{pars}{\_flag} & Engineering parameters to be used: \\
 & \hspace{0.5em} 0 $\Rightarrow$ Australian Engineers Workshop  \\
& \hspace{2.8em} Parameters (AEWP) with Edwards \\
& \hspace{2.8em}  modifications 1; \\
 & \hspace{0.5em} 1 $\Rightarrow$ AEWP; \\
 & \hspace{0.5em} 2 $\Rightarrow$ original HAZUS paramaters; \\
 & \hspace{0.5em} 3 $\Rightarrow$ AEWP with Edwards modifications 2; \\
 & \hspace{0.5em} 4 $\Rightarrow$ AEWP with Edwards modifications 3. \\
 \end{supertabular}


\vspace{2em} \noindent \textbf{Bclasses2:}

\begin{supertabular}{lp{0.6\textwidth}}
\typepar{force}{\_btype}{\_flag}    & Force all buildings to have the same usage and type classification: \\
 & \hspace{0.5em} 0 $\Rightarrow$ Use buildings as defined in database; \\
 & \hspace{0.5em} 1 $\Rightarrow$ Force modification. \small{\small{CURRENTLY NOT OPERATIONAL}}\\
\typepar{det}{erm}{\_btype}    &  Building type when \typepar{force}{\_btype}{\_flag}=1.    \\
\typepar{det}{erm}{\_buse} &   Building usage when \typepar{force}{\_btype}{\_flag}=1.     \\
 \end{supertabular}


\vspace{2em} \noindent \textbf{CSM:}

\begin{supertabular}{lp{0.6\textwidth}}
\typepar{var}{\_bcap}{\_flag}   &   Variability with building capacity curve\index{capacity curve}: \\
 & \hspace{0.5em} 0 $\Rightarrow$ NOT included; \\
 & \hspace{0.5em}  1 $\Rightarrow$ included.    \\
\typepar{bcap}{\_var}{\_method} & Method used to incorporate variability in capacity curve\index{capacity curve}: \\
 & \hspace{0.5em} 1 $\Rightarrow$ Random sampling used to adjust the \\
& \hspace{2.8em} vertical position of the yield and \\
& \hspace{2.8em} ultimate points; \\
 & \hspace{0.5em} 2 $\Rightarrow$ \small{BACK COMPARABILITY ONLY - } \\
& \hspace{2.8em} \small{CONTAINS ERROR}. \\
 & \hspace{0.5em} 3 $\Rightarrow$ \small{PREFERRED}: Random sampling \\
& \hspace{2.8em} applied to ultimate point only and yield \\
& \hspace{2.8em}  point re-calculated to satisfy capacity \\
& \hspace{2.8em}  curve\index{capacity curve} `shape' constraint. \\
\typepar{st}{d}{cap}  & Standard deviation for capacity curve\index{capacity curve} log--normal PDF.      \\
\typepar{damp}{\_fla}{gs}  & Vector to control nature of linear damping (see below for explanation of each element). \\
\typepar{damp}{\_fla}{gs}$(1)$ & Damping multiplicative formula to
be
used - \manual{Section 7.2.2}: \\
 & \hspace{0.5em} 0 $\Rightarrow$ \small{PREFERRED}: use $R_a$, $R_v$, and $R_d$; \\
 & \hspace{0.5em} 1 $\Rightarrow$ use $R_a$, $R_v$ and assign $R_d= R_v$; \\
 & \hspace{0.5em} 2 $\Rightarrow$ use $R_v$ only and assign $R_a=R_d=R_v$. \\
\typepar{damp}{\_fla}{gs}$(2)$ & Modify transition building period i.e. corner period $T_{av}$: \\
 & \hspace{0.5em} 0 $\Rightarrow$ \small{PREFERRED}: modify as in HAZUS; \\
 & \hspace{0.5em} 1 $\Rightarrow$ do NOT modify. \\
\typepar{damp}{\_fla}{gs}$(3)$ & Smoothing of damped curve: \\
 & \hspace{0.5em} 0 $\Rightarrow$ \small{PREFERRED}: apply smoothing; \\
 & \hspace{0.5em} 1 $\Rightarrow$ NO smoothing.\\
\typepar{Har}{ea}{\_flag} & Technique for Hysteretic  damping: \\
 & \hspace{0.5em} 0 $\Rightarrow$ NO Hysteretic  damping; \\
 & \hspace{0.5em} 1 $\Rightarrow$ \small{BACK COMPARABILITY ONLY - }  \\
& \hspace{2.8em} \small{CONTAINS ERROR - AS SHOWN IN} \\
& \hspace{2.8em} \small{MANUAL};\\
 & \hspace{0.5em} 2 $\Rightarrow$ Hysteretic  damping via trapezoidal \\
 & \hspace{2.8em} approximation; \\
 & \hspace{0.5em} 3 $\Rightarrow$ \small{PREFERRED}: Hysteretic  damping via \\
 & \hspace{2.8em} curve fitting. \\
\typepar{SD}{Rel}{Tol}    &    Tolerance (as a percentage) for SDcr (in nonlinear damping calculations).   \\
\typepar{max}{\_iter}{ations}  & Maximum iterations for nonlinear damping calculations.\\
 \end{supertabular}

\vspace{2em} \noindent \textbf{Diagnostics:}

\begin{supertabular}{lp{0.6\textwidth}}
\typepar{qa}{\_switch}{\_ampfactors} & \small{CURRENTLY NOT OPERATIONAL}   \\
\typepar{qa}{\_switch}{\_attn}  &   \small{CURRENTLY NOT OPERATIONAL} \\
\typepar{qa}{\_switch}{\_fuse}  &   \small{CURRENTLY NOT OPERATIONAL} \\
\typepar{qa}{\_switch}{\_map}   &  \small{CURRENTLY NOT OPERATIONAL} \\
\typepar{qa\_switch}{\_mke}{\_evnts} &  \small{CURRENTLY NOT OPERATIONAL} \\
\typepar{qa\_switch}{\_water}{check} &   \small{CURRENTLY NOT OPERATIONAL}  \\
\typepar{qa}{\_switch}{\_soc}   &  \small{CURRENTLY NOT OPERATIONAL} \\
\typepar{qa}{\_switch}{\_vun}   &  \small{CURRENTLY NOT OPERATIONAL} \\
 \end{supertabular}

\vspace{2em} \noindent \textbf{Loss:}

\begin{supertabular}{lp{0.6\textwidth}}
 \typepar{pga}{\_min}{damage}   &  minimum PGA(g) below which financial loss is assigned to zero. \\
 \typepar{c}{}{i} & Regional cost index multiplier to convert dollar values in building database\index{building database} to desired regional and temporal (i.e. inflation) values.\\
\typepar{aus}{\_contents}{\_flag}   &  Contents value for residential buildings and salvageability after complete building damage:   \\
& \hspace{0.5em} 0 $\Rightarrow$ contents value as defined in building \\
& \hspace{2.8em} database\index{building database} and salvageability of $50\%$;\\
& \hspace{0.5em} 1 $\Rightarrow$ $60\%$ contents value as defined in \\
& \hspace{2.8em} building database\index{building database} and salvageability of zero.\\
 \end{supertabular}

\vspace{2em} \noindent \textbf{Save:}

\begin{supertabular}{lp{0.6\textwidth}}
\typepar{hazard}{\_map}{\_flag} & Save data for hazard maps: \\
& \hspace{0.5em} 0 $\Rightarrow$ do NOT save; \\
& \hspace{0.5em} 1 $\Rightarrow$ save.      \\
\typepar{save}{\_ecloss}{\_flag}    & Save direct financial loss data: \\
& \hspace{0.5em} 0 $\Rightarrow$ do NOT save; \\
& \hspace{0.5em} 1 $\Rightarrow$ save total financial loss; \\
& \hspace{0.5em} 2 $\Rightarrow$ save contents and building loss \\
& \hspace{3em} separately.        \\
\typepar{save}{\_socloss}{\_flag} & \small{CURRENTLY NOT OPERATIONAL} \\
\typepar{save}{\_motion}{\_flag}   &  Save RSA motion: \\
& \hspace{0.5em} 0 $\Rightarrow$ do NOT save; \\
& \hspace{0.5em} 1 $\Rightarrow$ save.     \\
\typepar{save}{\_probdam}{\_flag} &  \small{CURRENTLY NOT OPERATIONAL}  \\
\typepar{save}{\_deagecloss}{\_flag}   & Save structural
non-cumulative probability of being in each
damage state.  Note this is only allowed when there is one event. \\
& \hspace{0.5em} 0 $\Rightarrow$ do NOT save; \\
& \hspace{0.5em} 1 $\Rightarrow$ save.      \\
 \end{supertabular}


\end{document}

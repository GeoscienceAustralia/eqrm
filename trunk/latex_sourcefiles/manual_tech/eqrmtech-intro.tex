\chapter{Introduction}
\label{ch:intro}

\section{Overview}

The EQRM application is a computer model for estimating earthquake
hazard and earthquake risk. Modelling earthquake hazard involves
assessing the probability that certain levels of ground motion
will be exceeded. Modelling of earthquake risk involves estimating
the probability of a building portfolio experiencing a range of
earthquake induced losses. For any number of synthetic
earthquakes, the EQRM application can be used to estimate:
\begin{enumerate}
\item the ground motion and its likelihood of occurrence
(earthquake hazard), \item the direct financial loss and its
likelihood of occurrence (earthquake risk), and \item less
reliably the number of fatalities and their
likelihood of occurrence (earthquake risk).
\end{enumerate}

The EQRM application is Geoscience Australia's centerpiece for
modelling earthquake hazard and risk. Its use formed the basis for
Geoscience Australia's recent reports on Earthquake risk in the
Newcastle and Lake Macquarie \citep*{dr_Dhu02a} and Perth
\citep*{dr_Sinadinovski05a} regions.

EQRM is a product of Geoscience Australia, an Australian Government
Agency and is open-source.  It can be downloaded from
\texttt{https://sourceforge.net/projects/eqrm/}.

The process for computing earthquake hazard can be described by
the following steps:
\begin{enumerate}
\item the generation of a catalogue of synthetic earthquakes (or
events) (see \cref{ch:source}), 
\item the propagation or
attenuation (see \cref{ch:atten}) of the `seismic wave' from each
of the events in 1 to locations of interest (see \cref{ch:application}),
\item accounting for the interactions between the propagating
`seismic wave' and the local geology or regolith (see
\cref{ch:reg}), and \item accounting for the probability of each
event and the estimation of hazard (see \cref{ch:risk}).
\end{enumerate}

The process for computing earthquake risk shares the same first
three steps as the earthquake hazard. The fourth step and onwards
can be described as follows:
\begin{enumerate}
\item[4.] estimating the probability that the portfolio buildings
(see \cref{ch:application}) will experience different levels of damage
(see \cref{ch:damage}), \item[5.] the computation of direct
financial loss as a result of the probabilities computed in 4 (see
\cref{ch:losses}), and \item[6.] assembling the results to compute
the risk (see \cref{ch:risk}).
\end{enumerate}

\section{Using this manual}

This manual describes the EQRM application; it has been designed
to serve the following three purposes:
\begin{enumerate}
\item describe the theory and methodology behind the EQRM
application; \item explain how to use the EQRM application to
model earthquake hazard and risk; and \item provide enough
information to assist those who may wish to access individual
modules and modify them.
\end{enumerate}

A number of features have been included to assist readers of the
manual. These features include:
\begin{itemize}
\item \textbf{Text highlighting -} Important parameters, file
names and code are identified by a \texttt{typeset font}. \item
\textbf{Index -} An index has been introduced to allow speedy of
navigation of the manual.

\end{itemize}


\section{About this manual}

\textit{\cref{ch:application}: The EQRM application} introduces
the EQRM software. The chapter describes the directory structure
and introduces important input parameter files. Setup and
operational processes are discussed and the conversion of the EQRM
to a stand-alone executable described. Readers who do not wish to
familiarise themselves with details of the EQRM application may
wish to skip this chapter.

\textit{\cref{ch:source}: Earthquake source generation} discusses
the creation of an earthquake catalogue. At the core of any hazard
or risk assessment is the simulation of synthetic earthquakes and
the creation of a catalogue of synthetic events. The chapter
describes how the simulation of earthquakes can focus around a
specific earthquake of interest (a scenario based simulation) or
how it can model the effect of `all foreseeable' events (a
probabilistic simulation).

\textit{\cref{ch:atten}: Ground-Motion Prediction Equations} discusses
the propagation (or attenuation) of the motion from each of the
synthetic earthquakes in the event catalogue to locations of interest.
Different measures of distance between an event and a site of interest
are introduced. The chapter also describes the attenuation models that
can be used with the EQRM application.

\textit{\cref{ch:reg}: Regolith amplification} discusses how local
geology, or regolith, can be incorporated in an earthquake hazard
or risk assessment. The chapter illustrates how the presence of
regolith can lead to amplification (or de-amplification) of both
ground and building motion, and consequently result in higher (or
lower) levels of hazard and risk.

\textit{\cref{ch:damage}: Building damage} describes how the EQRM
application estimates the probability that each building will
experience different levels of damage. The chapter gives a brief
discussion of the theory behind making such estimates as well as
outlining how to extrapolate the estimates to an entire portfolio
of buildings in a region of interest.

\textit{\cref{ch:losses}: Losses} illustrates how to model the
direct financial loss as a result of the damage estimates
described in \textit{\cref{ch:damage}}. The chapter also provides
a brief description of how the EQRM application can estimate
fatalities.

\textit{\cref{ch:risk}: Hazard and Risk Results}
describes how the results from the EQRM application can be
summarised and displayed. There are many ways to display estimates
of earthquake hazard and risk. Earthquake hazard is commonly
modelled in terms of a probability (often 10\%) of a particular
ground motion (usually acceleration) being exceeded in some time
frame (often 50 years). In the case of a scenario based
simulation, earthquake risk can be modelled in terms of dollar
estimates of building loss. In the case of a probabilistic simulation,
earthquake risk is often modelled in terms of one or both of the
following:
\begin{enumerate}
\item the loss per year averaged over a long period of time,
typically 10000 years. This is known as the annualised loss. \item
the probability that different levels of loss will be exceeded.
The plot of loss against such exceedance probabilities is referred
to as a probable maximum loss curve.
\end{enumerate}

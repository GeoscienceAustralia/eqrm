\chapter{The EQRM application}
\label{ch:application}

The Earthquake Risk Model (EQRM) is capable of earthquake scenario
ground motion and scenario loss modeling as well as probabilistic
seismic hazard (PSHA) and risk (PSRA) modeling. It is a product of
Geoscience Australia, an Australian Government Agency.

This chapter describes the EQRM application. Input files and
parameters are discussed and directions on how to run the EQRM
provided. Readers who are interested in only the EQRM methodology
and not the EQRM software package may wish to skip this chapter.

\section{EQRM inputs}

The input files required by the EQRM depend on the nature of the
simulation conducted. For example, the inputs for a scenario loss
simulation are different to those required for a probabilistic
seismic hazard analysis. \tref{tab:input-overview} provides a
summary of the inputs required by the EQRM. The EQRM Demos (see
\sref{sec:application-demo}) provide examples of each input file and
demonstrate how to run the EQRM for each of the four main simulation
types. srefmulti provides an overview of each of the main input
files. Further details are provide in the following chapters.

\begin{table}
\caption{Input files required for different types of simulation with
the EQRM. The astericks indicates optional input files, the
requirement for which depends on settings in the EQRM control file
(see \cref{ch:reg} for more information).}
\label{tab:input-overview} \centering
\begin{tabular}{|l|l|l|}
\hline
 & \textbf{hazard} & \textbf{risk} \\
\hline
\textbf{scenario} & EQRM controlfile & EQRM controlfile \\
  & amplification factors* & amplification factors* \\
  & hazard grid & building database \\
\hline
\textbf{probabilistic} & EQRM controlfile & EQRM controlfile\\
  & source file(s) & source file(s) \\
  & event type controlfile & event type controlfile \\
  & amplification factors* & amplification factors* \\
  & hazard grid & building database \\
\hline
\end{tabular}
\end{table}

It contains a series of input parameters and flags that must be set
by the user. \sref{sec:application-parfiles} describes other files
that are required by the EQRM modules.

\subsection{The EQRM control file}
\label{sec:application-EQRMcf}

% First lets include the descriptom of the EQRM control file and its parameters
\input param_list_text

% Now lets include an example EQRM controlfile

\clearpage For example, the following grey shaded box provides an
example of an EQRM controlfile to undertake a PSHA. It is actually
one of the demos described in more detail in
\sref{sec:application-demos}.

\lstinputlisting{../../demo/setdata_ProbHaz.py}


\clearpage
\subsection{The source files}

\lstinputlisting{xml_examples/source_model_zone_example.xml}



\subsection{Parameter files}
\label{sec:application-parfiles}

This section identifies the important input files that are used by
different components of the EQRM. Input files can be dependent or
independent of region. For example the file
 \typeparfile{newc}{\_par}{\_ampfactors}{.mat} contains the amplification factors
for the Newcastle region (see \sref{ch:reg}), whereas the file
 \typeparfile{attn\_toro}{\_midcontinent}{\_momag}{.txt}
contains attenuation coefficients for the \citet*{dr_Toro97a}
attenuation model (see \sref{ch:atten}) which are not dependent on
region. The EQRM identifies the desired region dependent files
using the user defined parameter \typepar{site}{\_loc}{} (see
\sref{sec:application-setdata}).

The EQRM makes use of default data to reduce the need for a large
number of defined input files. Default data files are stored in
the directory \typedir{*/eqrm/eqrm\_app}{/resources}{/data} (see
\sref{sec:app-dirstruct}). The EQRM facilitates the use of
non-default data by first looking in the user defined
 \typepar{inp}{ut}{dir} (see \sref{sec:application-setdata}) for
required files. If the required files are found in
 \typepar{inp}{ut}{dir} they will be used by the EQRM, if the files
are not found in  \typepar{inp}{ut}{dir} then the default files
are used.

Note that default files are available for pre-defined regions
only. For example; Newcastle and Perth region default data can be
accessed by defining \typepar{site}{\_}{loc} of \typenewc and
\typeperth respectively.

\begin{table}
\caption{Default parameter files in
\typedircaption{*/eqrm/eqrm\_app}{/resources}{/data}.}
\label{tab:app-inputfiles} \vspace{0.8em}
\begin{tabular}{|l|p{0.7\textwidth}|}
\hline
Chapter & Files \\
\hline Earthquake & $\typeparfile{<site\_loc>}{\_par}{\_sourcepolys}{.txt}$ \\
Source generation &  $\typeparfile{<site\_loc>}{\_par}{\_sourcezones}{.txt}$ \\
\hline
Grids and & $\typeparfile{au}{s}{\_mat}{.txt}$\\
building & $\typeparfile{aust}{ralia}{\_bndy}{.txt}$\\
databases & $\typeparfile{<site\_loc>}{\_par}{\_site}{.mat}$\\
 &  $\typeparfile{<site\_loc>}{\_par\_site}{\_uniform}{.mat}$ \\
 &  $\typeparfile{<site\_loc>}{\_par\_study}{\_region\_box}{.txt}$ \\
  & $\typeparfile{sitedb}{\_<site}{\_loc>}{.mat}$ \\
 & $\typeparfile{suburb}{\_post}{code}{.csv}$\\
  & $\typeparfile{text}{b}{types}{.txt}$ \\
 & $\typeparfile{text}{busage}{FCB}{.txt}$ \\
 &  $\typeparfile{text}{b}{uses}{.txt}$ \\
\hline
Attenuation & $\typeparfile{attn}{\_atkboore}{\_momag}{.txt}$\\
 &  $\typeparfile{attn\_sadigh}{\_coeff\_momag}{\_great65}{.txt}$ \\
 &  $\typeparfile{attn\_sadigh}{\_coeff\_momag}{\_less65}{.txt}$ \\
 &  $\typeparfile{attn\_somer}{\_nonrift}{\_hori}{.txt}$ \\
 &  $\typeparfile{attn\_somer}{\_nonrift}{\_vert}{.txt}$ \\
 &  $\typeparfile{attn\_somer}{\_rift}{\_hori}{.txt}$ \\
 &  $\typeparfile{attn\_somer}{\_rift}{\_vert}{.txt}$ \\
 & $\typeparfile{attn\_toro}{\_midcontinent}{\_momag}{.txt}$,              \\
\hline
Regolith &  $\typeparfile{<site\_loc>}{\_par}{\_ampfactors}{.mat}$\\
amplification &  $\typeparfile{<site\_loc>\_par}{\_site\_class}{\_mask}{.mat}$ \\
 &  $\typeparfile{<site\_loc>\_par}{\_site\_class}{\_polys}{.mat}$ \\
\hline
Building & $\typeparfile{b}{factors}{db}{.mat}$\\
damage & $\typeparfile{bfactors}{db}{\_hazus}{.mat}$ \\
 & $\typeparfile{bfactors}{db}{\_wshop}{.mat}$ \\
 & $\typeparfile{bfactorsdb}{\_wshop}{\_update2}{.mat}$ \\
 & $\typeparfile{bfactorsdb}{\_wshop}{\_update3}{.mat}$ \\
 & $\typeparfile{factors}{-nonstruct}{dam}{.csv}$ \\
 &  $\typeparfile{factors}{-struct}{dam}{.csv}$ \\
\hline
Losses & $\typeparfilelong{rc\_perRepl}{CostwrtBuildC}{EdwardsFCBusage}{.mat}{rc\_perRepl...EdwardsFCBusage}$ \\
& $\typeparfilelong{rc\_perReplCost}{wrtBuildCEdwards}{Hazususage}{.mat}{rc\_perRepl...EdwardsHazususage}$\\
% & $\typeselfnoindex{rc\_perReplCost}{wrtBuildCEdwards}{Hazususage.mat}\index{\scriptsize\texttt{rc\_perReplCostwrtBuildCEdwardsHazususage.mat}\normalsize}$\\
& $\typeparfile{rc\_perReplCost}{wrtBuildC}{Hazusage}{.mat}$\\
& $\typeparfile{econ}{\_p}{ar}{.mat}$\\
\hline
\end{tabular}
\end{table}

\tref{tab:app-inputfiles} summarises the important input files
that can be found in the default data area. The files are listed
against the chapter where more information can be found.


\subsection{Running the EQRM in MATLAB}
\label{app:runEQRM}


The EQRM has a special startup function
(\typefunc{eqrm}{\_start}{up}) that adds all the required
directories to the MATLAB path. The following instructions
describe how to use \typefunc{eqrm}{\_start}{up} to set the MATLAB
path:
\begin{itemize}
\item move to the directory
\typedir{*/eqrm/eqrm\_app}{/m\_code}{/startup} \item at the
command line type \newline \texttt{>>
eqrm\_startup\indexfunc{eqrm\_startup}(`*$\backslash$eqrm$\backslash$eqrm\_app')}
\end{itemize}

The EQRM has a special GUI known as the
\typegui{eqrm}{\_param}{\_gui} that can be used to create a
\typeself{set}{da}{ta} file, run the EQRM application or analyse
the EQRM outputs. To launch the GUI simply type the following at
the command line:
\begin{itemize} \item[\texttt{>>}] \texttt{eqrm\_param\_gui}\indexgui{eqrm\_param\_gui} \end{itemize}
It is beyond the scope of this manual to discuss the
\typeself{eqrm}{\_param}{\_gui} further. However it is user
friendly and reasonably self explanatory. Alternatively the EQRM
can be run from the command line using the function
\typefunc{eqrm}{\_analy}{sis}. To find out how to use
\typefunc{eqrm}{\_analy}{sis} type the following at the MATLAB
command line: \begin{itemize} \item[\texttt{>>}] \texttt{help
eqrm\_analysis}
\end{itemize}

The EQRM has a special Output Manager GUI that can be used to
view, analyse and plot EQRM outputs. The Output Manager is also
known as the Data Interrogation Tool
(DIT)\index{DIT}\index{GUI!DIT}\index{Data Interrogation
Tool}\index{GUI!Data Interrogation Tool}. A report generation
facility has been included with the Output Manager and can be used
to create a \LaTeX \texttt{*.tex} file with EQRM output figures.
The \LaTeX file can be compiled into a \texttt{.ps} and/or
\texttt{.pdf} file if the appropriate MikTeX software is
installed.


\subsection{Compiling and using a stand--alone version of the EQRM in MATLAB R14SP1}

Using the MATLAB compiler it is possible to create a stand--alone
version of the EQRM that can be used without requiring MATLAB. The
instructions for MATLAB R13 are given in \aref{app:MATLABR13}.

\section{Creating a stand--alone executable of the EQRM GUI in MATLAB R14SP1:}
\label{app:MATLABR13}

Type the following at the MATLAB command line for the source
machine \begin{itemize} \item[\texttt{>>}]
\texttt{cnt\_do\_mcc\_build\_to\_system(`eqrm\_param\_gui',<*/path>)}
\end{itemize}

This process will create the following files in \texttt{<*/path>}:
\begin{enumerate}
\item \typeselfnoindexnoline{eqrm}{\_param}{\_gui.exe};  \item a
directory named \typeselfnoindexnoline{eqrm}{\_param}{\_gui\_mcr};
and \item \typeselfnoindexnoline{eqrm}{\_param}{\_gui.ctf}.
\end{enumerate}

\vspace{2em}

\textbf{Prepare the files for transportation as follows:}
\begin{enumerate} \item Create a folder entitled
\typeselfnoindexnoline{*/te}{st}{exe}. \item In
\typeselfnoindexnoline{*/te}{st}{exe} create two folders
 \typeselfnoindexnoline{*/}{testexe}{/system} and
 \typeselfnoindexnoline{*/}{testexe}{/mcr}.
\item Copy  \typeselfnoindexnoline{eqrm}{\_param}{\_gui.exe},
\typeselfnoindexnoline{eqrm}{\_param}{\_gui\_mcr} and
\typeselfnoindexnoline{eqrm}{\_param}{\_gui.ctf} (created above)
into \typeselfnoindexnoline{*/}{testexe}{/system}. \item Copy the
following files from \typeselfnoindexnoline{*/EQRM}{ROOT}{/system}
into \typeselfnoindex{*/}{testexe}{/system}:
\begin{enumerate}
\item \typeselfnoindexnoline{root\_mat}{\_creator}{\_cnt.exe};
\item the directory
\typeselfnoindexnoline{root\_mat}{\_creator}{\_cnt\_mcr}; and
\item \typeselfnoindexnoline{root\_mat}{\_creator}{\_cnt.ctf}.
\end{enumerate}
\item Copy \typeself{\_MCR}{Instal}{ler.exe} from
\typeselfnoindexnoline{<MATLABROOT>/toolbox}{/compiler/deploy}{/win32/MCRInstaller.exe}
into \typeselfnoindexnoline{*/}{testexe}{/mcr}. \item Copy the
directory \typeselfnoindexnoline{*/EQRM}{ROOT}{/resources} to
\typeselfnoindexnoline{*/}{testexe}{/resources}.
\end{enumerate}

\vspace{2em} \textbf{Installation onto the target machine is
achieved as follows:}
\begin{enumerate}
\item Move the contents of \typeselfnoindex{*/te}{st}{exe} onto
the target machine. \item Run \typeself{\_MCR}{Instal}{ler.exe} on
\typeselfnoindex{*/}{testexe}{/mcr} by double clicking on it.
\item Set environment path on the target machine
\begin{enumerate}
\item open the control panel; \item open system; \item select the
advanced tab; \item select environment variables; \item create new
user variable entitled \typeselfnoindex{p}{a}{th}; \item add the
following path names (separated by semi colons) into the value
field for \texttt{path};
\begin{itemize} \item
\typeselfnoindexnoline{*/}{testexe}{/system}; and \item
\typeselfnoindexnoline{*/testexe}{/mcr/v71}{runtime/win32};
\end{itemize}
\item Log off and log back on.
\end{enumerate}
\item Set the software path as follows
\begin{enumerate}
\item Open a DOS shell; \item Type
\typeselfnoindex{root}{\_mat\_creator}{\_cnt.exe}; \item On the
`EQRM root file creation' GUI select the EQRM root directory (i.e.
\typeselfnoindex{*/te}{st}{exe}).
\end{enumerate}
\item Open the EQRM GUI by typing
\typeselfnoindex{eqrm}{\_param}{\_gui.exe} at the DOS command
prompt.
\end{enumerate}

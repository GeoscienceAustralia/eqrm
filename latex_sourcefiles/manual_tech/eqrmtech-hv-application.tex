\chapter{The EQRM application}
\label{ch:application}

The Earthquake Risk Model (EQRM) is capable of earthquake scenario
ground motion and scenario loss modeling as well as probabilistic
seismic hazard (PSHA) and risk (PSRA) modeling. It is a product of
Geoscience Australia, an Australian Government Agency.

This chapter describes the EQRM application. Input files and
parameters are discussed and directions on how to run the EQRM
provided. Readers who are interested in only the EQRM methodology
and not the EQRM software package may wish to skip this chapter.

\section{EQRM inputs}

The input files required by the EQRM depend on the nature of the
simulation conducted. For example, the inputs for a scenario loss
simulation are different to those required for a probabilistic
seismic hazard analysis. \tref{tab:input-overview} provides a
summary of the inputs required by the EQRM. The EQRM Demos (see
\sref{sec:application-demo}) provide examples of each input file and
demonstrate how to run the EQRM for each of the four main simulation
types. srefmulti provides an overview of each of the main input
files. Further details are provide in the following chapters.

\begin{table}
\caption{Input files required for different types of simulation with
the EQRM. The astericks indicates optional input files, the
requirement for which depends on settings in the EQRM control file
(see \cref{ch:reg} for more information).}
\label{tab:input-overview} \centering
\begin{tabular}{|l|l|l|}
\hline
 & \textbf{hazard} & \textbf{risk} \\
\hline
\textbf{scenario} & EQRM controlfile & EQRM controlfile \\
  & amplification factors* & amplification factors* \\
  & hazard grid & building database \\
\hline
\textbf{probabilistic} & EQRM controlfile & EQRM controlfile\\
  & source file(s) & source file(s) \\
  & event type controlfile & event type controlfile \\
  & amplification factors* & amplification factors* \\
  & hazard grid & building database \\
\hline
\end{tabular}
\end{table}

It contains a series of input parameters and flags that must be set
by the user. \sref{sec:application-parfiles} describes other files
that are required by the EQRM modules.

\subsection{The EQRM control file}
\label{sec:application-EQRMcf}

% First lets include the descriptom of the EQRM control file and its parameters
\input param_list_text

% Now lets include an example EQRM controlfile

\clearpage For example, the following grey shaded box provides an
example of an EQRM controlfile to undertake a PSHA. It is actually
one of the demos described in more detail in
\sref{sec:application-demos}.

\lstinputlisting{../../demo/setdata_ProbHaz.py}


\clearpage
\subsection{The source files}

The EQRM source files for probabilistic modelling (PSHA and PSRA)
come in two forms. These are:
\begin{itemize}
\item source zones, and
\item faults.
\end{itemize}
The EQRM can be run with either of these inputs seprately or both of
them together


\subsubsection{Source Zones}
\label{sec:source-zone-file}

The source zone file is used to describe one or more areal source
zones. Earthquakes are assumed to be equally likely to occur
anywhere within a source zone. The magnitude recurence relationship
for each source zone is defined by a bounded Gutenberg-Richter
relationship. The following grey shaded box provides an example of a
source zone file. A description of each of the parameters follows.
\lstinputlisting{xml_examples/source_model_zone_example.xml}

\textbf{General inputs} (\texttt{source\_model\_zone})
\begin{itemize}
\item \texttt{magnitude\_type}: earthquake magnitude used derive the
recurence parameters. NOTE - the EQRM only supports moment magnitude
\texttt{Mw}.
\end{itemize}

\textbf{General zone inputs} (\texttt{zone})
\begin{itemize}
\item \texttt{area}: area of the source zone in km$^2$ \\
\item \texttt{name}: name for the source zone \\
\item \texttt{event\_type}: pointer to the a collection of inputs described in the
\texttt{event\_type\_controlfile}.
\end{itemize}

\textbf{Geometry inputs} (\texttt{geometry})
\begin{itemize}
\item \texttt{azimuth}: center azimuth for randomly generated synthetic ruptures \\
\item \texttt{delta\_azimuth}: range over which randomly generated azimuths will
be sampled. That is, the azimuth of all synthetic earthquake will be randomly drawn
from a uniform distribution between \texttt{azimuth}$\pm$\texttt{delta\_azimuth}. \\
\item \texttt{dip}: center dip for randomly generated synthetic ruptures \\
\item \texttt{delta\_dip}: range over which randomly generated dips will
be sampled. That is, the dip of all synthetic earthquake will be
randomly drawn from a uniform distribution between \texttt{dip}$\pm$\texttt{delta\_dip}.\\
\item \texttt{depth\_top\_seismogenic}: Depth to the top of the seimogenic zone in km.
No component of a synthetic rupture will be located above this value. \\
\item \texttt{depth\_bottom\_seismogenic}: Depth to the bottom of the seimogenic zone in km.
No component of a synthetic rupture will be located below this value. \\
\item \texttt{boundary}: Boundary of the areal source zone as
defined on the surface of the Earth in longitude (column 1) and
latitude (column 2). \\
\item \texttt{excludes}: Boundary of any regions in the source in
which events are not required. Boundary defined on the surface of
the Earth in longitude (column 1) and latitude (column 2).
\end{itemize}

\textbf{Recurrence inputs} (\texttt{recurrence\_model})
\begin{itemize}
\item \texttt{distribution}: Distribution used to define the magnitude recurence relations. Note that
the EQRM current only supports a Bounded Gutenberg-Richter recurence
relationship for source zones (i.e.
\texttt{bounded\_gutenberg\_richter}) \\
\item \texttt{recurrence\_min\_mag}: Minimum magnitude used to
define the recurence relationship \\
\item \texttt{recurrence\_max\_mag}: Maximum magnitude used to
define the recurence relationship. Typically this is the magnitude
of the largest earthquake expected in the zone. \\
\item \texttt{A\_min}: Expected number of earthquake with magnitude \texttt{recurrence\_min\_mag}
or higher in the source zone per year.
\item \texttt{b}: Gutenberg-Richter b value forounded Gutenberg-Richter recurence
relationship \\
\item \texttt{generation\_min\_mag}: Minimum magnitude for generation of synthetic
earthquake generation. The EQRM will only generate synthetic
earthquakes with magnitudes equal to or greater than
\texttt{generation\_min\_mag}. \\
\item \texttt{number\_of\_mag\_sample\_bins}: Number of magnitude
bins used to discretise the recurence relationship in the range
magnitude range \texttt{generation\_min\_mag} to
\texttt{recurrence\_max\_mag} \\
\item \texttt{number\_of\_events}: Number of syntehtic ruptures to
be generated in the source zone.
\end{itemize}

\subsubsection{Faults}

The source faults file is used to describe one or more faults
(including crsutal faults and subduction interfaces) and/or one or
more dipping slabs for intraslab earthquakes. Earthquakes are
assumed to be equally likely to occur anywhere within the fault (or
slab). The magnitude recurence for faults can be defined by a
bounded Gutenberg-Richter relationship or a combination of bounded
Gutenberg-Richter and Characteric. The magnitude recurence for the
intraslab earthquakes must be defined by a bounded Gutenberg-Richter
relationship. The following is an example of a source fault file
with the following source types:
\begin{enumerate}
\item crustal fault with recurnece defined by a bounded Gutenberg-Richter
relationship (\texttt{fault 1}), \\
\item crustal fault with recurnece defined by a combined bounded Gutenberg-Richter
(for small earthquakes) and a characteristic recurence for larger
earthquakes (\texttt{fault 2}), \\
\item a 3D dipping volume to represent intraslab earthquakes in the
subducting slab (\texttt{intraslab 1}).
\end{enumerate}
The following grey shaded box provides an example of a source zone
file. Many of the parameters are identical to those described in
\sref{sec:source-zone-file} and are not described seprarately here.
A description of the new parameters is provided below.
\lstinputlisting{xml_examples/source_model_fault_example.xml}

\textbf{Parameters unique to the source fault file}
\begin{itemize}
\item \texttt{dip}: Dip of fault, defined as angle in degrees from horizontal \\
\item \texttt{out\_of\_dip\_theta}: Out of plane dip, used for intraslab events. Angle between fault plane
and out of dip rupture plane \\
\item \texttt{delta\_theta}: ange of dips for intraslab events. This will be $\pm$ \texttt{delta\_theta}
taken from the angle \texttt{dip + out\_of\_dip\_theta} \\
\item \texttt{slab\_width}: Width of slab (km) when using a fault source to represent intraslab
in the subducting slab \\
\item \texttt{trace}: Surface trace of the fault along the surface
of the Earth. Note that it is the projection of fault along the
direction of dip. It is defined by the latitude (\texttt{lat}) and
longitude (\texttt{lon}) of the start and end of the trace. \\
\item \texttt{slip\_rate}: Slip rate of fault in mm per year.
\end{itemize}


\subsection{Event Type Control File}


\subsection{Site File}


\subsection{Amplification File}

\chapter{The EQRM application}
\label{ch:application}

This chapter describes the EQRM application. The directory
structure, input parameters and directions on how to run the EQRM
are all described. Converting the EQRM into a stand-alone
executable is also explained at the end of the chapter. Readers
who are interested in only the EQRM methodology and not the EQRM
software package may wish to skip this chapter.


\section{Directory structure}
\label{sec:app-dirstruct}

The code is currently organised in the following directory
structure:

\begin{supertabular}{lp{0.6\textwidth}}
 \typedir{*/eqrm}{/datacvt}{} & `off-line' routines for
preparing input data. \\
\typedir{*/eqrm}{/eqrm}{\_app} & software and default data files
comprising the EQRM application. \\
\typedir{*/eqrm}{/testing}{\_harness} & testing suite to
ensure new development meets pre-defined back-compatibility criteria. \\
\end{supertabular}

\vspace{2em}

The most significant sub-directories of
\typedir{*/eqrm}{/datacvt}{} are:

\begin{supertabular}{lp{0.7\textwidth}}
\typedir{*/buildingdb}{}{} & tools to assist the preparation of building databases\index{building database} (see \cref{ch:grids}).\\
\typedir{*/buildingpars}{}{} & tools to assist the preparation of engineering parameters (see \cref{ch:damage}).\\
\typedir{*/econsoc}{pars}{} & tools to assist the preparation of direct financial loss parameters  (see \cref{ch:losses}). \\
\typedir{*/site}{\_classes}{} & tools to prepare site class polygons (see \cref{ch:reg}). \\
\typedir{*/stats}{}{} & tools to extract statistics from building database\index{building database}s (see \cref{ch:grids}). Note that these tools were originally written to extract Newcastle information however they can easily be modified to work with other data sets. \\
\end{supertabular}

\vspace{2em} The most significant sub-directories of
\typedir{*/eqrm}{/eqrm}{\_app} are:

\begin{supertabular}{lp{0.7\textwidth}}
\typedir{*/m}{\_code}{} & EQRM source code, predominately MATLAB functions (see below for more details). \\
\typedir{*/resources}{}{} & data files to support the EQRM (see below for more details). \\
\typedir{*/system}{}{} &  tools and path information directing the EQRM to individual components (e.g. \typedir{*/m}{\_code}{} and \typedir{*/resources}{}{}). \\
\end{supertabular}

\vspace{2em}

The most significant sub-directories of
\typedir{*/eqrm}{/eqrm\_app}{/m\_code} are:

\begin{supertabular}{lp{0.7\textwidth}}
\typedir{*/general}{}{} & general functions (includes error manager). \\
\typedir{*/startup}{}{} & functions to assist with setting up the MATLAB Path (\sref{app:runEQRM}).\\
\typedir{*/hazard}{}{} & functions for source generation
(\cref{ch:source}), attenuation (\cref{ch:atten}),
amplification (\cref{ch:reg}) and hazard calculation (\cref{ch:risk}).\\
\typedir{*/c3}{\_gui}{} & functions to support the \typegui{eqrm}{\_param}{\_gui} (\sref{app:runEQRM}).\\
\typedir{*/mcc}{\_build}{\_apps} & functions to assist in building a stand--alone executable.\\
\typedir{*/utils}{}{} & general utility tools used by other functions.\\
\typedir{*/compare}{\_utils}{} & important functions for comparing MATLAB binary files (used by the testing suite).\\
\typedir{*/vuln}{}{} & functions for building damage
(\cref{ch:damage}), financial loss (\cref{ch:losses}) and
risk calculation (\cref{ch:risk}).\\
\typedir{*/processing}{}{} & post-processing functions for manipulating and plotting EQRM outputs (\cref{ch:risk}).\\
\typedir{*/dit}{}{} & functions to support the data interrogation tool GUI (output manager: \sref{app:runEQRM}).\\
\end{supertabular}


\vspace{2em} The most significant sub-directories of
\typedir{*/eqrm}{/eqrm\_app}{/resources} are:

\begin{supertabular}{lp{0.7\textwidth}}
\typedir{*/data}{}{} & parameter files and input data for the EQRM (e.g. attenuation model coefficients, engineering parameters etc.).\\
\typedir{*/c3}{\_gui}{} & figure files for the \typegui{eqrm}{\_param}{\_gui}.\\
\typedir{*/dit}{\_gui}{} & figure files for the DIT GUI (output manager).\\
\end{supertabular}

\vspace{2em} The most significant sub-directories of
\typedir{*/eqrm}{/testing}{\_harness} are:

\begin{supertabular}{lp{0.7\textwidth}}
\typedir{*/hazard}{}{} & input and output data for EQRM hazard back compatibility tests.\\
\typedir{*/vuln}{}{} & input and output data for EQRM risk back compatibility tests.\\
\end{supertabular}


\section{EQRM inputs}

For the purpose of this manual the EQRM inputs are discussed in
two separate sections. \sref{sec:application-setdata} describes
the contents of the \typeself{set}{da}{ta} file, which contains a
series of input parameters and flags that must be set by the user.
\sref{sec:application-parfiles} describes other files that are
required by the EQRM modules. In most cases these files can be
used directly from \typedir{*/eqrm/eqrm\_app}{/resources}{/data},
however the user has the ability to provide alternatives.

\subsection{The \texttt{setdata} file}
\label{sec:application-setdata}

The full path to the \typeself{set}{da}{ta} file must be passed to
the EQRM when the program is run. It contains a series of input
variables (or parameters) that define the manner in which the EQRM
is operated. For example; there is a flag to control whether the
EQRM models hazard or risk. Other flags can be used to define
which attenuation model is selected, whether site amplification is
considered or which set of engineering parameters is used. A
description of all of the variables in the \typeself{set}{da}{ta}
file is given below. The \typeself{set}{da}{ta} file contains a
single variable
\typeself{eqrm}{\_param}{\_T}\index{setdata!\texttt{eqrm\_param\_T}}.
For the purpose of this manual the fields of
\typeself{eqrm}{\_param}{\_T}\index{\texttt{setdata}!\texttt{eqrm\_param\_T}}
are segregated into 13 sub-groups as they appear on the EQRM GUI
(\sref{app:runEQRM}) however no such separation exists in the
\typeself{set}{da}{ta} file itself.

\textbf{Operation Mode:}

\begin{tabular}{lp{0.6\textwidth}}
\typepar{run}{\_type}{} & Defines the operation mode of the EQRM: \\
  & \hspace{0.5em} 1 $\Rightarrow$ hazard; \\
  & \hspace{0.5em} 2 $\Rightarrow$ risk; \\
  & \hspace{0.5em} 3 $\Rightarrow$ risk from hazard. \\
 \end{tabular}

\vspace{2em}
 \textbf{General:}

\begin{supertabular}{lp{0.6\textwidth}}
  \typepar{inp}{ut}{dir} &  Directory containing any local input files.\\
 \typepar{sav}{e}{dir}{} &   Directory for output files.    \\
 \typepar{site}{\_}{loc}    &   String used in input and output file names.    \\
 \typepar{i}{ea}{st}   &   Direction of increasing longitude: \\
  &  \hspace{0.5em} 1 $\Rightarrow$ increases toward East; \\
  &  \hspace{0.5em} -1 $\Rightarrow$ increases toward West. \\
\typepar{grid}{\_}{flag}   & Define grid type: \\
  & \hspace{0.5em} 1 $\Rightarrow$ Grid produced in GIS; \\
  & \hspace{0.5em} 2 $\Rightarrow$ Grid produced by \typeim. \\
\typepar{small}{\_site}{\_flag} &   Sampling of sites: \\
 & \hspace{0.5em} 0 $\Rightarrow$ use all sites; \\
 & \hspace{0.5em} 1 $\Rightarrow$  use sub-sample defined by \typepar{Si}{te}{Ind}.  \\
\typepar{Si}{te}{Ind} & Vector whose elements represent the site indices to be used in the sub-sample  \\
& \hspace{0.5em} (if $\typepar{small}{\_site}{\_flag}=1$). \\
\typepar{de}{str}{ing}    &   Optional string to be used in logfile. Currently only used with risk runs.    \\
\typepar{rtrn}{\_}{per}   &   Vector whose elements represent the return periods to be considered for hazard.    \\
 \end{supertabular}

\vspace{2em} \textbf{Source:}

\begin{supertabular}{lp{0.6\textwidth}}
\typepar{a}{z}{i} &     Predominant azimuth of events.  \\
\typepar{d}{\_a}{zi}   &   Azimuth range for events (i.e. \typepar{a}{z}{i} $\pm$ \typepar{d}{\_a}{zi}).   \\
\typepar{d}{i}{p} &    Dip of virtual fault\index{virtual fault}s.    \\
\typepar{ntrg}{vec}{tor}  &  Vector whose elements represent the desired number of simulated event\index{simulated event}s for each generation zone.   \\
\typepar{f}{ty}{pe}   &  Fault type for attenuation models.
\textit{Note that this parameter is currently not operational}.
 \\
 & \hspace{0.5em} 0 $\Rightarrow$ ss; \\
 & \hspace{0.5em} 0.5 $\Rightarrow$ ro; \\
 & \hspace{0.5em} 1 $\Rightarrow$ rv, nr or no; \\
 & \hspace{0.5em} 2 $\Rightarrow$ uk. \\
\typepar{w}{d}{th}    &  Width of virtual fault\index{virtual fault}.      \\
\typepar{min}{\_mag}{\_cutoff}  &  Minimum magnitude below which hazard is not considered.     \\
\typepar{n}{bin}{s}   &   Number of magnitude bins used to sample
event magnitudes.    \\
 \end{supertabular}


\vspace{2em} \textbf{Event Spawn:}

\begin{supertabular}{lp{0.6\textwidth}}
\typepar{src}{\_eps}{\_switch}  &   Make copies of events for
incorporation of attenuation uncertainty  i.e. spawning: \\
& \hspace{0.5em} 0 $\Rightarrow$ events NOT copied; \\
& \hspace{0.5em} 1 $\Rightarrow$ events copied: logic tree collapsed.  \\
& \hspace{0.5em} 2 $\Rightarrow$ events copied: logic tree NOT collapsed.  \\
\typepar{m}{bn}{d} &  Only events with magnitude greater than \typepar{m}{bn}{d} are spawned. \\
\typepar{nsa}{mpl}{es} & Creates \typepar{nsa}{mpl}{es} copies of each spawned event. \\
\typepar{n}{sig}{ma} &  Defines the x-limits of the probability density function to be considered when spawning.\\
 \end{supertabular}


\vspace{2em} \textbf{Scenario:}

\begin{supertabular}{lp{0.6\textwidth}}
\typepar{det}{erm}{\_flag} &  Event simulation type: \\
 & \hspace{0.5em} 0 $\Rightarrow$ standard probabilistic simulation; \\
 & \hspace{0.5em} 1 $\Rightarrow$ consider a specific scenario event. \\
\typepar{det}{erm}{\_ntrg} & The desired number of copies to be generated \\
 & \hspace{0.5em} (if $\typepar{det}{erm}{\_flag}=1$). \\
\typepar{det}{erm}{\_azi}  &   Azimuth of event    \\
 & \hspace{0.5em} (if $\typepar{det}{erm}{\_flag}=1$). \\
\typepar{det}{erm}{\_lat}  &  Latitude of rupture centroid \\
 & \hspace{0.5em} (if $\typepar{det}{erm}{\_flag}=1$). \\
\typepar{det}{erm}{\_lon}  &  Longitude of rupture centroid    \\
 & \hspace{0.5em} (if $\typepar{det}{erm}{\_flag}=1$). \\
\typepar{det}{erm}{\_mag}  &  Moment magnitude of event    \\
 & \hspace{0.5em} (if $\typepar{det}{erm}{\_flag}=1$). \\
\typepar{det}{erm}{\_r\_z}  &  Depth to event     \\
 & \hspace{0.5em} (if $\typepar{det}{erm}{\_flag}=1$). \\
 \end{supertabular}


\vspace{2em} \textbf{Attenuation:}

\begin{supertabular}{lp{0.6\textwidth}}
\typepar{atten}{uation}{\_flag}    &  A 2 row matrix containing one column for each attenuation model to be used.
The first row contains a pointer to the attenuation model (see below) and the second row contains its weight. \\
&  Attenuation Model Pointers: \\
 & \hspace{0.5em} 0 $\Rightarrow$ \cite{dr_Gaull90a}; \\
 & \hspace{0.5em} 1 $\Rightarrow$ \cite{dr_Toro97a}; \\
 & \hspace{0.5em} 2 $\Rightarrow$ \cite{dr_Atkinson97a}; \\
 & \hspace{0.5em} 3 $\Rightarrow$ \cite{dr_Sadigh97a}; \\
 & \hspace{0.5em} 4 $\Rightarrow$ \cite{dr_Somerville01a}; \\
 & \hspace{0.5em} 100 $\Rightarrow$ \cite{dr_Gaull90a} (MMI). \\
\typepar{attn}{\_reg}{ion}    &   Defines version of selected
attenuation model (i.e. region or type
 of magnitude). Options vary depending on the value of \typepar{atten}{uat}{ion\_flag} (see \cref{ch:atten}).  \\
\typepar{var}{\_attn}{\_flag}   &  Variability for attenuation: \\
 & \hspace{0.5em} 0 $\Rightarrow$ NOT included; \\
  & \hspace{0.5em}  1 $\Rightarrow$ included.   \\
\typepar{var}{\_attn}{\_method} & Technique used to
incorporate attenuation aleatory uncertainty: \\
 & \hspace{0.5em} 1 $\Rightarrow$ Sample PDF - spawning; \\
 & \hspace{0.5em} 2 $\Rightarrow$ random sampling; \\
 & \hspace{0.5em} 3 $\Rightarrow$ $+2\sigma$; \\
 & \hspace{0.5em} 4 $\Rightarrow$ $+\sigma$; \\
 & \hspace{0.5em} 5 $\Rightarrow$ $-\sigma$; \\
 & \hspace{0.5em} 6 $\Rightarrow$ $-2\sigma$.\\
\typepar{resp}{\_crv}{\_flag} & Secondary options for choice of
response spectral acceleration\index{response spectral acceleration} $S_a$: \\
 & \hspace{0.5em} 0 $\Rightarrow$ $S_a$ from attenuation model; \\
 & \hspace{0.5em} 1 $\Rightarrow$ $S_a$ from attenuation model and cutoff\\
 & \hspace{2.8em} maximum spectral displacement\index{response spectral displacement}; \\
 & \hspace{0.5em} 2 $\Rightarrow$ use PGA from attenuation model to scale\\
 & \hspace{2.8em} Australia Standard $S_a$; \\
 & \hspace{0.5em} 3 $\Rightarrow$ use PGA from attenuation model to scale \\
 & \hspace{2.8em} Australia Standard and cutoff at \\
 & \hspace{2.8em} maximum spectral displacement\index{response spectral displacement}; \\
 & \hspace{0.5em} 4 $\Rightarrow$ use PGA from attenuation model to scale \\
 & \hspace{2.8em} HAZUS $S_a$. \\
\typepar{R}{thr}{sh}  &  Threshold distance, beyond which motion is assigned to zero.       \\
\typepar{per}{io}{ds} &  Periods for $S_a$.     \\
\typepar{pga}{cut}{off}   & PGA cutoff for re-scaling $S_a$ with PGA greater than \typepar{pga}{cut}{off}.      \\
\typepar{smoothed}{\_response}{\_flag}  & Controls smoothing of $S_a$: \\
 & \hspace{0.5em} 0 $\Rightarrow$ No smoothing; \\
 & \hspace{0.5em} 1 $\Rightarrow$ Smoothing.  \\
 \end{supertabular}


\vspace{2em} \textbf{Amplification:}

\begin{supertabular}{lp{0.6\textwidth}}
\typepar{amp}{\_swit}{ch}  &  Amplification associated with local regolith: \\
 & \hspace{0.5em} 0 $\Rightarrow$ NOT included; \\
 & \hspace{0.5em}  1 $\Rightarrow$ included. \\
\typepar{var}{\_amp}{\_flag}    &   Variability for amplification: \\
 & \hspace{0.5em} 0 $\Rightarrow$ NOT included; \\
 & \hspace{0.5em} 1 $\Rightarrow$ included.  \\
\typepar{var}{\_amp}{\_method} & Technique used to incorporate amplification aleatory uncertainty: \\
 & \hspace{0.5em} 1 $\Rightarrow$ Sample PDF - spawning; \\
 & \hspace{0.5em} 2 $\Rightarrow$ random sampling; \\
 & \hspace{0.5em} 3 $\Rightarrow$ $+2\sigma$; \\
 & \hspace{0.5em} 4 $\Rightarrow$ $+\sigma$; \\
 & \hspace{0.5em} 5 $\Rightarrow$ $-\sigma$; \\
 & \hspace{0.5em} 6 $\Rightarrow$ $-2\sigma$.\\
\typepar{Max}{Amp}{Factor}    &   Maximum accepted value for amplification factor.   \\
\typepar{Min}{Amp}{Factor}    &   Minimum accepted value for amplification factor.    \\
 \end{supertabular}


\vspace{2em} \textbf{Bclasses:}

\begin{supertabular}{lp{0.6\textwidth}}
\typepar{b\_usage}{\_type}{\_flag} & Building usage classification system: \\
 & \hspace{0.5em} 1 $\Rightarrow$ HAZUS usage\index{building usgage!HAZUS} classification; \\
 & \hspace{0.5em} 2 $\Rightarrow$ FCB\index{building usgage!FCB} usage classification.\\
\typepar{hazus}{\_btypes}{\_flag}   & Building construction type classification system: \\
 & \hspace{0.5em} 0 $\Rightarrow$ refined HAZUS building classification\index{building type}; \\
 & \hspace{0.5em} 1 $\Rightarrow$ HAZUS building classification\index{building type}.     \\
\typepar{hazus}{\_damping}{is5\_flag}   &   Level of damping when \typepar{build}{pars}{\_flag} = 2: \\
 & \hspace{0.5em} 0 $\Rightarrow$ Use damping level corresponding to  \\
 & \hspace{2.8em} \typepar{build}{pars}{\_flag}=0; \\
 & \hspace{0.5em} 1 $\Rightarrow$ Use $5\%$.    \\
\typepar{build}{pars}{\_flag} & Engineering parameters to be used: \\
 & \hspace{0.5em} 0 $\Rightarrow$ Australian Engineers Workshop  \\
& \hspace{2.8em} Parameters (AEWP) with Edwards \\
& \hspace{2.8em}  modifications 1; \\
 & \hspace{0.5em} 1 $\Rightarrow$ AEWP; \\
 & \hspace{0.5em} 2 $\Rightarrow$ original HAZUS paramaters; \\
 & \hspace{0.5em} 3 $\Rightarrow$ AEWP with Edwards modifications 2; \\
 & \hspace{0.5em} 4 $\Rightarrow$ AEWP with Edwards modifications 3. \\
 \end{supertabular}


\vspace{2em} \textbf{Bclasses2:}

\begin{supertabular}{lp{0.6\textwidth}}
\typepar{force}{\_btype}{\_flag}    & Force all buildings to have the same usage and type classification: \\
 & \hspace{0.5em} 0 $\Rightarrow$ Use buildings as defined in database; \\
 & \hspace{0.5em} 1 $\Rightarrow$ Force modification. \\
\typepar{det}{erm}{\_btype}    &  Building type when \typepar{force}{\_btype}{\_flag}=1.    \\
\typepar{det}{erm}{\_buse} &   Building usage when \typepar{force}{\_btype}{\_flag}=1.     \\
\typepar{ignore}{\_post89}{\_flag}  &  Ignore buildings built after 1989: \\
 & \hspace{0.5em} 0 $\Rightarrow$ include all ages; \\
 & \hspace{0.5em} 1 $\Rightarrow$ ignore the post 89 buildings.           \\
 \end{supertabular}


\vspace{2em} \textbf{CSM:}

\begin{supertabular}{lp{0.6\textwidth}}
\typepar{var}{\_bcap}{\_flag}   &   Variability with building capacity curve\index{capacity curve}: \\
 & \hspace{0.5em} 0 $\Rightarrow$ NOT included; \\
 & \hspace{0.5em}  1 $\Rightarrow$ included.    \\
\typepar{bcap}{\_var}{\_method} & Method used to incorporate variability in capacity curve\index{capacity curve}: \\
 & \hspace{0.5em} 1 $\Rightarrow$ Random sampling used to adjust the \\
& \hspace{2.8em} vertical position of the yield and ultimate \\
& \hspace{2.8em} points; \\
 & \hspace{0.5em} 2 $\Rightarrow$ Preferred technique: Random sampling \\
& \hspace{2.8em} applied to ultimate point only and yield \\
& \hspace{2.8em}  point re-calculated to satisfy capacity \\
& \hspace{2.8em}  curve\index{capacity curve} `shape' constraint. \\

\typepar{st}{d}{cap}  & standard deviation for capacity curve\index{capacity curve} log--normal PDF.      \\
\typepar{damp}{\_fla}{gs}  & Vector to control nature of linear damping (see below for explanation of each element). \\
\typepar{damp}{\_fla}{gs}$(1)$ & Damping multiplicative formula to
be
used - \sref{subsec:v-dam-damping}: \\
 & \hspace{0.5em} 0 $\Rightarrow$ Preferred technique: use $R_a$, $R_v$, and $R_d$; \\
 & \hspace{0.5em} 1 $\Rightarrow$ use $R_a$, $R_v$ and assign $R_d= R_v$; \\
 & \hspace{0.5em} 2 $\Rightarrow$ use $R_v$ only and assign $R_a=R_d=R_v$. \\
\typepar{damp}{\_fla}{gs}$(2)$ & Modify transition building period i.e. corner period $T_{av}$: \\
 & \hspace{0.5em} 0 $\Rightarrow$ Preferred technique: modify as in HAZUS; \\
 & \hspace{0.5em} 1 $\Rightarrow$ do NOT modify. \\
\typepar{damp}{\_fla}{gs}$(3)$ & Smoothing of damped curve: \\
 & \hspace{0.5em} 0 $\Rightarrow$ Preferred technique: apply smoothing; \\
 & \hspace{0.5em} 1 $\Rightarrow$ NO smoothing.\\
\typepar{Har}{ea}{\_flag} & Technique for Hysteretic  damping: \\
 & \hspace{0.5em} 0 $\Rightarrow$ NO Hysteretic  damping; \\
 & \hspace{0.5em} 1 $\Rightarrow$ Hysteretic  damping via trapezoidal approximation; \\
 & \hspace{0.5em} 2 $\Rightarrow$ Hysteretic  damping via curve fitting.  \\
\typepar{SD}{Rel}{Tol}    &    Tolerance (as a percentage) for SDcr (in nonlinear damping calculations).   \\
\typepar{max}{\_iter}{ations}  & Maximum iterations for nonlinear damping calculations.\\
 \end{supertabular}

\vspace{2em} \textbf{Diagnostics:}

\begin{supertabular}{lp{0.6\textwidth}}
\typepar{qa}{\_switch}{\_ampfactors} & Diagnostics for \typefunc{do}{\_amp}{lification} (a few sites only): \\
 & \hspace{0.5em} 0 $\Rightarrow$ NO diagnostic plots; \\
 & \hspace{0.5em} 1 $\Rightarrow$ display diagnostic plots.      \\
\typepar{qa}{\_switch}{\_attn}  &   Diagnostics for \typefunc{do}{\_atten}{uation} (a few sites only): \\
& \hspace{0.5em} 0 $\Rightarrow$ NO diagnostic plots; \\
& \hspace{0.5em} 1 $\Rightarrow$ display diagnostic plots.  \\
\typepar{qa}{\_switch}{\_fuse}  &   Diagnostics for \typefunc{fuse}{\_4}{hzd}: \\
& \hspace{0.5em} 0 $\Rightarrow$ NO diagnostic plots; \\
& \hspace{0.5em} 1 $\Rightarrow$ display diagnostic plots.  \\
\typepar{qa}{\_switch}{\_map}   &  Diagnostics for \typefunc{do}{\_ana}{lysis} (works in hazard mode only): \\
& \hspace{0.5em} 0 $\Rightarrow$ NO diagnostic plots; \\
& \hspace{0.5em} 1 $\Rightarrow$ display diagnostic plots. \\
\typepar{qa\_switch}{\_mke}{\_evnts} &  Diagnostics for \typefunc{mke}{\_ev}{nts}: \\
& \hspace{0.5em} 0 $\Rightarrow$ NO diagnostics and save all events; \\
& \hspace{0.5em} 1 $\Rightarrow$ display diagnostic plots and ask user about \\
& \hspace{2.8em} saving; \\
& \hspace{0.5em} 2 $\Rightarrow$ display diagnostic plots and save events \\
& \hspace{2.8em} inside polygon; \\
& \hspace{0.5em} 3 $\Rightarrow$ display diagnostic plots and save all events.  \\
\typepar{qa\_switch}{\_water}{check}    & No longer used.      \\
\typepar{qa}{\_switch}{\_soc}   &  Diagnostics for casualty model: \\
& \hspace{0.5em} 0 $\Rightarrow$ NO diagnostics; \\
& \hspace{0.5em} 1 $\Rightarrow$ plot injuries and casualties.     \\
\typepar{qa}{\_switch}{\_vun}   &  Diagnostics for capacity spectrum method\index{capacity spectrum method}: \\
& \hspace{0.5em} 0 $\Rightarrow$ NO diagnostic plots; \\
& \hspace{0.5em} 1 $\Rightarrow$ display building info to screen; \\
& \hspace{0.5em} 2 $\Rightarrow$ Do $1$ and plot each building's convergence \\
& \hspace{2.8em} for the 1st earthquake (pause after \\
& \hspace{2.8em} every iteration); \\
& \hspace{0.5em} 3 $\Rightarrow$ Do $1$ and plot each building's convergence \\
& \hspace{2.8em} for the worst earthquake (pause after \\
& \hspace{2.8em} final convergence); \\
& \hspace{0.5em} 4 $\Rightarrow$ Do $1$ and plot each building's convergence \\
for the worst earthquake (pause after \\
& \hspace{2.8em} every iteration).    \\
 \end{supertabular}

\vspace{2em} \textbf{Loss:}

\begin{supertabular}{lp{0.6\textwidth}}
 \typepar{pga}{\_min}{damage}   &  minimum PGA(g) below which financial loss is assigned to zero. \\
 \typepar{c}{}{i} & Regional cost index multiplier to convert dollar values in building database\index{building database} to desired regional and temporal (i.e. inflation) values.\\
\typepar{aus}{\_contents}{\_flag}   &  Contents value for residential buildings and salvageability after complete building damage:   \\
& \hspace{0.5em} 0 $\Rightarrow$ contents value as defined in building \\
& \hspace{2.8em} database\index{building database} and salvageability of $50\%$;\\
& \hspace{0.5em} 1 $\Rightarrow$ $60\%$ contents value as defined in building \\
& \hspace{2.8em} database\index{building database} and salvageability of zero.\\
 \end{supertabular}

\vspace{2em} \textbf{Save:}

\begin{supertabular}{lp{0.6\textwidth}}
\typepar{hazard}{\_map}{\_flag} & Save data for hazard maps: \\
& \hspace{0.5em} 0 $\Rightarrow$ do NOT save; \\
& \hspace{0.5em} 1 $\Rightarrow$ save.      \\
\typepar{save}{\_ecloss}{\_flag}    & Save direct financial loss data: \\
& \hspace{0.5em} 0 $\Rightarrow$ do NOT save; \\
& \hspace{0.5em} 1 $\Rightarrow$ save total financial loss; \\
& \hspace{0.5em} 2 $\Rightarrow$ save contents and building loss separately.        \\
\typepar{save}{\_socloss}{\_flag}  & Save casualty and injury data: \\
& \hspace{0.5em} 0 $\Rightarrow$ do NOT save; \\
& \hspace{0.5em} 1 $\Rightarrow$ save.     \\
\typepar{save}{\_motion}{\_flag}   &  Save RSA motion when conducting a risk simulation: \\
& \hspace{0.5em} 0 $\Rightarrow$ do NOT save; \\
& \hspace{0.5em} 1 $\Rightarrow$ save.     \\
\typepar{save}{\_probdam}{\_flag}   &  Save damage state probabilities for each building (\textit{currently not operational}): \\
& \hspace{0.5em} 0 $\Rightarrow$ do NOT save; \\
& \hspace{0.5em} 1 $\Rightarrow$ save.      \\
\typepar{save}{\_deagecloss}{\_flag}   &  Save loss associated with individual damage types i.e. structural, non-structural (\textit{currently not operational}): \\
& \hspace{0.5em} 0 $\Rightarrow$ do NOT save; \\
& \hspace{0.5em} 1 $\Rightarrow$ save.     \\
 \end{supertabular}




\subsection{Parameter files}
\label{sec:application-parfiles}

This section identifies the important input files that are used by
different components of the EQRM. Input files can be dependent or
independent of region. For example the file
 \typeparfile{newc}{\_par}{\_ampfactors}{.mat} contains the amplification factors
for the Newcastle region (see \sref{ch:reg}), whereas the file
 \typeparfile{attn\_toro}{\_midcontinent}{\_momag}{.txt}
contains attenuation coefficients for the \citet*{dr_Toro97a}
attenuation model (see \sref{ch:atten}) which are not dependent on
region. The EQRM identifies the desired region dependent files
using the user defined parameter \typepar{site}{\_loc}{} (see
\sref{sec:application-setdata}).

The EQRM makes use of default data to reduce the need for a large
number of defined input files. Default data files are stored in
the directory \typedir{*/eqrm/eqrm\_app}{/resources}{/data} (see
\sref{sec:app-dirstruct}). The EQRM facilitates the use of
non-default data by first looking in the user defined
 \typepar{inp}{ut}{dir} (see \sref{sec:application-setdata}) for
required files. If the required files are found in
 \typepar{inp}{ut}{dir} they will be used by the EQRM, if the files
are not found in  \typepar{inp}{ut}{dir} then the default files
are used.

Note that default files are available for pre-defined regions
only. For example; Newcastle and Perth region default data can be
accessed by defining \typepar{site}{\_}{loc} of \typenewc and
\typeperth respectively.

\begin{table}
\caption{Default parameter files in
\typedircaption{*/eqrm/eqrm\_app}{/resources}{/data}.}
\label{tab:app-inputfiles} \vspace{0.8em}
\begin{tabular}{|l|p{0.7\textwidth}|}
\hline
Chapter & Files \\
\hline Earthquake & $\typeparfile{<site\_loc>}{\_par}{\_sourcepolys}{.txt}$ \\
Source generation &  $\typeparfile{<site\_loc>}{\_par}{\_sourcezones}{.txt}$ \\
\hline
Grids and & $\typeparfile{au}{s}{\_mat}{.txt}$\\
building & $\typeparfile{aust}{ralia}{\_bndy}{.txt}$\\
databases & $\typeparfile{<site\_loc>}{\_par}{\_site}{.mat}$\\
 &  $\typeparfile{<site\_loc>}{\_par\_site}{\_uniform}{.mat}$ \\
 &  $\typeparfile{<site\_loc>}{\_par\_study}{\_region\_box}{.txt}$ \\
  & $\typeparfile{sitedb}{\_<site}{\_loc>}{.mat}$ \\
 & $\typeparfile{suburb}{\_post}{code}{.csv}$\\
  & $\typeparfile{text}{b}{types}{.txt}$ \\
 & $\typeparfile{text}{busage}{FCB}{.txt}$ \\
 &  $\typeparfile{text}{b}{uses}{.txt}$ \\
\hline
Attenuation & $\typeparfile{attn}{\_atkboore}{\_momag}{.txt}$\\
 &  $\typeparfile{attn\_sadigh}{\_coeff\_momag}{\_great65}{.txt}$ \\
 &  $\typeparfile{attn\_sadigh}{\_coeff\_momag}{\_less65}{.txt}$ \\
 &  $\typeparfile{attn\_somer}{\_nonrift}{\_hori}{.txt}$ \\
 &  $\typeparfile{attn\_somer}{\_nonrift}{\_vert}{.txt}$ \\
 &  $\typeparfile{attn\_somer}{\_rift}{\_hori}{.txt}$ \\
 &  $\typeparfile{attn\_somer}{\_rift}{\_vert}{.txt}$ \\
 & $\typeparfile{attn\_toro}{\_midcontinent}{\_momag}{.txt}$,              \\
\hline
Regolith &  $\typeparfile{<site\_loc>}{\_par}{\_ampfactors}{.mat}$\\
amplification &  $\typeparfile{<site\_loc>\_par}{\_site\_class}{\_mask}{.mat}$ \\
 &  $\typeparfile{<site\_loc>\_par}{\_site\_class}{\_polys}{.mat}$ \\
\hline
Building & $\typeparfile{b}{factors}{db}{.mat}$\\
damage & $\typeparfile{bfactors}{db}{\_hazus}{.mat}$ \\
 & $\typeparfile{bfactors}{db}{\_wshop}{.mat}$ \\
 & $\typeparfile{bfactorsdb}{\_wshop}{\_update2}{.mat}$ \\
 & $\typeparfile{bfactorsdb}{\_wshop}{\_update3}{.mat}$ \\
 & $\typeparfile{factors}{-nonstruct}{dam}{.csv}$ \\
 &  $\typeparfile{factors}{-struct}{dam}{.csv}$ \\
\hline
Losses & $\typeparfilelong{rc\_perRepl}{CostwrtBuildC}{EdwardsFCBusage}{.mat}{rc\_perRepl...EdwardsFCBusage}$ \\
& $\typeparfilelong{rc\_perReplCost}{wrtBuildCEdwards}{Hazususage}{.mat}{rc\_perRepl...EdwardsHazususage}$\\
% & $\typeselfnoindex{rc\_perReplCost}{wrtBuildCEdwards}{Hazususage.mat}\index{\scriptsize\texttt{rc\_perReplCostwrtBuildCEdwardsHazususage.mat}\normalsize}$\\
& $\typeparfile{rc\_perReplCost}{wrtBuildC}{Hazusage}{.mat}$\\
& $\typeparfile{econ}{\_p}{ar}{.mat}$\\
\hline
\end{tabular}
\end{table}

\tref{tab:app-inputfiles} summarises the important input files
that can be found in the default data area. The files are listed
against the chapter where more information can be found.


\subsection{Running the EQRM in MATLAB}
\label{app:runEQRM}


The EQRM has a special startup function
(\typefunc{eqrm}{\_start}{up}) that adds all the required
directories to the MATLAB path. The following instructions
describe how to use \typefunc{eqrm}{\_start}{up} to set the MATLAB
path:
\begin{itemize}
\item move to the directory
\typedir{*/eqrm/eqrm\_app}{/m\_code}{/startup} \item at the
command line type \newline \texttt{>>
eqrm\_startup\indexfunc{eqrm\_startup}(`*$\backslash$eqrm$\backslash$eqrm\_app')}
\end{itemize}

The EQRM has a special GUI known as the
\typegui{eqrm}{\_param}{\_gui} that can be used to create a
\typeself{set}{da}{ta} file, run the EQRM application or analyse
the EQRM outputs. To launch the GUI simply type the following at
the command line:
\begin{itemize} \item[\texttt{>>}] \texttt{eqrm\_param\_gui}\indexgui{eqrm\_param\_gui} \end{itemize}
It is beyond the scope of this manual to discuss the
\typeself{eqrm}{\_param}{\_gui} further. However it is user
friendly and reasonably self explanatory. Alternatively the EQRM
can be run from the command line using the function
\typefunc{eqrm}{\_analy}{sis}. To find out how to use
\typefunc{eqrm}{\_analy}{sis} type the following at the MATLAB
command line: \begin{itemize} \item[\texttt{>>}] \texttt{help
eqrm\_analysis}
\end{itemize}

The EQRM has a special Output Manager GUI that can be used to
view, analyse and plot EQRM outputs. The Output Manager is also
known as the Data Interrogation Tool
(DIT)\index{DIT}\index{GUI!DIT}\index{Data Interrogation
Tool}\index{GUI!Data Interrogation Tool}. A report generation
facility has been included with the Output Manager and can be used
to create a \LaTeX \texttt{*.tex} file with EQRM output figures.
The \LaTeX file can be compiled into a \texttt{.ps} and/or
\texttt{.pdf} file if the appropriate MikTeX software is
installed.


\subsection{Compiling and using a stand--alone version of the EQRM in MATLAB R14SP1}

Using the MATLAB compiler it is possible to create a stand--alone
version of the EQRM that can be used without requiring MATLAB. The
instructions for MATLAB R13 are given in \aref{app:MATLABR13}.

\section{Creating a stand--alone executable of the EQRM GUI in MATLAB R14SP1:}
\label{app:MATLABR13}

Type the following at the MATLAB command line for the source
machine \begin{itemize} \item[\texttt{>>}]
\texttt{cnt\_do\_mcc\_build\_to\_system(`eqrm\_param\_gui',<*/path>)}
\end{itemize}

This process will create the following files in \texttt{<*/path>}:
\begin{enumerate}
\item \typeselfnoindexnoline{eqrm}{\_param}{\_gui.exe};  \item a
directory named \typeselfnoindexnoline{eqrm}{\_param}{\_gui\_mcr};
and \item \typeselfnoindexnoline{eqrm}{\_param}{\_gui.ctf}.
\end{enumerate}

\vspace{2em}

\textbf{Prepare the files for transportation as follows:}
\begin{enumerate} \item Create a folder entitled
\typeselfnoindexnoline{*/te}{st}{exe}. \item In
\typeselfnoindexnoline{*/te}{st}{exe} create two folders
 \typeselfnoindexnoline{*/}{testexe}{/system} and
 \typeselfnoindexnoline{*/}{testexe}{/mcr}.
\item Copy  \typeselfnoindexnoline{eqrm}{\_param}{\_gui.exe},
\typeselfnoindexnoline{eqrm}{\_param}{\_gui\_mcr} and
\typeselfnoindexnoline{eqrm}{\_param}{\_gui.ctf} (created above)
into \typeselfnoindexnoline{*/}{testexe}{/system}. \item Copy the
following files from \typeselfnoindexnoline{*/EQRM}{ROOT}{/system}
into \typeselfnoindex{*/}{testexe}{/system}:
\begin{enumerate}
\item \typeselfnoindexnoline{root\_mat}{\_creator}{\_cnt.exe};
\item the directory
\typeselfnoindexnoline{root\_mat}{\_creator}{\_cnt\_mcr}; and
\item \typeselfnoindexnoline{root\_mat}{\_creator}{\_cnt.ctf}.
\end{enumerate}
\item Copy \typeself{\_MCR}{Instal}{ler.exe} from
\typeselfnoindexnoline{<MATLABROOT>/toolbox}{/compiler/deploy}{/win32/MCRInstaller.exe}
into \typeselfnoindexnoline{*/}{testexe}{/mcr}. \item Copy the
directory \typeselfnoindexnoline{*/EQRM}{ROOT}{/resources} to
\typeselfnoindexnoline{*/}{testexe}{/resources}.
\end{enumerate}

\vspace{2em} \textbf{Installation onto the target machine is
achieved as follows:}
\begin{enumerate}
\item Move the contents of \typeselfnoindex{*/te}{st}{exe} onto
the target machine. \item Run \typeself{\_MCR}{Instal}{ler.exe} on
\typeselfnoindex{*/}{testexe}{/mcr} by double clicking on it.
\item Set environment path on the target machine
\begin{enumerate}
\item open the control panel; \item open system; \item select the
advanced tab; \item select environment variables; \item create new
user variable entitled \typeselfnoindex{p}{a}{th}; \item add the
following path names (separated by semi colons) into the value
field for \texttt{path};
\begin{itemize} \item
\typeselfnoindexnoline{*/}{testexe}{/system}; and \item
\typeselfnoindexnoline{*/testexe}{/mcr/v71}{runtime/win32};
\end{itemize}
\item Log off and log back on.
\end{enumerate}
\item Set the software path as follows
\begin{enumerate}
\item Open a DOS shell; \item Type
\typeselfnoindex{root}{\_mat\_creator}{\_cnt.exe}; \item On the
`EQRM root file creation' GUI select the EQRM root directory (i.e.
\typeselfnoindex{*/te}{st}{exe}).
\end{enumerate}
\item Open the EQRM GUI by typing
\typeselfnoindex{eqrm}{\_param}{\_gui.exe} at the DOS command
prompt.
\end{enumerate}

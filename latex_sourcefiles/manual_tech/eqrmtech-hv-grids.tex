\chapter{Grids and building database\index{building database}s}
\label{ch:grids}

\section{Overview}

The calculation of earthquake hazard and risk are spatial
problems. This chapter describes how grids are required to compute
earthquake hazard and how building databases\index{building
database} are represented to model earthquake risk.
Computationally the EQRM operates by looping over the grid
(building database\index{building database}) points one at a time
i.e. the hazard or risk is computed iteratively at each point.
Tools to assist the user create building databases\index{building
database} can be found in \typedir{*/eqrm}{/data}{cvt}.


\section{Hazard grids}

The term `hazard grid' merely refers to a set of spatially located
points at each of which earthquake hazard can be computed (see
\cref{ch:risk} for a definition of earthquake hazard). The set can
contain one or more points. Typically however, the grids are
uniform and regularly spaced in the horizontal and vertical
direction. The EQRM can work with two different hazard grid types,
the choice of which is controlled by the \typeself{set}{da}{ta}
parameter \typepar{grid}{\_}{flag}.

When $\typepar{grid}{\_}{flag}=1$ the EQRM loads the file
\typeparfile{<site\_loc>}{\_par}{\_site}{.mat} which contains the
following variables:
\begin{enumerate}
\item \typeselfnoline{Site}{Loc}{ations}: a double array
containing a row for each grid point and two columns. The first
and second columns contain the latitude and longitude of the grid
point respectively. \item \typeselfnoline{site}{\_clas}{ses}: a
character column array containing one row for each grid point. The
content of each row is a single letter identifying the regolith
site class of the grid point (see \cref{ch:reg}).
\end{enumerate}Typically such a grid originates in the GIS world where a grid of
points is created and assigned to site classes. This information
is imported into MATLAB where it is manipulated into the above
format and saved ready for use. Note that hazard values computed
on a \typeparfile{<site\_loc>}{\_par}{\_site}{.mat} must be
exported to the GIS environment for plotting (i.e. they can not be
plotted by the EQRM).

The \typeparfile{<site\_loc>}{\_par}{\_site}{.mat} is an older
format that has been kept for back compatibility. The preferred
technique for creating hazard grids is through the \typeim. The
\typeim is a GUI, specifically designed to create earthquake
hazard grids. It can be accessed through the
\typeself{eqrm}{\_param}{\_gui} by pressing the \typeim button.
The \typeim GUI allows the user to create a grid over a region of
interest. To do this the \typeim needs to access a polygon input
file that describes the region of interest as a single polygon
(e.g. \typeparfile{perth\_par}{\_study\_region}{\_box}{.txt}) or
as a set of site classes covering the study region (e.g.
\typeparfile{perth\_par}{\_site\_class}{\_polys}{.mat}). Note that
in the first case the EQRM will not access any information about
site classes and hence amplification can not be included. The user
must indicate how many points are desired in the latitude and
longitude directions. The name of the created grid file is
\typeparfile{<site\_loc>}{\_par\_site}{\_uniform}{.mat} and it can
be used with the EQRM by setting $\typepar{grid}{\_}{flag}=2$. Its
format is more complicated than the above mentioned
\typeparfile{<site\_loc>}{\_par}{\_site}{.mat}, however the user
does not need knowledge of the format since this is handled by the
\typeim. Note that earthquake hazard estimates produced on a
\typeparfile{<site\_loc>}{\_par\_site}{\_uniform}{.mat} file can
be plotted using the EQRM hazard plotting tools (see
\cref{ch:risk}).

\section{Building databases}
\label{sec:grids-bdatabase}

The building database\index{building database} contains attributes
for the portfolio of buildings under consideration in the risk
assessment. The building database\index{building database} is a
MATLAB data file with filename
\typeparfile{sitedb}{\_<site}{\_loc>}{.mat} containing the
following variables:
\begin{table}
\caption{Columns of the building database\index{building
database!columns} file
\typeparfilecaption{sitedb}{\_<site}{\_loc>}{.mat}.}
\vspace{0.8em} \label{tab:grids-bdatabasecolumns}
\begin{tabular}{|c|p{0.8\textwidth}|}
\hline
Column & Description \\
\hline
1 & Integer site identifier for EQRM (typically the same as column 10)\\
2 & Latitude of building \\
3 & Longitude of building \\
4 & Index to building construction type... \textbf{expanded} HAZUS list (\sref{sec:grids-constructionclass}) \\
5 & Index to HAZUS usage\index{building usgage!HAZUS} classification (\sref{sec:grids-usageclass}) \\
6 & Total floor area in square meters (summed over all stories)\\
7 & Survey factor indicating how many `real' buildings the database entry represents\\
8 & Index to suburb name for building (variable: \typeself{all}{\_sub}{urbs})\\
9 & Index to postcode name for building (variable: \typeself{all}{\_sub}{urbs})\\
10 & Integer site identifier for comparison against original database\\
11 & Logical index stating whether the building is pre- (0) or post- (1) the 1989 Newcastle earthquake\\
12 & Index to building construction type... HAZUS list (\sref{sec:grids-constructionclass})\\
13 & Replacement cost of building in dollars per square meter (\sref{sec:grids-replacecosts})\\
14 & Replacement cost of contents in dollars per square meter (\sref{sec:grids-replacecosts})\\
15 & Index to FCB\index{building usgage!FCB} usage classification (\sref{sec:grids-usageclass})\\
\hline
\end{tabular}
\end{table}

\begin{enumerate}
\item \typeselfnoline{all}{\_post}{codes}: a list of all postcodes
currently included in the EQRM for use in linking database
postcode id numbers. \item \typeselfnoline{all}{\_sub}{urbs}: a
list of all suburbs currently included in the EQRM for use in
linking database suburb id numbers. \item
\typeselfnoline{b}{\_site}{mat} a matrix containing one row for
each building and 15 columns describing attributes of the
building. Details of the columns are given in
\tref{tab:grids-bdatabasecolumns}. \item
\typeselfnoline{b}{\_so}{il}: a character column array containing
one row for each grid point. The contents of each row is a single
letter identifying the regolith site class of the grid point (see
\cref{ch:reg}).
\end{enumerate}
Typically the building database\index{building database} used with
the EQRM represents a subset of the true portfolio of interest.
When creating a database that sub-samples a larger portfolio,
individual database entries are used to represent more than one
`real' building. Such sub-sampling is undertaken to reduce run
times and memory requirements. Results from an EQRM simulation can
be re-converted to the full portfolio using the \typesf defined in
\typeparfile{sitedb}{\_<site}{\_loc>}{.mat} (see
\sref{sec:loss-surveyfacts}).

Tools to assist the creation of building database\index{building
database}s for use with the EQRM can be found in
\typedir{*/eqrm}{/datacvt}{/buildingdb}.


\subsection{Building construction types}
\label{sec:grids-constructionclass}

Buildings have been subdivided into a number of building
types\index{building type} each with their own set of building
parameters uniquely defining the median capacity
curve\index{capacity curve} and the random variability around the
median (see \cref{ch:damage}). The building construction types are
based upon the HAZUS definitions \citep{dr_FEMA99b}, with some
further subdivisions recommended by Australian engineers for
Australian building construction types \citep{dr_Stehle01a}.

In essence, the seven basic HAZUS types are
\begin{itemize}
\item Timber frame (W) \item Steel frame (S) \item Concrete frame
(C) \item Pre-cast concrete (PC) \item Reinforced masonry (R)
\item Unreinforced masonry (URM) \item Mobile homes (MH)
\end{itemize}
\begin{table}
 \caption{Definitions of the basic HAZUS building
construction types.} \vspace{0.8em} \label{tab:hazus-types}.
\centering
\begin{tabular}{|c|c|c|}
\hline
code & description & Stories\\
\hline
W1 & timber frame $<5\,000$ square feet & (1--2)\\
W2 & timber frame $> 5\,000$ square feet & (All)\\
\hline S1L &   & Low-Rise (1--3) \\
S1M & steel moment frame & Mid-Rise (4--7)\\
S1H & & High-Rise ($8+$)\\
\hline
S2L & & Low-Rise (1--3)\\
S2M & steel light frame & Mid-Rise (4--7)\\
S2M &  & High-Rise ($8+$)\\
\hline
S3 & steel frame + cast &  (All)\\
  & concrete shear walls  & \\
\hline
S4L & steel frame + & Low-Rise (1--3)\\
S4M & unreinforced masonry  & Mid-Rise (4--7)\\
S4H & in-fill walls& High-Rise ($8+$)\\
\hline
S5L & steel frame + & Low-Rise (1--3)\\
S5M & concrete shear & Mid-Rise (4--7)\\
S5H & walls  & High-Rise ($8+$)\\
\hline
C1L & & Low-Rise (1--3)\\
C1M &  concrete moment frame & Mid-Rise (4--7)\\
C1H & & High-Rise ($8+$)\\
\hline
C2L & & Low-Rise (1--3)\\
C2M & concrete shear walls  & Mid-Rise (4--7)\\
C2H & & High-Rise ($8+$)\\
\hline
C3L & concrete frame + & Low-Rise (1--3)\\
C3M & unreinforced masonry & Mid-Rise (4--7)\\
C3H & in-fill walls & High-Rise ($8+$)\\
\hline
PC1 & pre-cast concrete tilt-up walls & (All)\\
\hline
PC2L & pre-cast concrete  & Low-Rise (1--3)\\
PC2M &  frames with concrete  & Mid-Rise (4--7)\\
PC2H & shear walls & High-Rise ($8+$)\\
\hline
RM1L & reinforced masonry walls + & Low-Rise (1--3)\\
RM1M & wood or metal diaphragms  & Mid-Rise (4+)\\
\hline
RM2L & reinforced masonry & Low-Rise (1--3)\\
RM2M & walls + pre-cast & Mid-Rise (4--7)\\
RM2H & concrete diaphragms & High-Rise ($8+$)\\
\hline
URML & unreinforced & Low-Rise (1--2)\\
URMM & masonry & Mid-Rise (3+)\\
\hline
MH & Mobile homes & (All)\\
  \hline
\end{tabular}
\end{table}
There are further subdivisions of the HAZUS types into subtypes
according to numbers of stories in the building. These are given
in \tref{tab:hazus-types}.

The new Australian sub-types, developed in the Australian
engineers workshop, further subdivide some of the HAZUS types
\citep{dr_Stehle01a}. In particular, the timber frame category
(W1) is subdivided into wall types (timber or brick veneer walls)
and roof types (metal or tiled); the unreinforced masonry types
(URML and URMM) into roof type (metal, tile or  otherwise), and
the concrete frame types are subdivided into soft-story or
non-soft story types. Soft-story refers to buildings that may have
a concrete basement or parking area but wood frame stories.

In total, we currently have 56 possible construction types
although some are rarely used. For example; the original Hazus W1
is still there, however this is rarely used in favor of the more
detailed classification into W1TIMBMETAL, W1BVTILE, etc. A
complete list of all the building construction types is given in
\tref{tab:v-dam-allbtypes}. The \typeself{set}{da}{ta} flag
\typepar{hazus}{\_btypes}{\_flag} can be used to select between
the use of the HAZUS building type\index{building type}
classification and the Australian engineers extended HAZUS
building type\index{building type} classification.

\begin{table}[htp]
\centering \caption{Complete list of all building construction
types (with those that are rarely used in italics).The integers
corresponding to each building construction type represent the
integer index used in the building database\index{building
database} Column 4 for expanded HAZUS types (column 12 for HAZUS
only types).} \vspace{0.8em} \label{tab:v-dam-allbtypes}.
\begin{tabular}{|llll|}
 \hline
\it 1: W1 &   15: S5H &     29: RM1L &           43: C1LSOFT  \\
    2: W2 &   16: \it C1L & 30: RM1M &           44: C1LNOSOFT \\
    3: S1L &  17: \it C1M & 31: RM2L &           45: C1MMEAN \\
    4: S1M &  18: \it C1H & 32: RM2M          &  46: C1MSOFT \\
    5: S1H &  19: C2L &     33: RM2H         &   47: C1MNOSOFT \\
    6: S2L &  20: C2M &     34: \it URML      &  48: C1HMEAN    \\
    7: S2M &  21: C2H &     35: \it URMM     &   49: C1HSOFT \\
    8: S2H &  22: C3L &     36: MH &             50: C1HNOSOFT \\
    9: S3 &   23: C3M &     37: W1MEAN &         51: URMLMEAN \\
    10: S4L & 24: C3H &     38: W1BVTILE&        52: URMLTILE  \\
    11: S4M & 25: PC1 &     39: W1BVMETAL&       53: URMLMETAL \\
    12: S4H & 26: PC2L &    40: W1TIMBERTILE  &  54: URMMMEAN \\
    13: S5L & 27: PC2M &    41: W1TIMBERMETAL  & 55: URMMTILE  \\
    14: S5M & 28: PC2H &    42: C1MMEAN       &  56: URMMMETAL \\
 \hline
\end{tabular}
\end{table}

The building capacity curve\index{capacity curve}s used in the
Newcastle and Lake Macquarie study \citep{dr_Fulford02a} are
provided in \aref{app:buildingpars},
\tref{tab:vdam-building-parameters}. Other examples of the
building capacity curve\index{capacity curve}s, as well as tools
for manipulating them, can be found in the directory
\typedir{*/eqrm}{/datacvt}{buildingpars}.


\subsection{Building usage types}
\label{sec:grids-usageclass}

The cost models used by the EQRM require knowledge of the
building's use in society. For example the value of a factory's
contents will vary from the value of a residents house. Similarly,
the cost associated with building a hospital and the cost of
building a local shop may differ even if the same materials are
used because the buildings may be built to different standards. To
transfer this information to the EQRM the building
database\index{building database} must store information about
each building's usage. There are two different schemes that can be
used; the functional classification of building (FCB)
usage\index{building usgage!FCB} \citep{dr_ABS01a} and the HAZUS
usage\index{building usgage!HAZUS} classification
\citep{dr_FEMA99b}.


\begin{table}
\centering \caption{Functional classification of building
(FCB)\index{building usgage!FCB} \citep{dr_ABS01a} and integer
index used in the building database\index{building database}
column 15.} \vspace{0.8em} \label{tab:grids-FCB}  {\footnotesize .
\begin{tabular}{|c|p{0.8\textwidth}|}
 \hline
& \textbf{Residential: Separate, kit and transportable homes} \\
1 & 111: Separate Houses\\
2 & 112: Kit Houses\\
3 & 113: Transportable/relocatable homes\\
& \textbf{Residential: Semi-detached, row or terrace houses, townhouses}  \\
4 & 121: One storey\\
5 & 122: Two or more storeys\\
& \textbf{Residential: Flats, units or apartments} \\
6 & 131:  In a one or two storey block\\
7 & 132: In a three storey block\\
8 & 133:  In a four or more storey block\\
9 & 134:  Attached to a house\\
& \textbf{Residential: Other residential buildings} \\
10 & 191:  Residential: not otherwise classified \\
\hline
& \textbf{Commercial: Retail and wholesale trade building} \\
11 & 211:  Retail and wholesale trade buildings \\
& \textbf{Commercial: Transport buildings} \\
12 & 221:  Passenger transport buildings\\
13 & 222:  Non-passenger transport buildings\\
14 & 223:  Commercial carparks\\
15 & 224:  Transport: not otherwise classified\\
& \textbf{Commercial: Offices} \\
16 & 231:  Offices \\
& \textbf{Commercial: Other commercial buildings} \\
17 & 291:  Commercial: not otherwise classified\\
& \textbf{Industrial: Factories and other secondary production buildings} \\
18 & 311:  Factories and other secondary production buildings \\
& \textbf{Industrial: Warehouses} \\
19 & 321:  Warehouses (excluding produce storage) \\
& \textbf{Industrial: Agricultural and aquacultural buildings} \\
20 & 331:  Agricultural and aquacultural buildings \\
& \textbf{Industrial: Other industrial buildings} \\
21 & 391:  Industrial: not otherwise classified\\
\hline
& \textbf{Other Non-Residential: Education buildings} \\
22 & 411:  Education buildings \\
& \textbf{Other Non-Residential: Religion buildings} \\
23 & 421:  Religion buildings \\
& \textbf{Other Non-Residential: Aged care buildings} \\
24 & 431:  Aged care buildings \\
& \textbf{Other Non-Residential: Health facilities (not in 431)} \\
25 & 441:  Hospitals \\
26 & 442:  Health: not otherwise classified\\
& \textbf{Other Non-Residential: Entertainment and recreation buildings} \\
27 & 451:  Entertainment and recreation buildings \\
& \textbf{Other Non-Residential: Short term accommodation buildings} \\
28 & 461:  Self contained, short term apartments\\
29 & 462:  Hotels (predominately accommodation), motels, boarding houses, hostels or lodges\\
30 & 463:  Short Term: not otherwise classified\\
& \textbf{Other Non-Residential: Other non-residential buildings} \\
31 & 491:  Non-residential:not otherwise classified\\
\hline
\end{tabular}
}
\end{table}


\begin{table}
\centering \caption{HAZUS building usage
classification\index{building usgage!HAZUS} \citep{dr_FEMA99b} and
integer index used in the building database\index{building
database} column 5.} \label{tab:grids-HAZUSusage} \vspace{0.8em}
\begin{tabular}{|c|p{0.8\textwidth}|}
\hline
& \textbf{Residential}\\
1 & RES1: Single family dwelling (house)\\
2 & RES2: Mobile home\\
3 & RES3: Multi family dwelling (apartment/condominium) \\
4 & RES4: Temporary lodging (hotel/motel)\\
5 & RES5: Institutional dormitory (jails, group housing - military, colleges)\\
6 & RES6: Nursing home\\
\hline
& \textbf{Commercial}\\
7 & COM1: Retail trade (store)\\
8 & COM2: Wholesale trade (warehouse)\\
9 & COM3: Personal and repair services (service station, shop)\\
10 & COM4: Professional and technical services (offices)\\
11 & COM5: Banks\\
12 & COM6: Hospital\\
13 & COM7: Medical office and clinic\\
14 & COM8: Entertainment and recreation (restaurants, bars)\\
15 & COM9: Theaters\\
16 & COM10: Parking (garages)\\
\hline
& \textbf{Industrial}\\
17 & IND1: Heavy (factory)\\
18 & IND2: Light (factory)\\
19 & IND3: Food, drugs and chemicals (factory)\\
20 & IND4: Metals and mineral processing (factory)\\
21 & IND5: High technology (factory)\\
22 & IND6: Construction (office)\\
\hline
& \textbf{Agriculture}\\
23 & AGR1: Agriculture\\
\hline
& \textbf{Religion/Non/Profit}\\
24 & REL1: Church and non-profit\\
\hline
& \textbf{Government}\\
25 & GOV1: General services (office)\\
26 & GOV2: Emergency response (police, fire station, EOC)\\
\hline
& \textbf{Education}\\
27 & EDU1: Grade schools\\
28 & EDU2: Colleges and Universities (not group housing)\\
\hline
\end{tabular}
\end{table}

The FCB\index{building usgage!FCB} usage is summarised in
\tref{tab:grids-FCB} and the HAZUS usage
classification\index{building usgage!HAZUS} is summarised in
\tref{tab:grids-HAZUSusage}. The \typeself{set}{da}{ta} flag
\typepar{b\_usage}{\_type}{\_flag} can be used to switch between
the two usage classifications.


\subsection{Replacement costs}
\label{sec:grids-replacecosts}


The replacement cost in dollars per square metre for each building
and the replacement cost of the contents of each building are
contained within the building database\index{building database}
(columns 13 and 14 respectively: see \sref{sec:grids-bdatabase}).
Typically these costs are a function of the usage classification
of the building and are hence also dependent on whether the
HAZUS\index{building usgage!HAZUS} or FCB\index{building
usgage!FCB} classification system is used. The EQRM does not cross
check how the costings were created. Note that in some instances
it may be appropriate to use costings created from one usage
classification with the EQRM using the other usage mode (effects
cost splits - see below) and in some instance it may not be
appropriate to do so. Users are encouraged to familiarise
themselves with database metadata to ensure that they are using
the EQRM appropriately for their own application.

\clearpage
\section{Key functions, flags and parameters}

\begin{tabular}{llp{0.5\textwidth}}
\hline
\textbf{Name} & \textbf{Type} & \textbf{Description} \\
\hline \vspace{0.5em}
\keyrowsep \splitrowdir{*/datacvt}{/buildingpars} & Dir & Tools for manipulating and preparing engineering parameter files.\\
\keyrowsep \splitrowdir{*/datacvt}{/buildingdb}  & Dir & Tools for manipulating and preparing building database\index{building database}s.\\
\keyrowsep \splitrowdir{*/datacvt}{/econsoclosspars}  & Dir & Tools for manipulating and preparing costing split ratios.\\
\keyrowsep \typeim  & GUI & Tool for creating earthquake hazard grids.\\
\keyrowsep \typeself{eqrm}{\_param}{\_gui} & GUI & Main EQRM GUI.\\
\keyrowsep \splitrowparfile{<site\_loc>\_par\_study}{\_region\_box}{.txt}& Par File & Describes study region with a box - can be used to generate hazard grids.\\
\keyrowsep \splitrowparfile{<site\_loc>\_par\_site}{\_class\_polys}{.mat}& Par File & Describes study region as a set of site classes  - can be used to generate hazard grids.\\
\keyrowsep \splitrowparfile{<site\_loc>\_par\_site}{\_uniform}{.mat} & Par File & Earthquake hazard grid.\\
\keyrowsep \typeparfile{sitedb}{\_<site}{\_loc>}{.mat}  & Par File & Building database.\\
\keyrowsep \typepar{b\_usage}{\_type}{\_flag}  & Par & Flag for selecting the usage classification system. \\
\keyrowsep \typepar{hazus}{\_btypes}{\_flag} & Par & Flag for selecting the building construction type classification system. \\
\keyrowsep \typepar{build}{pars}{\_flag} & Par & Flag for selecting engineering parameters.\\
\keyrowsep \typepar{grid}{\_}{flag} & Par & Flag for selecting the type of earthquake hazard grid to be used.\\
  \hline
\end{tabular}

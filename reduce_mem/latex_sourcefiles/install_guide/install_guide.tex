% Use pdflatex to compile.
\documentclass[a4paper, 12pt]{article}

\usepackage{amsmath}
\usepackage{graphicx}
\usepackage{supertabular}
\usepackage{hyperref}

\input ../manual_tech/eqrmtech-macros  % load in a bunch of purpose built macros


% A couple of changes to the formatting
\setlength{\parindent}{0in}  % remove paragraph indenting
\setlength{\parskip}{1em} % put a blank line between paragraphs
%\value{10}
% \setcounter{counter}{value}




\makeatletter
\def\maketitle{%
\null
\thispagestyle{empty}%
\hrule height 0.5pt \vskip 2.5cm
\begin{center}
\normalfont
{\Large \textbf\@title\par}%
\vskip 2.5cm \normalsize \textbf{EQRM Development Team} \vskip 2.5cm
{\normalsize \textbf \@author\par}%
\vskip 4.53cm
{\normalsize \textbf \@date\par}%
\vskip 3.5cm \hrule height 0.5pt
\end{center}%
\null \clearpage }

\makeatother
\title{EQRM Install Guide}
\author{Geoscience Australia}



\begin{document}

\maketitle

\section{Introduction}

The EQRM builds on existing open source packages and it uses
multiple languages (C and Python) to obtain an optimal trade-off
between useability and speed. For these two reasons the installation
process contains multiple steps.

Currently EQRM can be installed on the Windows and Linux operating
system. Installation on the Mac OS has not been tested.

For any help, please post to:
\begin{itemize}
\item eqrm-user@lists.sourceforge.net; or
\item \url{https://sourceforge.net/mail/?group_id=198293}.
\end{itemize}
Also, to keep up to date with EQRM developments, join the developer
or user lists.

We would also like to hear from you if you are not able to install
the software.

EQRM is currently dependant on;
\begin{itemize}
\item python;
\item numpy;
\item scipy;
\item Shapely; and
\item gcc (MinGW on Windows)
\item Matplotlib (Plotting utilities only)
\item GMT (Plotting utilities only)
\end{itemize}

The above requirements are for the 'core' of EQRM. The plotting
functions require one or more of the following: matplotlib GMT

\begin{supertabular}{|p{1\textwidth}|}
\hline  \\
 Note: when \texttt{*/eqrm\_core/eqrm\_code} is mentioned
below you need to replace the * with the actual place you have
installed EQRM on your computer.  \\
\\ \hline
\end{supertabular}

\section{Getting a copy of the EQRM}

The most stable release of the EQRM can be downloaded from \\
\url{http://sourceforge.net/projects/eqrm/}.

If you are interested in the bleeding edge version then please
contact the developers at eqrm-user@lists.sourceforge.net

\section{Instalation}

This section deals with the installation of the open source packages
and the setup of paths to get the EQRM running. More packages are
required for the plotting tools (\sref{sec:-plotting-tools}) and to
analyse the test coverage (\sref{sec:test-coverage}).

\begin{supertabular}{|p{1\textwidth}|}
\hline  \\
The EQRM can be sensitive to the versions chosen. We recomend that you select the versions indicated below.  \\
\\ \hline
\end{supertabular}


\subsection{WINDOWS INSTALL INSTRUCTIONS:}

%%%%% NOTE %%%%%
% Reference to Python(xy) removed because it is difficult to find the
% required version.....

%There are two options for installing the EQRM and the required open
%source packages in Windows. Option 1 (\sref{sec:windows-option1})
%involves installing each package individually. Option 2
%(\sref{sec:windows-option2}) uses Python(x,y) which conveniently
%packages many of the required packages.

%\subsubsection{Option 1 - Installing all open source packages individually}
\label{sec:windows-option1}

\begin{enumerate}

\item Install python 2.5 from \\
\url{http://www.python.org/download/}. \\
Note that the EQRM is not compatable with Python 3$+$ \\

\item Install numpy from \\
  \url{http://numpy.scipy.org/} \\
  Use version 1.3.0 or later \\

\item Install scipy from \\
 \url{http://scipy.org}
 Use version 0.7.1 or later \\

\item Install the full package of MinGW.
When installing using \\
\texttt{mingw-5.X.X.exe}, you are
asked to choose components.  \\
\underline{Tick the g++ compiler} and do not untick any components. \\
 \url{http://www.mingw.org/}

\item Install Shapely 1.X.X (python GIS kit) from \\
\url{http://pypi.python.org/pypi/Shapely/.}

\item Add \texttt{C:/MinGW/bin} (so you can compile c extensions
to python) and \texttt{C:/Python25} (so you can call Python from a
DOS prompt) to your \underline{``path"} and \texttt{*/eqrm\_core} to
your \underline{``pythonpath"}:
\begin{enumerate}

\item right click My Computer - go to properties - Advanced - Environment Variables

\item If a variable called \underline{``path"} already exists in User
Variables: \\
select path then ``press Edit..." \\
otherwise press "New..." \\

\item Variable Name = PATH, \\
add \texttt{"C:/MinGW/bin;C:/Python25;}" to  Variable Value \\
(use ";" to separate values).

\item Variable Name = PYTHONPATH, \\
add \texttt{"*/eqrm\_core;}" to Variable Value \\
(use ";" to separate values).
\end{enumerate}

\item At the DOS prompt: \\
\texttt{cd */eqrm\_core} \\
\texttt{python test\_all.py} \\
\texttt{python check\_scenarios.py} \\
If you get a Could not locate executable g++ error, it means MinGW
is not working.  Check steps 9 and 10.
\end{enumerate}
%
%-----------------------------------------------------------------------------
%
%
%\subsubsection{Option 2 - Installing python(x,y)}:
%\label{sec:windows-option2}

%\begin{enumerate}
%\item Install python(x,y) \\
%\url{http://www.pythonxy.com/}
%\item Repeat steps 5 to 7 from \sref{sec:windows-option1}
%\end{enumerate}


\subsection{GENERAL LINUX INSTALL INSTRUCTIONS}

There are a variety of Linux distributions.  The best instructions
for installing each of the necessary packages is found at the
website of each package, so only general instructions are given
here.

Install;
\begin{enumerate}
\item python 2.5 \\
\url{http://www.python.org/download/} \\
Note that the EQRM is not compatable with Python 3$+$ \\
\item python-dev
\item python package numpy \\
\url{http://numpy.scipy.org/}
\item python package scipy \\
\url{http://www.scipy.org/}
\item \texttt{python package libgeos\_c} (for shapely)
\item python package \texttt{python-setuptolls} \\
(for installing shapely using \texttt{easy\_install})
\item python package \texttt{shapely}  (see notes below) \\
\url{http://pypi.python.org/pypi/Shapely/}
\item \texttt{gcc} also needs to be installed. Using versions higher than 4.2
have caused problems with weave.
\end{enumerate}

Unzip the EQRM install file (e.g.
\texttt{eqrm\_versionX.X.svn\_X.zip}) to where ever you want it.

Set the environment variable PYTHONPATH to \texttt{*/eqrm\_core}
(where the EQRM software was installed.)

Run the following in the \texttt{*/eqrm\_core} directory:
\begin{enumerate}
\item \texttt{python test\_all.py} \\
to test that individual parts of the code work
\item \texttt{python check\_scenarios.py} \\
to run some scenarios and check the output
\end{enumerate}

6 March 2009 - Installed EQRM on a debian system and
\texttt{test\_all} failed due to the inline compiler (weave)
failing.  To fix I get g++ to work with g$++$v4.2, instead of
g$++$v4.3.  Weave will also fail if python-dev is not installed.


\textbf{Notes for installing Shapely on a Linux system:} \\

Please note that shapely is not produced by us so the install
process may vary over time.
\begin{enumerate}

\item download and install \texttt{ez\_setup.py} - need to be root \\
\scriptsize
\url{http://peak.telecommunity.com/DevCenter/EasyInstall#installing-easy-install}
\normalsize

\item download and install the GIS library (i.e.
\texttt{libgeos\_c}) \\
To install on debian \\
\texttt{apt-get install libgeos-c1}\\
 \\
To install on Suse (need to be root) \\
download geos (e.g. \texttt{geos-3.0.0.tar.bz2}) \\
from \url{http://trac.osgeo.org/geos/} \\
 \\
\texttt{tar xvfj geos-3.0.0.tar.bz2} \\
\texttt{cd geos-3.0.0} \\
\texttt{./configure} \\
\texttt{./make} \\
\texttt{./make install} \\
\texttt{ldconfig}

\item install shapely \\
\url{http://pypi.python.org/pypi/Shapely} \\
\texttt{sudo easy\_install Shapely}

\end{enumerate}

\section{Setting up for Plotting Tools} \label{sec:-plotting-tools}


To install the components required for any plotting you may require
matplotlib (xy plotting, histograms etc.) and GMT (maps). Follow
these steps:

\begin{enumerate}
\item Install matplotlib from \\
\url{http://sourceforge.net/projects/matplotlib/]} \\
You must choose the file to install that matches your operating
system 'type' and python version. For example, the matplotlib
install file \texttt{matplotlib-0.98.5.2.win32-py2.5.exe} is meant
for a Win32 system running python 2.5.

\item If you require any visualisation with a map in it, install GMT
from \\
\url{http://gmt.soest.hawaii.edu/gmt/gmt_windows.html} \\
You will need the basic GMT distribution and the High and Full
resolution GSHHS coastlines installs.  Get these from \\
\url{http://gmt.soest.hawaii.edu/} \\
by clicking on the 'DOWNLOAD' link and then clicking on the 'INSTALL
FORM' link.  You choose many options on the form - if you do not
understand a choice it is usually safe to leave the default choice,
with the exception of the NetCDF question. Unless you are sure you
already have NetCDF 3.6.x installed, always ask for NetCDF to be
installed.  You should also select the high-resolution and
full-resolution coastline data installs. \\
\\
Note that you can install GMT anywhere you wish, though you may
choose to use the suggested directories (such as /usr/local, for
example). If you choose a 'system' location you will need to log in
as administrator.

\end{enumerate}


\section{Setting up for Test Coverage} \label{sec:test-coverage}

For developers running using the SVN repository:
\begin{enumerate}
\item If you want to run the coverage scrips(\texttt{coverage\_of\_tests.py}) the
python coverage module needs to be installed.  To do this go to; \\
\url{http://www.nedbatchelder.com/code/modules/coverage.html} and
follow the install instructions. Note, if the file is downloaded as
\\ \texttt{coverage-2.77.tar.tar}, change it to
\texttt{coverage-2.77.tar.gz}
\end{enumerate}

%
\section{General Notes}

Scipy is dependent on numpy, so numpy must be installed before
scipy, and the two versions must be compatible.

To find out the version of numpy, execute the following at the Python prompt \\
\texttt{import numpy} \\
\texttt{numpy.version.version}

To find out the version of scipy, execute the following at the Python prompt \\
\texttt{import scipy} \\
\texttt{scipy.version.version}

You may be able to install \texttt{libgeos\_c} by \\
\texttt{yum install libgeos-devel}, or \\
\texttt{yum install geos-devel}




\end{document}
